\documentclass[french, a4paper, 11pt]{article}

\usepackage[utf8]{inputenc} % ~ Encodage
\usepackage[T1]{fontenc}    % ~ Encodage
\usepackage[left=1cm, right=1cm]{geometry} % ~ Mise en page et marges
\usepackage{amssymb} % ~ Pour écrire les maths
\usepackage{xspace}  % ~ Commandes à texte
\usepackage{varioref} % ~ Références croisées
\usepackage{enumitem} % ~ Listes
\usepackage{xcolor}   % ~ Couleurs fs
\usepackage{float}
\usepackage{graphicx}
\usepackage[f]{esvect}
\usepackage[many]{tcolorbox}
\usepackage{euler}
\usepackage[nointegrals]{wasysym}
\usepackage[french]{babel}


%______FONCTIONS______%
\newcommand{\ssi}{si et seulement si\xspace}		% ~ ssi
\newcommand{\inv}[1]{\dfrac{1}{#1}}
% --ENSEMBLES--%
\newcommand{\N}{\mathbb{N}}   % ~ Entiers naturels
\newcommand{\Z}{\mathbb{Z}}   % ~ Entiers relatifs
\newcommand{\D}{\mathbb{D}}   % ~ Decimaux
\newcommand{\Q}{\mathbb{Q}}   % ~ Rationnels
\newcommand{\R}{\mathbb{R}}   % ~ Réels
\newcommand{\C}{\mathbb{C}}   % ~ Complexes
% --TRIGO--%
\let\cosh\relax
\DeclareMathOperator{\cosh}{ch}       % ~ cosinus hyperbolique
\DeclareMathOperator{\sh}{sh}         % ~ sinus hyperbolique
\let\tanh\relax
\DeclareMathOperator{\tanh}{th}       % ~ tangente hyperbolique
\DeclareMathOperator{\argch}{Argch}   % ~ Argument cosinus hyperbolique
\DeclareMathOperator{\argsh}{Argsh}   % ~ Argument sinus hyperbolique
\DeclareMathOperator{\argth}{Argth}   % ~ Argument tangente hyperbolique
\DeclareMathOperator{\cotan}{cotan}   % ~ cotangente
% --PARENTHESES--%
\newcommand{\po}{\left(}         % ~ (
\newcommand{\pf}{\right)}        % ~ )
\newcommand{\pof}[1]{\po #1 \pf} % ~ ( )
\newcommand{\co}{\left[}         % ~ [
\newcommand{\cf}{\right]}        % ~ ]
\newcommand{\cof}[1]{\co #1 \cf} % ~ [ ]
\newcommand{\chof}[1]{\left\langle #1 \right\rangle } % ~ < >
\newcommand{\interoo}[2]{\left]#1\,;#2\right[}   % ~ ]a,b[
\newcommand{\interof}[2]{\left]#1\,;#2\right]}   % ~ ]a,b]
\newcommand{\interfo}[2]{\left[#1\,;#2\right[}   % ~ [a,b[
\newcommand{\interff}[2]{\left[#1\,;#2\right]}   % ~ [a,b]

\newtcolorbox{cadre}[2][]
{
  enhanced,
  attach boxed title to top left={yshift=-3.4mm, xshift = -2.3mm},
  adjusted title=#2,
  colback=white, colframe=black,
  colbacktitle=white, coltitle=black, fonttitle=\bfseries,
  breakable, sharp corners,
  boxed title style={colback=white, sharp corners, colframe=white},
  boxrule = 0.5mm, drop fuzzy shadow
}


\newcommand{\ooint}{\ocircle\hspace{-3.65mm}\int\hspace{-2mm}\int}

\title{Exercices Séries}
\author{Nathan \textsc{Maillet}}
\date{}

\begin{document}
\maketitle
\begin{cadre}{Analyse réelle}
    Soit \[S=\left\{(u_n)\in (\R^+)^{\N} / \sum_{n=0}^{+\infty}u_n=1\right\}.\]
    Pour \(u \in S\), montrer que
      \[\phi(u)=\sum_{n=0}^{\infty}\pof{u_n\sum_{k=0}^n u_k} \]
      est bien défini. Déterminer la borne inférieure des \(\phi(u)\) quand $u$ décrit $S$.
\end{cadre}

\begin{cadre}{Caclul de sommes}
  Calculer les sommes des séries de termes généraux :
  \[\alpha)\frac{(-1)^n}{3n+1} \hspace*{1cm} \beta)\frac{E(\sqrt{n-1})-E(\sqrt{n})}{n} \hspace*{1cm} \gamma)\frac{4n}{n^4+2n^2+9}\]
\end{cadre}

\begin{cadre}{Nature d'une série}
Soit $(u_{n})_{n \in \N} $ une suite croissante de réels strictement positifs. Etudier la nature de $\sum_{n \ge 0}^{} \frac{u_{n+1}-u_{n}}{u_{n}}$.
\tcblower
\paragraph*{}
L'hypothèse de croissance dans les exercices est très utile car elle nous assure l'existence d'une limite en l'infini. Nous sommes donc amenés à faire une disjonction de cas, selon que la suite soit bornée ou non. De plus, il faut remarquer que le terme de général de la série est positif, ce qui nous donne la possibilité d'utiliser tous les outils de comparaison.

\begin{itemize}[label=$\bullet$]
\item Si $(u_n)$ est majorée, la suite converge vers une valeur réelle $l$. On a donc, en plus l'infini:
\[\dfrac{u_{n+1}-u_{n}}{u_{n}}\sim \dfrac{u_{n+1}-u_{n}}{l}\] qui est un terme général de série convergente, la série étudiée est donc convergente.

\item Si $(u_n)$ n'est pas majorée, elle tend vers $+\infty$ en $+\infty$. On a pour $n \in \N$, par croissance de $(u_n)$:
    \[
    \dfrac{u_{n+1}-u_{n}}{u_{n}}=\displaystyle \int_{u_n}^{u_{n+1}} \dfrac{1}{u_n} \, \mathrm{d}t \ge \displaystyle \int_{u_n}^{u_{n+1}} \dfrac{1}{t} \, \mathrm{d}t =\ln(u_{n+1})-\ln(u_n)
    \]
La série est donc dans ce cas divergente.
\end{itemize}
\end{cadre}

\end{document}
