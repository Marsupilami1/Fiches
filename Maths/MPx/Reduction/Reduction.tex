\documentclass[french, a4paper, 11pt, twocolumn]{article}

\usepackage[utf8]{inputenc} % ~ Encodage
\usepackage[T1]{fontenc}    % ~ Encodage
\usepackage[left=1cm, right=1cm]{geometry} % ~ Mise en page et marges
\usepackage{amssymb} % ~ Pour écrire les maths
\usepackage{xspace}  % ~ Commandes à texte
\usepackage{varioref} % ~ Références croisées
\usepackage{enumitem} % ~ Listes
\usepackage{xcolor}   % ~ Couleurs fs
\usepackage{float}
\usepackage{tikz}
\usepackage[straightvoltages]{circuitikz}
\usepackage[squaren, cdot, derived]{SIunits}
\usepackage{graphicx}
\usepackage[f]{esvect}
\usepackage[many]{tcolorbox}
\usepackage{euler}
\usepackage[nointegrals]{wasysym}
\usepackage[french]{babel}


%______FONCTIONS______%
\newcommand{\ssi}{si et seulement si\xspace}		% ~ ssi
\newcommand{\inv}[1]{\dfrac{1}{#1}}
\newcommand{\bigslant}[2]{{\raisebox{.2em}{\(#1\)}\left/\raisebox{-.2em}{\(#2\)}\right.}}
\DeclareMathOperator{\Sp}{Sp}
% --ENSEMBLES--%
\newcommand{\N}{\mathbb{N}}   % ~ Entiers naturels
\newcommand{\Z}{\mathbb{Z}}   % ~ Entiers relatifs
\newcommand{\D}{\mathbb{D}}   % ~ Decimaux
\newcommand{\Q}{\mathbb{Q}}   % ~ Rationnels
\newcommand{\R}{\mathbb{R}}   % ~ Réels
\newcommand{\C}{\mathbb{C}}   % ~ Complexes
% --TRIGO--%
\let\cosh\relax
\DeclareMathOperator{\cosh}{ch}       % ~ cosinus hyperbolique
\DeclareMathOperator{\sh}{sh}         % ~ sinus hyperbolique
\let\tanh\relax
\DeclareMathOperator{\tanh}{th}       % ~ tangente hyperbolique
\DeclareMathOperator{\argch}{Argch}   % ~ Argument cosinus hyperbolique
\DeclareMathOperator{\argsh}{Argsh}   % ~ Argument sinus hyperbolique
\DeclareMathOperator{\argth}{Argth}   % ~ Argument tangente hyperbolique
\DeclareMathOperator{\cotan}{cotan}   % ~ cotangente
% --PARENTHESES--%
\newcommand{\po}{\left(}         % ~ (
\newcommand{\pf}{\right)}        % ~ )
\newcommand{\pof}[1]{\po #1 \pf} % ~ ( )
\newcommand{\co}{\left[}         % ~ [
\newcommand{\cf}{\right]}        % ~ ]
\newcommand{\cof}[1]{\co #1 \cf} % ~ [ ]
\newcommand{\chof}[1]{\left\langle #1 \right\rangle } % ~ < >
\newcommand{\interoo}[2]{\left]#1\,;#2\right[}   % ~ ]a,b[
\newcommand{\interof}[2]{\left]#1\,;#2\right]}   % ~ ]a,b]
\newcommand{\interfo}[2]{\left[#1\,;#2\right[}   % ~ [a,b[
\newcommand{\interff}[2]{\left[#1\,;#2\right]}   % ~ [a,b]
% --VECTEURS--%
\newcommand{\ux}{\vect{u_x}}          % ~ Vecteur ux
\newcommand{\uy}{\vect{u_y}}          % ~ Vecteur uy
\newcommand{\uz}{\vect{u_z}}          % ~ Vecteur uz
\newcommand{\ur}{\vect{u_r}}          % ~ Vecteur ur
\newcommand{\uth}{\vect{u_\theta}}    % ~ Vecteur utheta
\newcommand{\uph}{\vect{u_\varphi}}   % ~ Vecteur uphi
\newcommand{\om}{\vect{OM}}           % ~ Vecteur position
\newcommand{\vvi}{\vect{v}}           % ~ Vecteur vitesse
\newcommand{\vvio}{\vect{v_0}}        % ~ Vecteur v0
\newcommand{\va}{\vect{a}}            % ~ Vecteur	accélération
\newcommand{\vp}{\vect{p}}            % ~ Vecteur quantité de mouvement
\newcommand{\fr}{\vect{F_r}}          % ~ Vecteur force de rappel
\newcommand{\vabla}{\vect{\nabla}}    % ~ nabla
\newcommand{\grad}{\vect{\mathrm{grad}}}  % ~ grad
\DeclareMathOperator{\diverg}{div}        % ~ grad
\newcommand{\rot}{\vect{\mathrm{rot}}}    % ~ grad

\newtcolorbox{theoreme}[2][]
{
  enhanced,
  attach boxed title to top left={yshift=-3.4mm, xshift = -2.3mm},
  adjusted title=#2,
  colback=white, colframe=black,
  colbacktitle=white, coltitle=black, fonttitle=\bfseries,
  breakable, sharp corners,
  boxed title style={colback=white, sharp corners, colframe=white},
  boxrule = 0.5mm, drop fuzzy shadow
}

\newtcolorbox{definition}[1][]
{
  enhanced,
  attach boxed title to top left={yshift=-3.4mm, xshift = -2.3mm},
  adjusted title=Définition,
  colback=white, colframe=black,
  colbacktitle=white, coltitle=black, fonttitle=\bfseries,
  breakable, sharp corners,
  boxed title style={colback=white, sharp corners, colframe=white},
  boxrule = 0.5mm, drop fuzzy shadow
}


\newcommand{\ooint}{\ocircle\hspace{-3.65mm}\int\hspace{-2mm}\int}

\title{Réduction}
\author{Martin \textsc{Andrieux}}
\date{}

\begin{document}
\maketitle

\section{Éléments propres}
\begin{definition}
  Soit \(f\) dans \(\mathcal L(E)\), \(x\) est un \emph{vecteur propre} pour \(f\) si \(x\neq 0\) et si \(f(x)\in \mathrm{Vect}(x)\).
  Si \(x\) est un vecteur propre pour \(f\), il existe un unique \(\lambda\) dans \(K\) tel que \(f(x) = \lambda x\).

  On note \(\Sp(f)\) l'ensemble des valeurs propres de \(f\), appelé \emph{spectre} de \(f\).
\end{definition}

\begin{definition}
  On pose \(\chi_{f}(\lambda) = \det\pof{\lambda\cdot\mathrm{Id} - f}\). \(\chi_{f}\) est appelé \emph{polynôme caractéristique} de \(f\). Avec \(A\) la matrice de \(f\), \(\chi_{A}\) est de la forme suivante:
  \[\chi_{f}(\lambda) = \lambda^{n}-\mathrm{tr}(A)\cdot\lambda^{n-1}+\cdots+\pof{-1}^{n}\det A\]
\end{definition}

\begin{theoreme}{Sous-espaces propres}
  On pose \(E_{\lambda_{i}} = \ker(f-\lambda_{i}\mathrm{Id})\) le sous-espace propre associé à \(\lambda_{i}\).
  \[E = \bigoplus_{i=0}^{k}E_{\lambda_{i}}\]
  Les dimensions des sous-espaces propres sont inférieures aux ordre de multiplicité des \(\lambda_{i}\) en tant que racines de \(\chi_{f}\).
\end{theoreme}

\begin{theoreme}{Première CNS de diagonalisabilité}
  \(f\) est diagonalisable \ssi \(\chi_{f}\) est scindé et si pour chaque \(\lambda\) dans \(\Sp(f)\), \(E_{\lambda}\) est de dimension l'ordre de multiplicité de \(\lambda\) dans \(\chi_{f}\).
\end{theoreme}

\begin{theoreme}{Trigonalisabilité}
  \(f\) est trigonalisable \ssi \(\chi_{f}\) est scindé.
  \tcblower
  Sur \(\C\), toute matrice est trigonalisable.
\end{theoreme}

\section{Polynômes d'endomorphismes}
\begin{theoreme}{Commutativité}
  Soient \(P\) et \(Q\) deux polynômes, comme \(PQ=QP\), on a \(P(f)\circ Q(f)=Q(f)\circ P(f)\).
\end{theoreme}

\begin{definition}
  L'ensemble \(I_{f}=\lbrace P\in K\cof{X} \slash P(f)=0\rbrace\) est un idéal de \(K\cof{X}\) non réduit à \(\lbrace 0 \rbrace\). Son générateur normalisé est le \emph{polynôme minimal} de \(f\), noté \(\pi_{f}\).
\end{definition}

\begin{theoreme}{Lemme de décompasition des noyaux}
  Si \(P\wedge Q = 1\), alors:
  \[\ker\, (QP)(f) = \ker P(f)\oplus \ker Q(f)\]
\end{theoreme}

\begin{theoreme}{Théorème de Cayley-Hamilton}
  Le polynôme minimal \(\pi_{f}\) divise \(\chi_{f}\):
  \begin{itemize}[label=\(\bullet\)]
    \item \(\pi_{f}\,|\,\chi_{f}\)
    \item \(\chi_{f}(f) = 0\)
    \item \(\chi_{f} \in I_{f}\)
  \end{itemize}
\end{theoreme}

\begin{theoreme}{Seconde CNS de diagonalisabilité}
  \(f\) est diagonalisable \ssi \(\pi_{f}\) est scindé à racines simples.
\end{theoreme}


\end{document}
