\documentclass[french, a4paper, 10pt, twocolumn]{article}

\usepackage[utf8]{inputenc} % ~ Encodage
\usepackage[T1]{fontenc}    % ~ Encodage
\usepackage[left=1cm, right=1cm]{geometry} % ~ Mise en page et marges
\usepackage{amssymb} % ~ Pour écrire les maths
\usepackage{xspace}  % ~ Commandes à texte
\usepackage{varioref} % ~ Références croisées
\usepackage{enumitem} % ~ Listes
\usepackage{xcolor}   % ~ Couleurs fs
\usepackage{float}
\usepackage{graphicx}
\usepackage[f]{esvect}
\usepackage[many]{tcolorbox}
\usepackage{euler}
\usepackage[nointegrals]{wasysym}
\usepackage[french]{babel}


%______FONCTIONS______%
\newcommand{\ssi}{si et seulement si\xspace}		% ~ ssi
\newcommand{\inv}[1]{\dfrac{1}{#1}}             % ~ inverse
\newcommand{\bigslant}[2]{{\raisebox{.2em}{\(#1\)}\left/\raisebox{-.2em}{\(#2\)}\right.}}
\DeclareMathOperator{\Sp}{Sp}
\newcommand{\norme}[1]{\left\| #1\right\|}
\newcommand{\abs}[1]{\left\lvert #1\right\rvert}
\newcommand{\limit}[1]{\underset{#1}{\rightarrow}}  % ~limite
% --ENSEMBLES--%
\newcommand{\N}{\mathbb{N}}   % ~ Entiers naturels
\newcommand{\Z}{\mathbb{Z}}   % ~ Entiers relatifs
\newcommand{\D}{\mathbb{D}}   % ~ Decimaux
\newcommand{\Q}{\mathbb{Q}}   % ~ Rationnels
\newcommand{\R}{\mathbb{R}}   % ~ Réels
\newcommand{\C}{\mathbb{C}}   % ~ Complexes
\newcommand{\czero}{\mathcal{C}^{0}}
\newcommand{\cun}{\mathcal{C}^{1}}
% --PARENTHESES--%
\newcommand{\po}{\left(}         % ~ (
\newcommand{\pf}{\right)}        % ~ )
\newcommand{\pof}[1]{\po #1 \pf} % ~ ( )
\newcommand{\co}{\left[}         % ~ [
\newcommand{\cf}{\right]}        % ~ ]
\newcommand{\cof}[1]{\co #1 \cf} % ~ [ ]
\newcommand{\chof}[1]{\left\langle #1 \right\rangle } % ~ < >
\newcommand{\interoo}[2]{\left]#1\,;#2\right[}   % ~ ]a,b[
\newcommand{\interof}[2]{\left]#1\,;#2\right]}   % ~ ]a,b]
\newcommand{\interfo}[2]{\left[#1\,;#2\right[}   % ~ [a,b[
\newcommand{\interff}[2]{\left[#1\,;#2\right]}   % ~ [a,b]
\renewcommand{\phi}{\varphi}

\newtcolorbox{theoreme}[2][]
{
  enhanced,
  attach boxed title to top left={yshift=-3.4mm, xshift = -2.3mm},
  adjusted title=#2,
  colback=white, colframe=black,
  colbacktitle=white, coltitle=black, fonttitle=\bfseries,
  breakable, sharp corners,
  boxed title style={colback=white, sharp corners, colframe=white},
  boxrule = 0.5mm, drop fuzzy shadow
}

\newtcolorbox{definition}[1][]
{
  enhanced,
  attach boxed title to top left={yshift=-3.4mm, xshift = -2.3mm},
  adjusted title=Définition,
  colback=white, colframe=black,
  colbacktitle=white, coltitle=black, fonttitle=\bfseries,
  breakable, sharp corners,
  boxed title style={colback=white, sharp corners, colframe=white},
  boxrule = 0.5mm, drop fuzzy shadow
}


\newcommand{\ooint}{\ocircle\hspace{-3.65mm}\int\hspace{-2mm}\int}

\title{Analyse}
\author{Martin \textsc{Andrieux}, Nathan \textsc{Maillet}}
\date{}

\begin{document}
\maketitle

\section{Continuité et Dérivabilité}
\begin{definition}
  \(f\) est dite \emph{convexe} \ssi pour \(a\) et \(b\) dans \(I\) et pour \(t\) dans \(\interff{0}{1}\):
  \[f((1-t)a+tb)\leqslant (1-t)f(a)+tf(b)\]
\end{definition}

\begin{theoreme}{Inégalité de Taylor-Lagrange}
  \[\left\lVert f(b)-\sum_{k=0}^{n}\dfrac{\pof{b-a}^{k}}{k!}f^{(k)}(a)\right\rVert \leqslant\dfrac{\pof{b-a}^{n+1}}{(n+1)!}\left\lVert f^{(n+1)}\right\rVert_{\infty}\]
\end{theoreme}

\begin{theoreme}{Taylor avec reste intégral}
\[f(b)=\sum_{k=0}^{n}\dfrac{\pof{b-a}^{k}}{k!}f^{(k)}(a)+\int_{a}^{b}\dfrac{\pof{b-t}^{n}}{n!}f^{(n+1)}(t)dt\]
\end{theoreme}

\begin{theoreme}{Taylor-Young}
\[f(x)=\sum_{k=0}^{n}\dfrac{\pof{x-a}^{k}}{k!}f^{(k)}(a)+o\pof{(x-a)^{n}}\]
\end{theoreme}

\begin{theoreme}{Prolongement \(\mathcal C^{1}\)}
  \(f\) continue de \(I\) dans \(\R\), \(a\) dans \(I\), \(f\) dérivable sur \(I\backslash \lbrace a\rbrace\).

  Si \(f'\) a une limite \(l\) en \(a\), alors \(f\) est dérivable en \(a\) et \(f'(a)=l\).
\end{theoreme}

\begin{theoreme}{Théorème fondamental}
  \[F : x\mapsto \int_{a}^{x}f(t)dt\]
  \(F\) est continue et dérivable avec \(F'=f\).

  Si \(f\) est continue, alors \(f\) possède des primitives.
\end{theoreme}

\section{Intégrale à paramètre}
\begin{theoreme}{Théorème de continuité}
  \begin{itemize}[label=\(\bullet\)]
    \item \(f:U\times I\rightarrow\C\)
    \item \(\forall x\in U\), \(t\mapsto f(x,t)\) est \(\czero_{\text{pm}}\)
    \item \(\forall t \in I\), \(x\mapsto f(x,t)\) est \(\czero\)
    \item Il existe \(\phi : I\rightarrow \R\), \(\czero_{\text{pm}}\) et sommable telle que
      \[\forall x\in U,\,\forall t\in I\quad \vert f(x,t)\vert\leqslant\phi(t)\]
  \end{itemize}
  \tcblower
  Alors \(F\) est définie et \(\czero\) sur \(U\), avec
  \[F(x)=\int_{I}f(x,t)dt\]
\end{theoreme}

\begin{theoreme}{Théorème de dérivabilité}
  \begin{itemize}[label=\(\bullet\)]
    \item \(f:U\times I\rightarrow\C\)
    \item \(\forall x\in U\), \(t\mapsto f(x,t)\) est \(\czero_{\text{pm}}\) et sommable
    \item \(\forall t \in I\), \(x\mapsto f(x,t)\) est \(\cun\)
    \item \(\forall x\in U\), \(t\mapsto \dfrac{\partial f}{\partial x}(x,t)\) est \(\czero_{\text{pm}}\)
    \item Il existe \(\phi : I\rightarrow \R\), \(\czero_{\text{pm}}\) et sommable telle que
      \[\forall x\in U,\,\forall t\in I\quad \left\vert\dfrac{\partial f}{\partial x}(x,t)\right\vert\leqslant\phi(t)\]
  \end{itemize}
  \tcblower
  Alors \(F\) est définie et \(\cun\) sur \(U\), avec
  \[F'(x)=\int_{I}\dfrac{\partial f}{\partial x}(x,t)dt\]
\end{theoreme}

\begin{theoreme}{fonction \(\Gamma\)}
  On défnit la fonction \(\Gamma\) comme suit:
  \[\Gamma:x\mapsto\int_{0}^{+\infty}t^{x-1}e^{-t}dt\]
  On a alors \(\Gamma(x+1)=x\Gamma(x)\) et \(\Gamma(n+1)=n!\) pour \(n\) dans \(\N\). \(\Gamma(\frac{1}{2})=\sqrt{\pi}\).
\end{theoreme}

\section{Suites et séries de fonctions}
\begin{definition}
  On dit qu'une suite de fonctions \(f_{n}\) converge \emph{simplement} vers \(f\) si:
  \[\forall x\quad f_{n(x)}\xrightarrow[n\rightarrow +\infty]{} f(x)\]
  La limite simple conserve les propriétés portant sur un nombre fini de points, comme la positivité, la croissance et la convexité.
\end{definition}

\begin{definition}
  On dit qu'une suite de fonctions \(f_{n}\) converge \emph{uniformément} vers \(f\) si
  \begin{align*}
    \forall \varepsilon >0,\,\exists\, n_{\varepsilon}\in\N \,\slash\, \forall n\geqslant n_{\varepsilon} \\
    \forall\, x,\quad \left\lVert f_{n}(x)-f(x)\right\rVert \leqslant \varepsilon
  \end{align*}
  La convergence uniforme de \((f_{n})\) est la convergence pour \(\lVert\,\rVert_{\infty}\)

  La convergence uniforme d'une série est équivalente à la convergence uniforme du reste vers \(0\).

  La convergence uniforme entraîne la convergence simple.
\end{definition}

\begin{definition}
  On dit qu'une série converge \emph{normalement} si la série des normes infinies converge.
  La convergence normale entraîne la convergence uniforme.
\end{definition}

\begin{theoreme}{Théorèmes pour la convergence uniforme}
  La convergence uniforme conserve:
  \begin{itemize}
    \item la continuité
    \item la limite
    \item la sommabilité
    \item la dérivabilité et la continuité de la dérivée
  \end{itemize}
\end{theoreme}

\end{document}
