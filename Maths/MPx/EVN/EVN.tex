\documentclass[french, a4paper, 11pt, twocolumn]{article}

\usepackage[utf8]{inputenc} % ~ Encodage
\usepackage[T1]{fontenc}    % ~ Encodage
\usepackage[left=1cm, right=1cm]{geometry} % ~ Mise en page et marges
\usepackage{amssymb} % ~ Pour écrire les maths
\usepackage{xspace}  % ~ Commandes à texte
\usepackage{varioref} % ~ Références croisées
\usepackage{enumitem} % ~ Listes
\usepackage{xcolor}   % ~ Couleurs fs
\usepackage{float}
\usepackage{graphicx}
\usepackage[f]{esvect}
\usepackage[many]{tcolorbox}
\usepackage{euler}
\usepackage[nointegrals]{wasysym}
\usepackage{babel}


%______FONCTIONS______%
\newcommand{\ssi}{si et seulement si\xspace}		% ~ ssi
\newcommand{\inv}[1]{\dfrac{1}{#1}}             % ~ inverse
\newcommand{\bigslant}[2]{{\raisebox{.2em}{\(#1\)}\left/\raisebox{-.2em}{\(#2\)}\right.}}
\DeclareMathOperator{\Sp}{Sp}
\newcommand{\norme}[1]{\left\| #1\right\|}
\newcommand{\abs}[1]{\left\lvert #1\right\rvert}
\newcommand{\limit}[1]{\underset{#1}{\rightarrow}}  % ~limite
% --ENSEMBLES--%
\newcommand{\N}{\mathbb{N}}   % ~ Entiers naturels
\newcommand{\Z}{\mathbb{Z}}   % ~ Entiers relatifs
\newcommand{\D}{\mathbb{D}}   % ~ Decimaux
\newcommand{\Q}{\mathbb{Q}}   % ~ Rationnels
\newcommand{\R}{\mathbb{R}}   % ~ Réels
\newcommand{\C}{\mathbb{C}}   % ~ Complexes
% --PARENTHESES--%
\newcommand{\po}{\left(}         % ~ (
\newcommand{\pf}{\right)}        % ~ )
\newcommand{\pof}[1]{\po #1 \pf} % ~ ( )
\newcommand{\co}{\left[}         % ~ [
\newcommand{\cf}{\right]}        % ~ ]
\newcommand{\cof}[1]{\co #1 \cf} % ~ [ ]
\newcommand{\chof}[1]{\left\langle #1 \right\rangle } % ~ < >
\newcommand{\interoo}[2]{\left]#1\,;#2\right[}   % ~ ]a,b[
\newcommand{\interof}[2]{\left]#1\,;#2\right]}   % ~ ]a,b]
\newcommand{\interfo}[2]{\left[#1\,;#2\right[}   % ~ [a,b[
\newcommand{\interff}[2]{\left[#1\,;#2\right]}   % ~ [a,b]

\newtcolorbox{theoreme}[2][]
{
  enhanced,
  attach boxed title to top left={yshift=-3.4mm, xshift = -2.3mm},
  adjusted title=#2,
  colback=white, colframe=black,
  colbacktitle=white, coltitle=black, fonttitle=\bfseries,
  breakable, sharp corners,
  boxed title style={colback=white, sharp corners, colframe=white},
  boxrule = 0.5mm, drop fuzzy shadow
}

\newtcolorbox{definition}[1][]
{
  enhanced,
  attach boxed title to top left={yshift=-3.4mm, xshift = -2.3mm},
  adjusted title=Définition,
  colback=white, colframe=black,
  colbacktitle=white, coltitle=black, fonttitle=\bfseries,
  breakable, sharp corners,
  boxed title style={colback=white, sharp corners, colframe=white},
  boxrule = 0.5mm, drop fuzzy shadow
}


\newcommand{\ooint}{\ocircle\hspace{-3.65mm}\int\hspace{-2mm}\int}

\title{Espaces vectoriels normés}
\author{Martin \textsc{Andrieux}, Nathan \textsc{Maillet}}
\date{Dans toute la suite, \(E\) et \(F\) désignent deux \(\mathbb{K}\) espaces vectoriels normés.}

\begin{document}
\maketitle

\begin{definition}
  Une \emph{norme} \(\norme{\cdot}\) est une application de \(E\) dans \(\R\) telle que:
  \begin{itemize}[label=\(\bullet\)]
    \item \(\forall x\in E,\,\norme{x}\geqslant 0\)
    \item \(\forall x\in E,\,\norme{x}=0\implies x=0\)
    \item \(\forall\lambda\in K,\,\forall x\in E,\, \norme{\lambda x}=\abs{\lambda} \cdot \norme{x}\)
    \item \(\forall x,y\in E,\, \norme{x+y}\leqslant\norme{x}+\norme{y}\)
  \end{itemize}
\end{definition}

\begin{theoreme}{Cauchy Schwarz}
  \[\left\lvert \langle x,y\rangle\right\rvert\leqslant\norme{x}\cdot\norme{y}\]
  \tcblower
  \[\abs{\sum_{i=1}^n x_{i}\cdot y_{i}}\leqslant \sqrt{\sum_{i=1}^n x_{i}^{2}}\sqrt{\sum_{i=1}^n y_{i}^{2}}\]
\end{theoreme}

\begin{theoreme}{Caractérisations séquentielles}
    Avec \(A \subset E\):
    \begin{itemize}[label=\(\bullet\)]
        \item \(A\) est ouvert \ssi pour toute suite \(x_{n}\) de \(E\) tendant vers \(a\) dans \(A\), \(x_{n}\) est dans \(A\) pour \(n\) assez grand.
        \item \(A\) est fermé \ssi \(A\) est stable par limite.
        \item Pour \(x\) dans \(E\):
        \begin{align*} 
            x \in \bar{A} & \iff \forall r \geq 0, B(x,r) \cap A \neq \emptyset \\
                          & \iff \exists {(x_n)}_{n\in N} \in A^{\N} / x_n \limit{\infty} x
        \end{align*}
    \end{itemize}
\end{theoreme}

\begin{theoreme}{Continuité et topolgie}
    \begin{itemize}[label=\(\bullet\)]
        \item \(f\) est continue \ssi pour \(A\) fermé de \(E\),\(\, f^{-1}(A)\) est un fermé de \(E\) (de même avec les ouverts).
    \end{itemize}
\end{theoreme}

\begin{theoreme}{Applications linéaires et topologie}
  Avec \(f\) linéaire de \(E\) dans \(F\), les propriétés suivantes sont équivalentes.
  \begin{itemize}[label=\(\bullet\)]
    \item \(f\in \mathcal{L}_{c}\)
        \item \(f\) continue en un point
        \item \(f\) continue en \(0\)
        \item \(f\) bornée sur \(B(0,1)\)
        \item \(\exists K \in \R^+/\forall x\in E, \norme{f(x)}\leq K\norme{x}\)
        \item \(f\) est lipchitzienne
    \end{itemize}
\end{theoreme}

\begin{theoreme}{Équivalence des normes}
  Deux normes \(\norme{\;}_{1}\) et \(\norme{\;}_{2}\) sont dites \emph{équivalentes} si (les propriétés suivantes sont équivalentes):
  \begin{itemize}[label=\(\bullet\)]
    \item \(x_{n} \xrightarrow[n\rightarrow \infty]{\norme{\;}_{1}} x \iff x_{n} \xrightarrow[n\rightarrow \infty]{\norme{\;}_{2}}x\)
    \item On a \(\alpha\) et \(\beta\) tels que \(\alpha\norme{x}_{1}\leqslant\norme{x}_{2}\leqslant\beta\norme{x}_{1}\)
  \end{itemize}
\end{theoreme}

\begin{theoreme}{En dimension finie}
    \begin{itemize}[label=\(\bullet\)]
        \item Le théorème d'équivalence des normes permet de choisir une norme adaptée au problème s'il y a besoin de norme
        \item Les compacts sont les fermés bornés
      \item \(f \in L(E) \implies f \in \mathcal{L}_c\)
        \item Les applications bi-linéraires sont continues
        \item Tout sous espace vectoriel est fermé
    \end{itemize}
\end{theoreme}

\begin{definition}
    Une partie \(K\) de \(E\) est dite \emph{compacte} si toute suite d'élement de \(K\) possède une valeur d’adhérence dans \(K\).
\end{definition}

\begin{theoreme}{Compacité et topologie}
    \begin{itemize}[label=\(\bullet\)]
        \item Tout compact est fermé borné (la réciproque est fausse en dimension infinie)
        \item Une partie \(A\) d'un compacte est compacte \ssi elle est fermée.
        \item Le produit de deux compactes est compacte
    \end{itemize}
\end{theoreme}

\begin{theoreme}{Compacité et continuité}
    \begin{itemize}[label=\(\bullet\)]
        \item Si \(f : K \rightarrow F\) est continue et \(K\) est un compact, alors \(f\) est uniformément continue et \(f(K)\) est compact.
        \item Soient \(K\) un compact non vide et \(f : K \rightarrow \R\), alors \(f\) est bornée et atteint ses bornes.
    \end{itemize}
    
\end{theoreme}

\begin{theoreme}{Connexité par arcs}
    \(A\) est dite connexe par arcs \ssi \(\forall a,b \in A, \exists \gamma : [0,1] \rightarrow A\) continue telle que \(\gamma(0)=a\) et \(\gamma(1)=b\)
\end{theoreme}

\begin{theoreme}{Absolue convergence}
    \[\sum_{n \geq 0} \norme{u_n} \text{converge} \implies \sum_{n \geq 0} u_n \text{ converge}\]
\end{theoreme}

\begin{definition}
    La norme subordonnée est définie par:
    \[|||f|||=\underset{x \in E}{\sup}\pof{\dfrac{\norme{f(x)}}{\norme{x}}}\]

    On a alors: \(|||g \circ f||| \leq |||g||| \times |||f|||\)
\end{definition}

\end{document}
