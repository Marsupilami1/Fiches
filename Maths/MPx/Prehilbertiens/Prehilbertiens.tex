\documentclass[french, a4paper, 11pt, twocolumn]{article}

\usepackage[utf8]{inputenc} % ~ Encodage
\usepackage[T1]{fontenc}    % ~ Encodage
\usepackage[left=1cm, right=1cm]{geometry} % ~ Mise en page et marges
\usepackage{amssymb} % ~ Pour écrire les maths
\usepackage{xspace}  % ~ Commandes à texte
\usepackage{varioref} % ~ Références croisées
\usepackage{enumitem} % ~ Listes
\usepackage{xcolor}   % ~ Couleurs fs
\usepackage{float}
\usepackage{graphicx}
\usepackage[f]{esvect}
\usepackage[many]{tcolorbox}
\usepackage{euler}
\usepackage[nointegrals]{wasysym}
\usepackage[french]{babel}


%______FONCTIONS______%
\newcommand{\ssi}{si et seulement si\xspace}		% ~ ssi
\newcommand{\inv}[1]{\dfrac{1}{#1}}             % ~ inverse
\newcommand{\bigslant}[2]{{\raisebox{.2em}{\(#1\)}\left/\raisebox{-.2em}{\(#2\)}\right.}}
\DeclareMathOperator{\Sp}{Sp}
\newcommand{\norme}[1]{\left\| #1\right\|}
\newcommand{\scal}[2]{\langle #1\, ,#2 \rangle}
\newcommand{\abs}[1]{\left\lvert #1\right\rvert}
\newcommand{\limit}[1]{\underset{#1}{\rightarrow}}  % ~limite
% --ENSEMBLES--%
\newcommand{\N}{\mathbb{N}}   % ~ Entiers naturels
\newcommand{\Z}{\mathbb{Z}}   % ~ Entiers relatifs
\newcommand{\D}{\mathbb{D}}   % ~ Decimaux
\newcommand{\Q}{\mathbb{Q}}   % ~ Rationnels
\newcommand{\R}{\mathbb{R}}   % ~ Réels
\newcommand{\C}{\mathbb{C}}   % ~ Complexes
% --PARENTHESES--%
\newcommand{\po}{\left(}         % ~ (
\newcommand{\pf}{\right)}        % ~ )
\newcommand{\pof}[1]{\po #1 \pf} % ~ ( )
\newcommand{\co}{\left[}         % ~ [
\newcommand{\cf}{\right]}        % ~ ]
\newcommand{\cof}[1]{\co #1 \cf} % ~ [ ]
\newcommand{\chof}[1]{\left\langle #1 \right\rangle } % ~ < >
\newcommand{\interoo}[2]{\left]#1\,;#2\right[}   % ~ ]a,b[
\newcommand{\interof}[2]{\left]#1\,;#2\right]}   % ~ ]a,b]
\newcommand{\interfo}[2]{\left[#1\,;#2\right[}   % ~ [a,b[
\newcommand{\interff}[2]{\left[#1\,;#2\right]}   % ~ [a,b]

\newtcolorbox{theorem}[2][]
{
  enhanced,
  attach boxed title to top left={yshift=-3.4mm, xshift = -2.3mm},
  adjusted title=#2,
  colback=white, colframe=black,
  colbacktitle=white, coltitle=black, fonttitle=\bfseries,
  breakable, sharp corners,
  boxed title style={colback=white, sharp corners, colframe=white},
  boxrule = 0.5mm, drop fuzzy shadow
}

\newtcolorbox{definition}[1][]
{
  enhanced,
  attach boxed title to top left={yshift=-3.4mm, xshift = -2.3mm},
  adjusted title=Définition,
  colback=white, colframe=black,
  colbacktitle=white, coltitle=black, fonttitle=\bfseries,
  breakable, sharp corners,
  boxed title style={colback=white, sharp corners, colframe=white},
  boxrule = 0.5mm, drop fuzzy shadow
}


\newcommand{\ooint}{\ocircle\hspace{-3.65mm}\int\hspace{-2mm}\int}

\title{Espaces préhilbertiens}
\author{Nathan \textsc{Maillet}}
\date{Dans toute la suite, \(\left(E,\scal{.}{.}\right)\) désigne un espace préhilbertien et \(x,y\) deux éléments de \(E\)}

\begin{document}
\maketitle

\begin{theorem}{Égalités de polarisations}
  \begin{align*}
    \scal{x}{y} &= \frac{\norme{x+y}^2-\norme{x-y}^2}{4} \\
                &= \frac{\norme{x+y}^2-\norme{x}^2-\norme{y}^2}{2}  
  \end{align*}

  \tcblower
  Égalité du parallélogramme :
    \[\norme{x+y}^2+\norme{x-y}^2=2\left(\norme{x}^2+\norme{y}^2\right)\]
   
  Théorème de Pythagore :
    \[x\perp y \iff \norme{x+y}^2=\norme{x}^2+\norme{y}^2\]
\end{theorem}

\begin{theorem}{Procédé de Gram-Schmidt}
  Soit \((e_i)_{1\leq i \leq n}\) une base de \(E\) euclidien.
  Il existe une unique base orthonormale \((\varepsilon_i)_{1\leq i \leq n}\) telle que \(P_{(e_i)}^{(\varepsilon_i)}\)
  soit triangulaire supérieure à diagonale strictement positive.
  La formule générale pour les \(\varepsilon_i\) est :
    \[\varepsilon_{i+1}=\frac{\varepsilon_{i+1}'}{\norme{\varepsilon_{i+1}'}} \text{ avec }  
  \varepsilon_{i+1}'=e_{i+1}-\sum_{j=1}^{i}\scal{e_{i+1}}{\varepsilon_j}\varepsilon_j\]
\end{theorem}

\begin{theorem}{Théorème de projection}
  Soit \(F\) un sous espace vectoriel de \(E\) de dimension fini.
  La projection sur \(F\) parallèlement à \(F^{\perp}\) est la projection orthogonale sur \(F\) notée \(P_F\).
  \(P_F\) est le seul point en lequel \(\mathrm{d}(x,F)\) est atteinte.
  Si \((e_i)_1^n\) est une base \emph{orthonormale} de \(E\), on a :
    \[P_F(x)=\sum_{i=1}^n \scal{x}{e_i}e_i\]
\end{theorem}

\begin{definition}
  Une famille \((e_i)_{i\in I}\) est orthonormale totale si elle est orthonormale et \(\mathrm{Vect}\left(e_n, n \in \N\right)\)
  est dense dans \(E\).

  \tcblower
  Si \((e_n)_{n\in \N}\) est une famille orthonormale totale, on a :
    \[\sum_{k=0}^n \scal{x}{e_k}e_k \limit{n\rightarrow +\infty} x\]
\end{definition}

\begin{definition}
  \(f \in \mathcal{L}(E)\) est dit symétrique (\(f\in S(E)\)) si \(\scal{x}{f(y)}=\scal{f(x)}{y}\)
\end{definition}

\begin{theorem}{Théorème spectrale}
  \(f \in S(E)\) si et seulement si il existe une base orthonormale qui diagonalise \(f\)
  
  \tcblower

  Si \(S \in S_n(\R)\), \(S\) est diagonale dans une base orthonormale
\end{theorem}

\begin{definition}
  Un élément \(f\) de \(\mathcal{L}(E)\) est dit orthogonal si \(\scal{f(x)}{f(u)}=\scal{x}{y}\).
  C'est équivalent à : \(f\) conserve la norme
\end{definition}

\begin{theorem}{Groupe orthogonal}
  Si \(E\) est de dimension \(n, f \in O(E)\) si et seulement si
  il existe une base orthonormale dans laquelle la matrice de \(f\) est de la forme :
  \[\begin{pmatrix}
    R_{\theta_1} & 0 & 0 & 0 & 0 \\
    0 & \ddots & 0 & 0 & 0 \\
    0 & 0 & R_{\theta_k} & 0 & 0 \\
    0 & 0 & 0 & I_p & 0 \\
    0 & 0 & 0 & 0 & -I_q
  \end{pmatrix}\]
  avec \(\forall 1\leq i\leq k,
  R_{\theta_i}=\begin{pmatrix}
    \cos(\theta_i) & -\sin(\theta_i) \\
    \sin(\theta_i) & \cos(\theta_i)
  \end{pmatrix}\)
\end{theorem}
\end{document}
