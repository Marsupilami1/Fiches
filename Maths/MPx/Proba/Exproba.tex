\documentclass[french, a4paper, 11pt]{article}

\usepackage[utf8]{inputenc} % ~ Encodage
\usepackage[T1]{fontenc}    % ~ Encodage
\usepackage[left=1cm, right=1cm, bottom=3cm]{geometry} % ~ Mise en page et marges
\usepackage{amssymb} % ~ Pour écrire les maths
\usepackage{xspace}  % ~ Commandes à texte
\usepackage{varioref} % ~ Références croisées
\usepackage{enumitem} % ~ Listes
\usepackage{xcolor}   % ~ Couleurs fs
\usepackage{tabularx}
\usepackage{multicol}
\usepackage{float}
\usepackage[load-configurations = abbreviations]{siunitx}
\usepackage{graphicx}
\usepackage[f]{esvect}
\usepackage[many]{tcolorbox}
\usepackage{euler}
\usepackage[nointegrals]{wasysym}
\usepackage[french]{babel}


\sisetup{
  locale=FR,
  detect-all,
}

% ______FONCTIONS______%
\newcommand{\ssi}{si et seulement si\xspace}		% ~ ssi
% --ENSEMBLES--%
\newcommand{\N}{\mathbb{N}}   % ~ Entiers naturels
\newcommand{\Z}{\mathbb{Z}}   % ~ Entiers relatifs
\newcommand{\D}{\mathbb{D}}   % ~ Decimaux
\newcommand{\Q}{\mathbb{Q}}   % ~ Rationnels
\newcommand{\R}{\mathbb{R}}   % ~ Réels
\newcommand{\C}{\mathbb{C}}   % ~ Complexes
% --TRIGO--%
\let\cosh\relax
\DeclareMathOperator{\cosh}{ch}       % ~ cosinus hyperbolique
\DeclareMathOperator{\sh}{sh}         % ~ sinus hyperbolique
\let\tanh\relax
\DeclareMathOperator{\tanh}{th}       % ~ tangente hyperbolique
\DeclareMathOperator{\argch}{Argch}   % ~ Argument cosinus hyperbolique
\DeclareMathOperator{\argsh}{Argsh}   % ~ Argument sinus hyperbolique
\DeclareMathOperator{\argth}{Argth}   % ~ Argument tangente hyperbolique
\DeclareMathOperator{\cotan}{cotan}   % ~ cotangente
% --PARENTHESES--%
\newcommand{\po}{\left(}         % ~ (
\newcommand{\pf}{\right)}        % ~ )
\newcommand{\pof}[1]{\po #1 \pf} % ~ ( )
\newcommand{\co}{\left[}         % ~ [
\newcommand{\cf}{\right]}        % ~ ]
\newcommand{\cof}[1]{\co #1 \cf} % ~ [ ]
\newcommand{\chof}[1]{\left\langle #1 \right\rangle } % ~ < >
\newcommand{\interoo}[2]{\left]#1\,;#2\right[}   % ~ ]a,b[
\newcommand{\interof}[2]{\left]#1\,;#2\right]}   % ~ ]a,b]
\newcommand{\interfo}[2]{\left[#1\,;#2\right[}   % ~ [a,b[
\newcommand{\interff}[2]{\left[#1\,;#2\right]}   % ~ [a,b]


\newtcolorbox{cadre}[2][]
{
  enhanced,
  attach boxed title to top left={yshift=-3.4mm, xshift = -2.3mm},
  adjusted title=#2,
  colback=white, colframe=black,
  colbacktitle=white, coltitle=black, fonttitle=\bfseries,
  breakable, sharp corners,
  boxed title style={colback=white, sharp corners, colframe=white},
  boxrule = 0.5mm, drop fuzzy shadow
}
\newcommand{\ooint}{\ocircle\hspace{-3.65mm}\int\hspace{-2mm}\int}

\title{Exercices de probabilités}
\author{Martin \textsc{Andrieux}, Nathan \textsc{Maillet}}
\date{}

\begin{document}
\maketitle

\begin{cadre}{Pile ou face}
  Soit \(p \in \interoo{0}{1}\) et \(r \in \N, r\geq 2\). On effectue une suite infinie de tirages à pile ou face.
  Les tirages sont indépendants et la probabilité de tirer face,à chaque tirage, vaut \(p\). Pour
  \(n\in \N\), on note \(F_n\) l'évènement "face sort au \(n\)èième tirage" et \(P_n\) l'événement "pile sort au \(n\)-ième
  tirage". Pour \(n \in \N^*,\) on note \(E_n\) l'événement "au \(n\)-ième tirage, on obtient \(r\) faces consécutives
  pour la première fois".
  \begin{enumerate}
    \item
      \begin{enumerate}
        \item Déterminer \(E_1 \cdots E_{r-1}\) et \(E_r\)
        \item Soit \(n \in \N\). Montrer que :
                \[E_{n+r+1}=\left(\bigcap_{i=n+2}^{n+r+1}F_i\right)\cap P_{n+1}\cap \left(\bigcap_{i=1}^{n}\overline{E_i}\right)\]
        \item En déduire que chaque \(E_n\) est un événement.
      \end{enumerate}
    \item On pose \(p_0=0\) et, pour \(n\in \N, p_n=P(E_n)\).
        Montrer que \(\sum p_n\) converge.
    \item 
      \begin{enumerate}
        \item Montrer que \[\forall n \in \N, p_{n+r+1}=p^r(1-p)\left(1-\sum_{i=1}^{n}p_i\right)\]
        \item Exprimer, pour \(n \in \N, p_{n+r+1}\) en fonction de \(p_,+r,p_n,p\) et \(q=1-p\)
      \end{enumerate}
    \item Soit \begin{align*}
                    G :\, & \interff{-1}{1} \rightarrow \R \\
                          & x \mapsto \sum_{k=0}^{+\infty}p_k x^k
                \end{align*}
          \begin{enumerate}
            \item Montrer que \(G\) est bien définie et qu'elle est continue
            \item Montrer que 
                  \[\forall x \in \interoo{-1}{1}, \frac{G(x)}{1-x}=\sum_{n=0}^{+\infty}\left(\sum_{k=0}^{n} p_k\right) x^k\]
            \item Exprimer \(G(x)\)
          \end{enumerate}
  \end{enumerate}
\end{cadre}

\begin{cadre}{Variable aléatoire}
  Soit \(\pof{X_{n}}_{n\geqslant 1}\) une suite de variables aléatoires indépendantes définies sur un espace probabilisé \(\pof{\Omega, A, P}\) à valeurs dans \(\lbrace -1;1\rbrace\), telles que, pour \(n\geqslant 1\) :
  \[P\pof{X_{n} = 1} = P\pof{X_{n} = -1} = \dfrac{1}{2}\]
  Pour \(n\geqslant 1\), on pose \(S_{n} = \sum_{k=1}^{n} X_{k}\).
  \begin{enumerate}
      \item
      \begin{enumerate}
        \item Démontrer que, pour tout \(n\) dans \(\N\), \(\dfrac{1}{(2n)!}\leqslant\dfrac{1}{2^{n}n!}\).
        \item Calculer, pour \(n\) dans \(\N\) et \(t\) réel \(E\pof{e^{tX_{n}}}\); en déduire \(E(e^{tX_{n}}) \leqslant e^{t^{2}/2}\).
      \end{enumerate}
    \item Soit \(a\) un nombre réel strictement positif.
      \begin{enumerate}
        \item Montrer que pour tout réel \(t\) positif: \(P(S_{n}\geqslant a) \leqslant e^{-ta}E(e^{tS_{n}})\).
        \item En déduire que \(P(S_{n}\geqslant a)\leqslant e^{-a^{2}/2n}\).
        \item En déduire un majorant de \(P(\lvert S_{n}\rvert \geqslant a)\).
      \end{enumerate}
  \end{enumerate}
\end{cadre}

\begin{cadre}{Inégalités - 1}
  Soit $X$ une variable aléatoire suivant la loi de Poisson de paramètre $\lambda >0$. On note $G_{X}$ sa série génératrice.
  \begin{enumerate}
    \item Montrer que $P\pof{\lvert X-\lambda\rvert \geqslant \lambda}\leqslant \dfrac{1}{\lambda}$; en déduire l'inégalité $P(X\geqslant 2\lambda) \leqslant \dfrac{1}{\lambda}$.
    \item Montrer que, pour tout $t$ dans $\interoo{1}{+\infty}$ et pour tout $a$ réel positif non nul, $P(X\geqslant a)\geqslant \dfrac{G_{x}(t)}{t^{a}}$.
    \item Déterminer le minimum sur $\interfo{1}{+\infty}$ de la fonction $g:x\mapsto\dfrac{e^{t-1}}{t^{2}}$.
    \item Calculer $G_{x}(t)$; en déduire $P(X\geqslant 2\lambda)\leqslant\pof{\dfrac{e}{4}}^{\lambda}$.
    \item Montrer que cette inégalité est meilleure que la première dès que $\lambda$ prend des valeurs assez grandes.
  \end{enumerate}
\end{cadre}

\begin{cadre}{Inégalités - 2}
  \begin{enumerate}
    \item Pour \(t \in \R, x \in \interff{-1}{1},\) montrer que \(e^{tx} \leq \frac{1}{2}(1-x)e^{-t} + \frac{1}{2}(1+x)e^t\)
    \item Soit \(X\) une variable aléatoire à valeurs dans \(\interff{-1}{1}\) et d'espérance nulle. Montrer que \(e^{X}\) est d'espérance finie
          et que \(E(e^{tX}) \leq \mathrm{ch}(t) \leq e^{t^2/2}\)
    \item Soient \(X_1 \cdots X_n\) des variables aléatoires centrées indépendantes telles que,
          pour tout \(i, |X_i| \leq a_i.\) On pose \(S_n=\sum_{i=1}^{n}X_i.\)
          \begin{enumerate}
            \item Montrer que \[E(e^{tS_n}) \leq \exp\left(\frac{t^2}{2} \sum_{i=1}^{n}a_i^2\right)\]
            \item Soit \(\epsilon > 0, t > 0.\) Montrer que 
                    \[P(S_n > \epsilon) \leq \exp\left(-t\epsilon + \frac{t^2}{2} \sum_{i=1}^{n} a_i^2\right)\]
            \item En choisissant une bonne valeur de \(t,\) montrer que
                    \[P(S_n > \epsilon) \leq \exp\left(-\dfrac{\epsilon^2}{2\sum_{i=1}^{n} a_i^2}\right)\]
          \end{enumerate}
  \end{enumerate}
\end{cadre}

\begin{cadre}{Stage militaire à l'X}
  \begin{minipage}{0.85\linewidth}
  Le jeune polytechnicien Guillaume s'est perdu dans la fôret. Il cherche à retrouver le reste de l'équipe. À chaque pas de temps, Guillaume et l'équipe changent de camp avec probabilité uniforme en suivant les chemins si contre. Au bout de combien de temps Guillaume peut-il espérer retrouver ses camarades ? Au départ, Guillaume est à l'ouest, l'équipe est au sud.
  \end{minipage}
  \begin{minipage}{0.1\linewidth}
  \begin{tikzpicture}
    \draw (-1,0) -- (0,-1) -- (1,0) -- (0,1) -- (-1,0) -- (1,0);
    \draw (0,1) -- (0,-1);
    \draw[fill=white] (0,0) circle (0.2);
    \draw[fill=white] (0,1) circle (0.2);
    \draw[fill=white] (1,0) circle (0.2);
    \draw[fill=white] (0,-1) circle (0.2);
    \draw[fill=white] (-1,0) circle (0.2);
  \end{tikzpicture}
  \end{minipage}
\end{cadre}

\end{document}
