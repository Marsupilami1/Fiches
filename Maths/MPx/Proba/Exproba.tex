\documentclass[french, a4paper, 11pt]{article}

\usepackage[utf8]{inputenc} % ~ Encodage
\usepackage[T1]{fontenc}    % ~ Encodage
\usepackage[left=1cm, right=1cm, bottom=3cm]{geometry} % ~ Mise en page et marges
\usepackage{amssymb} % ~ Pour écrire les maths
\usepackage{xspace}  % ~ Commandes à texte
\usepackage{varioref} % ~ Références croisées
\usepackage{enumitem} % ~ Listes
\usepackage{xcolor}   % ~ Couleurs fs
\usepackage{tabularx}
\usepackage{multicol}
\usepackage{float}
\usepackage[load-configurations = abbreviations]{siunitx}
\usepackage{graphicx}
\usepackage[f]{esvect}
\usepackage[many]{tcolorbox}
\usepackage{euler}
\usepackage[nointegrals]{wasysym}
\usepackage[french]{babel}


\sisetup{
  locale=FR,
  detect-all,
}

% ______FONCTIONS______%
\newcommand{\ssi}{si et seulement si\xspace}		% ~ ssi
% --ENSEMBLES--%
\newcommand{\N}{\mathbb{N}}   % ~ Entiers naturels
\newcommand{\Z}{\mathbb{Z}}   % ~ Entiers relatifs
\newcommand{\D}{\mathbb{D}}   % ~ Decimaux
\newcommand{\Q}{\mathbb{Q}}   % ~ Rationnels
\newcommand{\R}{\mathbb{R}}   % ~ Réels
\newcommand{\C}{\mathbb{C}}   % ~ Complexes
% --TRIGO--%
\let\cosh\relax
\DeclareMathOperator{\cosh}{ch}       % ~ cosinus hyperbolique
\DeclareMathOperator{\sh}{sh}         % ~ sinus hyperbolique
\let\tanh\relax
\DeclareMathOperator{\tanh}{th}       % ~ tangente hyperbolique
\DeclareMathOperator{\argch}{Argch}   % ~ Argument cosinus hyperbolique
\DeclareMathOperator{\argsh}{Argsh}   % ~ Argument sinus hyperbolique
\DeclareMathOperator{\argth}{Argth}   % ~ Argument tangente hyperbolique
\DeclareMathOperator{\cotan}{cotan}   % ~ cotangente
% --PARENTHESES--%
\newcommand{\po}{\left(}         % ~ (
\newcommand{\pf}{\right)}        % ~ )
\newcommand{\pof}[1]{\po #1 \pf} % ~ ( )
\newcommand{\co}{\left[}         % ~ [
\newcommand{\cf}{\right]}        % ~ ]
\newcommand{\cof}[1]{\co #1 \cf} % ~ [ ]
\newcommand{\chof}[1]{\left\langle #1 \right\rangle } % ~ < >
\newcommand{\interoo}[2]{\left]#1\,;#2\right[}   % ~ ]a,b[
\newcommand{\interof}[2]{\left]#1\,;#2\right]}   % ~ ]a,b]
\newcommand{\interfo}[2]{\left[#1\,;#2\right[}   % ~ [a,b[
\newcommand{\interff}[2]{\left[#1\,;#2\right]}   % ~ [a,b]


\newtcolorbox{cadre}[2][]
{
  enhanced,
  attach boxed title to top left={yshift=-3.4mm, xshift = -2.3mm},
  adjusted title=#2,
  colback=white, colframe=black,
  colbacktitle=white, coltitle=black, fonttitle=\bfseries,
  breakable, sharp corners,
  boxed title style={colback=white, sharp corners, colframe=white},
  boxrule = 0.5mm, drop fuzzy shadow
}
\newcommand{\ooint}{\ocircle\hspace{-3.65mm}\int\hspace{-2mm}\int}

\title{Exercices de probabilités}
\author{Martin \textsc{Andrieux}, Nathan \textsc{Maillet}}
\date{}

\begin{document}
\maketitle

\begin{cadre}{Variable aléatoire}
  Soit \(\pof{X_{n}}_{n\geqslant 1}\) une suite de variables aléatoires indépendantes définies sur un espace probabilisé \(\pof{\Omega, A, P}\) à valeurs dans \(\lbrace -1;1\rbrace\), telles que, pour \(n\geqslant 1\) :
  \[P\pof{X_{n} = 1} = P\pof{X_{n} = -1} = \dfrac{1}{2}\]
  Pour \(n\geqslant 1\), on pose \(S_{n} = \sum_{k=1}^{n} X_{k}\).
  \begin{enumerate}
      \item
      \begin{enumerate}
        \item Démontrer que, pour tout \(n\) dans \(\N\), \(\dfrac{1}{(2n)!}\leqslant\dfrac{1}{2^{n}n!}\).
        \item Calculer, pour \(n\) dans \(\N\) et \(t\) réel \(E\pof{e^{tX_{n}}}\); en déduire \(E(e^{tX_{n}}) \leqslant e^{t^{2}/2}\).
      \end{enumerate}
    \item Soit \(a\) un nombre réel strictement positif.
      \begin{enumerate}
        \item Montrer que pour tout réel \(t\) positif: \(P(S_{n}\geqslant a) \leqslant e^{-ta}E(e^{tS_{n}})\).
        \item En déduire que \(P(S_{n}\geqslant a)\leqslant e^{-a^{2}/2n}\).
        \item En déduire un majorant de \(P(\lvert S_{n}\rvert \geqslant a)\).
      \end{enumerate}
  \end{enumerate}
\end{cadre}

\begin{cadre}{Inégalités}
  Soit $X$ une variable aléatoire suivant la loi de Poisson de paramètre $\lambda >0$. On note $G_{X}$ sa série génératrice.
  \begin{enumerate}
    \item Montrer que $P\pof{\lvert X-\lambda\rvert \geqslant \lambda}\leqslant \dfrac{1}{\lambda}$; en déduire l'inégalité $P(X\geqslant 2\lambda) \leqslant \dfrac{1}{\lambda}$.
    \item Montrer que, pour tout $t$ dans $\interoo{1}{+\infty}$ et pour tout $a$ réel positif non nul, $P(X\geqslant a)\geqslant \dfrac{G_{x}(t)}{t^{a}}$.
    \item Déterminer le minimum sur $\interfo{1}{+\infty}$ de la fonction $g:x\mapsto\dfrac{e^{t-1}}{t^{2}}$.
    \item Calculer $G_{x}(t)$; en déduire $P(X\geqslant 2\lambda)\leqslant\pof{\dfrac{e}{4}}^{\lambda}$.
    \item Montrer que cette inégalité est meilleure que la première dès que $\lambda$ prend des valeurs assez grandes.
  \end{enumerate}
\end{cadre}

\end{document}
