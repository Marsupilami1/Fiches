\documentclass[french, a4paper, 10pt, twocolumn]{article}

\usepackage[utf8]{inputenc} % ~ Encodage
\usepackage[T1]{fontenc}    % ~ Encodage
\usepackage[left=1cm, right=1cm]{geometry} % ~ Mise en page et marges
\usepackage{amssymb} % ~ Pour écrire les maths
\usepackage{xspace}  % ~ Commandes à texte
\usepackage{varioref} % ~ Références croisées
\usepackage{enumitem} % ~ Listes
\usepackage{xcolor}   % ~ Couleurs fs
\usepackage{float}
\usepackage{graphicx}
\usepackage[f]{esvect}
\usepackage[many]{tcolorbox}
\usepackage{euler}
\usepackage[nointegrals]{wasysym}
\usepackage[french]{babel}


%______FONCTIONS______%
\newcommand{\ssi}{si et seulement si\xspace}		% ~ ssi
\newcommand{\inv}[1]{\dfrac{1}{#1}}             % ~ inverse
\newcommand{\bigslant}[2]{{\raisebox{.2em}{\(#1\)}\left/\raisebox{-.2em}{\(#2\)}\right.}}
\DeclareMathOperator{\Sp}{Sp}
\newcommand{\norme}[1]{\left\| #1\right\|}
\newcommand{\abs}[1]{\left\lvert #1\right\rvert}
\newcommand{\limit}[1]{\underset{#1}{\rightarrow}}  % ~limite
% --ENSEMBLES--%
\newcommand{\N}{\mathbb{N}}   % ~ Entiers naturels
\newcommand{\Z}{\mathbb{Z}}   % ~ Entiers relatifs
\newcommand{\D}{\mathbb{D}}   % ~ Decimaux
\newcommand{\Q}{\mathbb{Q}}   % ~ Rationnels
\newcommand{\R}{\mathbb{R}}   % ~ Réels
\newcommand{\C}{\mathbb{C}}   % ~ Complexes
\newcommand{\czero}{\mathcal{C}^{0}}
\newcommand{\cun}{\mathcal{C}^{1}}
% --PARENTHESES--%
\newcommand{\po}{\left(}         % ~ (
\newcommand{\pf}{\right)}        % ~ )
\newcommand{\pof}[1]{\po #1 \pf} % ~ ( )
\newcommand{\co}{\left[}         % ~ [
\newcommand{\cf}{\right]}        % ~ ]
\newcommand{\cof}[1]{\co #1 \cf} % ~ [ ]
\newcommand{\chof}[1]{\left\langle #1 \right\rangle } % ~ < >
\newcommand{\interoo}[2]{\left]#1\,;#2\right[}   % ~ ]a,b[
\newcommand{\interof}[2]{\left]#1\,;#2\right]}   % ~ ]a,b]
\newcommand{\interfo}[2]{\left[#1\,;#2\right[}   % ~ [a,b[
\newcommand{\interff}[2]{\left[#1\,;#2\right]}   % ~ [a,b]
\renewcommand{\phi}{\varphi}

\newtcolorbox{theoreme}[2][]
{
  enhanced,
  attach boxed title to top left={yshift=-3.4mm, xshift = -2.3mm},
  adjusted title=#2,
  colback=white, colframe=black,
  colbacktitle=white, coltitle=black, fonttitle=\bfseries,
  breakable, sharp corners,
  boxed title style={colback=white, sharp corners, colframe=white},
  boxrule = 0.5mm, drop fuzzy shadow
}

\newtcolorbox{definition}[1][]
{
  enhanced,
  attach boxed title to top left={yshift=-3.4mm, xshift = -2.3mm},
  adjusted title=Définition,
  colback=white, colframe=black,
  colbacktitle=white, coltitle=black, fonttitle=\bfseries,
  breakable, sharp corners,
  boxed title style={colback=white, sharp corners, colframe=white},
  boxrule = 0.5mm, drop fuzzy shadow
}


\newcommand{\ooint}{\ocircle\hspace{-3.65mm}\int\hspace{-2mm}\int}

\title{Probabilités}
\author{Nathan \textsc{Maillet}}
\date{}

\begin{document}
\maketitle

\begin{definition}
    Soit \(\Omega\) un ensemble.
    Une tribu sur \(\Omega\) est une partie \(\mathcal{T}\) de \(\mathcal{P(T)}\) telle que :
        \begin{itemize}
            \item \(\varnothing \in \mathcal{T}\)
            \item \(\forall A \in \mathcal{P(T)}, A \in \mathcal{T} \implies A^c \in  \mathcal{T}\)
            \item \(\forall (A_i)_{i\in \N} \in \mathcal{T}^{\N}, \cup_{i\in \N}A_i \in \mathcal{T}\)
        \end{itemize}
    \tcblower
    Quand \(\Omega\) est finit ou dénombrable, on choisira \(\mathcal{P(\Omega)}\) comme tribu.

    Les tribus sont stables par intersection.
\end{definition}

\begin{definition}
    Une loi de probabilité sur $(\Omega,\mathcal{T})$ est une appication \(P : \mathcal{T} \rightarrow \R^+\) tel que :
    \begin{itemize}
        \item \(P(\Omega)=1\)
        \item \(\forall (A_i)_{i\in \N} \in \mathcal{T}^{\N}, A_i\cap A_j =\varnothing,\)
        \[\sum_{i\geq 0}^{}P(A_i) \text{ converge et } \sum_{i=0}^{+\infty}P(A_i)=P(\sqcup_{i\in \N}A_i)\] 
    \end{itemize}
\end{definition}

\begin{theoreme}{Continuité croissante et décroissante}
    Si \((A_n)_{n\in \N} \in \mathcal{T}^{\N}\) est croissante, \[P(A_n)\limit{+\infty}P(\cup_{k\in \N A_k})\]
    
    Si \((A_n)_{n\in \N} \in \mathcal{T}^{\N}\) est décroissante, \[P(A_n)\limit{+\infty}P(\cap_{k\in \N A_k})\]
\end{theoreme}

\begin{definition}
    On dit que \((A_i)_{i \in I} \in \mathcal{T}^I\) sont mutuellements indépendants si \(\forall n \in \N^*, \forall i_1,\dots,i_n\)
    éléments distincts de \(I\), \(P(A_{i_1}\cap\dots\cap A_{i_n})=P(A_{i_1})\times\dots\times P(A_{i_n})\)
\end{definition}

\begin{definition}
    Soit \((A_i)_{1 \leq i \leq n} \in \mathcal{T}^{\N}\), c'est un système complet d'évènements si :
    \begin{itemize}
        \item Les \(A_i\) sont deux à deux disjoints
        \item \[P(\cup_{i=1}^{n}A_i)=1\]
        \item \(\forall i, P(A_i)\neq 0\)
    \end{itemize}
\end{definition}

\begin{theoreme}{Formule des probabilités totales}
    Soit $(A_i)$ un système complet d'évènements, on a :
        \[\forall B, P(B)=\sum_{i}^{}P(A_i)P_{A_i}(B)\]
\end{theoreme}

\begin{definition}
    Soit \(\mathcal{X}\) un ensemble fini ou dénombrable.
    Une variable aléatoire discrète de \(\Omega\) dans \(\mathcal{X}\) est une application
        \[X : \Omega \rightarrow \mathcal{X} / \forall A \subset \mathcal{X}, X^{-1}(A) \in \mathcal{T}\]
    
    \tcblower
    \begin{itemize}
        \item Un couple de variables aléatoires est une variable aléatoire
        \item Les composantes d'une variable aléatoire sont des variables aléatoires
        \item \(X\) est une variable aléatoire, \(f\circ X\) en est une
        \item On retrouve la même définition d'indépendance mutuelle avec les variables aléatoires
    \end{itemize}
\end{definition}

\begin{theoreme}{Lemne des coalitions}
    Soit \(X_1\dots X_n\) des variables aléatoires indépendantes,
    \(\forall k \leq n-1, (X_1,\dots,X_k),(X_{k+1},\dots,X_n)\) sont indépendantes
\end{theoreme}

\begin{theoreme}{Théorème de transfert}
    Soit \(X\) une variable aléatoire de \(\Omega\) dans \(\mathcal{X}\) et \(f : \mathcal{X} \rightarrow \R\).
    \(f(x)\) a une espérance si et seulement si la famille \((f(x_i)P(X=x_i))_{i\in I}\) est sommable.
    Si \(f(x)\) a une espérance, alors : 
        \[E(f(x))=\sum_{i\in I}^{}f(x_i)P(X=x_i)\]
\end{theoreme}

\begin{theoreme}{Espérance}
    L'espérance est linéraire et si \(X,Y\) possèdent un moment d'ordre 1 et sont indépendants,
    \[E(XY)=E(X)E(Y)\]
\end{theoreme}

\begin{theoreme}{Inégalité de Markov}
    Inégalité de Markov : Soit $X$ une variable aléatoire réelle positive qui possède un moment d'ordre 1 :
     \[\forall a>0, {P(X\geqslant a)\leqslant \frac{E(|X|)}{a}}\]
\end{theoreme}
 
\begin{definition}
    Soient \((X,Y)\) deux variables aléatoires discretes qui possèdent un moment d'ordre 2, on a :
    \begin{itemize}[label=\(\bullet\)]
        \item \(X-E(X)\) possède un moment d'ordre 2 appelé variance de X
        \item \(V(X)=E((X-E(X))^2)\)
        \item On appel écart-type de \(X\) : \(\sigma(X)=\sqrt(V(X))\)
        \item On appel covariance de \((X,Y)\) : Cov\((X,Y)=E[(X-E(x))(Y-E(Y))]\)
    \end{itemize}

    \tcblower
    \begin{itemize}
        \item \(V(X)=E(X^2)-(E(X))^2\)
        \item \(\forall \alpha \in \R, V(X+\alpha)=V(X)\) et \(V(\alpha X)=\alpha^2V(x)\)
        \item La variable aléatoire \(\dfrac{X-E(X)}{\sigma(X)}\) est dite centrée réduite
        \item \(V(X+Y)=V(X)+V(Y)+2*\text{Cov}(X,Y)\) et si \(X,Y\) sont indépendants, Cov\((X,Y)=0\)
        \item \(\sqrt{V(X+Y)}\leq\sqrt{V(X)}+\sqrt{V(Y)}\)
        \item Définition : \(\rho(X,Y)=\dfrac{\text{Cov}(X,Y)}{\sigma(X)\sigma(Y)}\)
    \end{itemize}
\end{definition}

\begin{theoreme}{Inégalité de Cauchy-Swarz}
    Soient \((X,Y)\) deux variables aléatoires discretes qui possèdent un moment d'ordre 2, on a :
        \[|\text{Cov}(X,Y)|\leq \sigma(X)\sigma(Y)\] 
\end{theoreme}

\begin{theoreme}{Inégalité de Bienaymé-Tchebychev}
    Soit $X$ une variable aléatoire réelle possédant un moment d'ordre 2, on a :
     \[\forall a>0, P(|X-E(X)|\geqslant a)\leqslant \frac{V(X)}{a^2}\]
\end{theoreme}

Soit \(X : \Omega \rightarrow \N\) une variable aléatoire.

\begin{definition}
    La série génératrice de \(X\) est la série entière \(G_X(z)=\sum_{n\geq 0}P(X=n)z^n\).
    Son noyaux est au moins 1 et il y a convergence normale pour \(z \in \C, |z|\leq 1\).
\end{definition}

\begin{theoreme}{Application des série génératrice}
    \(X\) possède une espérance finie si et seulement si \(G_X\) est dérivable en \(1^-\), avec :
        \[E(X)=G'_X(1)\]

    \tcblower
    \begin{itemize}
        \item En dérivant \(G_X\) on retrouve facilement que \(X\) possède un moment d'ordre 2 si et seulement si \(G_X\) est
        2 fois dérivable en 1 et \(V(X)=G"_X(1)+G'_x(1)-(G'_X(1))^2\)
        \item Soit \(Y : \Omega \rightarrow \N\) une variable aléatoire indépendate de \(X\). On a : \(G_{X+Y}=G_X G_Y\)
    \end{itemize}
\end{theoreme}

\end{document}
