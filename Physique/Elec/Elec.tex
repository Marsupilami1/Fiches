\documentclass[a4paper, 11pt, french, twocolumn]{article}

\usepackage[utf8]{inputenc} 
\usepackage[T1]{fontenc}
\usepackage[left=1cm, right=1cm]{geometry}
\usepackage{enumitem}
\usepackage{amssymb}
\usepackage{mathtools}
\usepackage{amsmath}
\usepackage{amsfonts}
\usepackage{varioref}
\usepackage{graphicx}
\usepackage{gensymb}
\usepackage{xcolor}
\usepackage{systeme}
\usepackage{tikz}
\usepackage[straightvoltages]{circuitikz}
\usepackage{mathrsfs}
\usepackage{esvect}
\usepackage[many]{tcolorbox}

%Ensembles
\newcommand{\R}{\mathbb{R}}
\newcommand{\C}{\mathbb{C}}
\newcommand{\N}{\mathbb{N}}
\newcommand{\Z}{\mathbb{Z}}
\newcommand{\K}{\mathbb{K}}
\newcommand{\Q}{\mathbb{Q}}
\newcommand{\p}{\wedge}
\newcommand{\D}{\mathcal{D}}

%Letttres grecs
\newcommand{\al}{\alpha}
\newcommand{\be}{\beta}
\newcommand{\De}{\Delta}
\newcommand{\de}{\delta}
\newcommand{\la}{\lambda}
\newcommand{\te}{\theta}
\newcommand{\si}{\sigma}
\newcommand{\Om}{\Omega}
\newcommand{\om}{\omega}
\newcommand{\ph}{\varphi}
\newcommand{\ep}{\varepsilon}


\newcommand{\dx}{\mathrm{d}x}
\newcommand{\dt}{\mathrm{d}t}
\newcommand{\tr}{\mathrm{tr}}
\newcommand{\ev}{espace vectoriel}
\newcommand{\eve}{espace vectoriel euclidien}
\newcommand{\sce}{système complet d'évènements}
\newcommand{\som}[2]{\overset{#2}{\underset{#1}{\sum}}}
\newcommand{\produit}[2]{\overset{#2}{\underset{#1}{\prod}}}
\newcommand{\thm}{\textcolor{red}{\underline{Théorème} }}
\newcommand{\ppt}{\textcolor{red}{\underline{Propriété :} }}
\newcommand{\limit}[1]{\underset{#1}{\rightarrow}}
\newcommand{\eq}[1]{\underset{#1}{\sim}}
\newcommand{\inv}[1]{\frac{1}{#1}}
\newcommand{\acc}[1]{\left\{ #1 \right\}}


\newtcolorbox{cadre}[2][]
{
  enhanced,
  attach boxed title to top left={yshift=-3.4mm, xshift = -2.3mm},
  adjusted title=#2,
  colback=white, colframe=black,
  colbacktitle=white, coltitle=black, fonttitle=\bfseries,
  breakable, sharp corners,
  boxed title style={colback=white, sharp corners, colframe=white},
  boxrule = 0.5mm, drop fuzzy shadow
}

\title{Électrocinétique et traitement du signal}
\author{Maillet Nathan\\MP*}
\date{}

\begin{document}
\maketitle



\section{Amplificateur opérationnel}
	\begin{cadre}{Caractéristique d'un amplificateur opérationnel idéal}
		\begin{center}
		  \begin{tikzpicture}[scale=0.6][h]
		    \draw[->] (0,-4) -- ++(0,8) node[left]{$V_s$};
		    \draw[->] (-4,0) -- ++(8,0) node[right]{$\ep$};
		    \draw[ultra thick]  (-3,-1.5) -- ++(3,0) node[right]{$-V_{sat}$};
		    \draw[ultra thick]  (0,1.5) -- ++(3,0);
		    \draw[ultra thick] (0,-1.5) -- ++(0,3)node[left]{$+V_{sat}$};
		  \end{tikzpicture}
		\end{center}
	\end{cadre}
	
	
\section{Analyse de Fourier d'un signal périodique}
	\begin{cadre}{Théorème de Fourier}
		Toute fonction $f$ périodique de pulsation {$\om=2\pi/T$} peut s'écrire :
		\begin{align*}
			f(t)&=a_0+\som{n\geq 1}{}(a_n \cos(n\om t) + b_n \sin(n\om t)) \\
			    &=c_0+\som{n\geq 1}{} c_n \cos(n\om t+\ph_n)
		\end{align*}
	\end{cadre}
	\begin{itemize}
		\item Suivant la parité de f, les $a_n$ ou $b_n$ peuvent être nuls
		\item Les fonctions rectangulaires (resp. triangulaires) ont un spectre en $\inv{n}$ (resp. $\inv{n^2}$)
	\end{itemize}

\section{Électronique numérique}
	\begin{cadre}{Théorème de Nyquist-Shannon}
		Pour échantilloner un signal sans repliement du spectre, la fréquence d'échantillonage $f_e$ doit vérifier : $f_e>2f_{max}$
	\end{cadre}
	
	\begin{cadre}{Pas de quantification }
		Le pas de quantification $q$ est : $q=\frac{U_{max}-U_{min}}{2^n-1}$
	\end{cadre}
	
	\begin{itemize}
		\item Lors de l'échantillonage, le spectre de la sinusoïde présentera un pic en $f$, $f_e-f$, $f_e+f$, $2f_e-f$, $2f_e+f$...
		\item Pour un filtre numérique, on a : $\frac{\mathrm{d}y(t)}{\dt}=\frac{y_n-y_{n-1}}{T_e}$
	\end{itemize}
\end{document}