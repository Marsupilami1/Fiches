\documentclass[french, a4paper, 11pt]{article}

\usepackage[utf8]{inputenc} % ~ Encodage
\usepackage[T1]{fontenc}    % ~ Encodage
\usepackage[left=1cm, right=1cm, bottom=3cm]{geometry} % ~ Mise en page et marges
\usepackage{amssymb} % ~ Pour écrire les maths
\usepackage{xspace}  % ~ Commandes à texte
\usepackage{varioref} % ~ Références croisées
\usepackage{enumitem} % ~ Listes
\usepackage{xcolor}   % ~ Couleurs fs
\usepackage{tabularx}
\usepackage{float}
\usepackage{tikz}
\usepackage[straightvoltages]{circuitikz}
\usepackage[load-configurations = abbreviations]{siunitx}
\usepackage{mhchem}
\usepackage{graphicx}
\usepackage[f]{esvect}
\usepackage[many]{tcolorbox}
\usepackage{euler}
\usepackage[nointegrals]{wasysym}
\usepackage[french]{babel}


\sisetup{
  locale=FR,
  detect-all,
}

% ______FONCTIONS______%
\newcommand{\ssi}{si et seulement si\xspace}		% ~ ssi
% --ENSEMBLES--%
\newcommand{\N}{\mathbb{N}}   % ~ Entiers naturels
\newcommand{\Z}{\mathbb{Z}}   % ~ Entiers relatifs
\newcommand{\D}{\mathbb{D}}   % ~ Decimaux
\newcommand{\Q}{\mathbb{Q}}   % ~ Rationnels
\newcommand{\R}{\mathbb{R}}   % ~ Réels
\newcommand{\C}{\mathbb{C}}   % ~ Complexes
% --TRIGO--%
\let\cosh\relax
\DeclareMathOperator{\cosh}{ch}       % ~ cosinus hyperbolique
\DeclareMathOperator{\sh}{sh}         % ~ sinus hyperbolique
\let\tanh\relax
\DeclareMathOperator{\tanh}{th}       % ~ tangente hyperbolique
\DeclareMathOperator{\argch}{Argch}   % ~ Argument cosinus hyperbolique
\DeclareMathOperator{\argsh}{Argsh}   % ~ Argument sinus hyperbolique
\DeclareMathOperator{\argth}{Argth}   % ~ Argument tangente hyperbolique
\DeclareMathOperator{\cotan}{cotan}   % ~ cotangente
% --PARENTHESES--%
\newcommand{\po}{\left(}         % ~ (
  \newcommand{\pf}{\right)}        % ~ )
\newcommand{\pof}[1]{\po #1 \pf} % ~ ( )
\newcommand{\co}{\left[}         % ~ [
  \newcommand{\cf}{\right]}        % ~ ]
\newcommand{\cof}[1]{\co #1 \cf} % ~ [ ]
\newcommand{\chof}[1]{\left\langle #1 \right\rangle } % ~ < >
\newcommand{\interoo}[2]{\left]#1\,;#2\right[}   % ~ ]a,b[
\newcommand{\interof}[2]{\left]#1\,;#2\right]}   % ~ ]a,b]
\newcommand{\interfo}[2]{\left[#1\,;#2\right[}   % ~ [a,b[
\newcommand{\interff}[2]{\left[#1\,;#2\right]}   % ~ [a,b]
% --VECTEURS--%
\newcommand{\ux}{\vv{u_x}}          % ~ Vecteur ux
\newcommand{\uy}{\vv{u_y}}          % ~ Vecteur uy
\newcommand{\uz}{\vv{u_z}}          % ~ Vecteur uz
\newcommand{\ur}{\vv{u_r}}          % ~ Vecteur ur
\newcommand{\uth}{\vv{u_\theta}}    % ~ Vecteur utheta
\newcommand{\uph}{\vv{u_\varphi}}   % ~ Vecteur uphi
\newcommand{\om}{\vv{OM}}           % ~ Vecteur position
\newcommand{\vvi}{\vv{v}}           % ~ Vecteur vitesse
\newcommand{\vvio}{\vv{v_0}}        % ~ Vecteur v0
\newcommand{\va}{\vv{a}}            % ~ Vecteur	accélération
\newcommand{\vp}{\vv{p}}            % ~ Vecteur quantité de mouvement
\newcommand{\fr}{\vv{F_r}}          % ~ Vecteur force de rappel
\newcommand{\vabla}{\vv{\nabla}}    % ~ nabla
\newcommand{\grad}{\vv{\mathrm{grad}}}  % ~ grad
\DeclareMathOperator{\diverg}{div}        % ~ grad
\newcommand{\rot}{\vv{\mathrm{rot}}}    % ~ grad

\newtcolorbox{cadre}[2][]
{
  enhanced,
  attach boxed title to top left={yshift=-3.4mm, xshift = -2.3mm},
  adjusted title=#2,
  colback=white, colframe=black,
  colbacktitle=white, coltitle=black, fonttitle=\bfseries,
  breakable, sharp corners,
  boxed title style={colback=white, sharp corners, colframe=white},
  boxrule = 0.5mm, drop fuzzy shadow
}
\newcommand{\ooint}{\ocircle\hspace{-3.65mm}\int\hspace{-2mm}\int}

\ctikzset{
  resistors/scale=0.7,
  capacitors/scale=0.7,
  diodes/scale=0.7,
  sources/scale=0.5
}

\title{Exercices d'électronique}
\author{Martin \textsc{Andrieux}}
\date{}

\begin{document}
\maketitle

\begin{cadre}{Réponse d'un filtre linéaire à des signeaux rectangulaires}
  \begin{minipage}{0.6\linewidth}
    L'A.O. est idéal et fonctionne en régime linéaire. On prendra \(C=\SI{10}{\nano\farad}\), \(R_{1}=\SI{1}{\kilo\ohm}\) et \(R_{2}=\SI{10}{\kilo\ohm}\).
    \begin{enumerate}[label=\upshape\alph*)]
      \item Quel est le comprtement de ce filtre en basse et haute fréquence?
      \item Calculer sa fonction de transfert (on pourra poser \(x=R_{2}C\omega\)).
      \item On alimente ce filtre par des signaux rectangulaires de fréquence \(f=\SI{12}{\kilo\hertz}\), et prenant les valeurs \(\pm V_{0}\) avec \(V_{0}=\SI{4.7}{\volt}\). Quel signal obtiendra-t-on en sortie ? Quelle sera l'amplitude crête à crête de ce signal?
      \item On a maintenant en entrée des crénaux de fréquence \(f=\SI{120}{\kilo\hertz}\), avec \(V_{0}=\SI{0.51}{\volt}\). Prévoir qualitativement l'allure du signal de sortie.

        Justifier plus précisément avec l'équation différentielle du circuit. Quelle est l'amplitude crête à crête du signal de sortie \(s\)?
    \end{enumerate}
  \end{minipage}
  \begin{minipage}{0.3\linewidth}
    \shorthandoff{:!}
    \begin{circuitikz}
      \node at (0,0) [en amp] (ao) {};
      \draw (ao.out) to[short, -*] ++(1,0) coordinate (a) to[short] ++ (0.3, 0);
      \draw (a) to[short] ++(0,1.7) coordinate (b) to[C, l_=\(C\), *-*] ++(-4.5,0) |- (ao.-) to[short] ++(-1.12,0) to[generic, l=\(R_{1}\), *-] ++(-2,0);
      \draw (ao.+) -| ++(-0.5,-0.9) coordinate (g) node[eground]{};
      \draw (b) to[short] ++(0,1.2) to[generic, l=\(R_{2}\)] ++(-2.25,0) coordinate(c) to[generic, l=\(R_{2}\)] ++(-2.25,0) -| ++(0, -1.2);
      \draw (c) to[C, l=\(C\), *-] ++(0,1) -- ++(0,-0.2) node[eground, rotate=180]{};
      \draw (g) to[short] ++(-2.42,0) node[name=d]{} to[short] ++(-0.2,0);
      \draw (g) to[short] ++(4,0) node[name=e]{} to[short] ++(0.2,0);
      \draw[->, >=stealth] (d.north) -- node[midway, left]{\(e\)} ++(0,1.3);
      \draw[->, >=stealth] (e.north) -- node[midway, left]{\(s\)} ++(0,1);
    \end{circuitikz}
    \shorthandon{:!}
  \end{minipage}
  \tcblower
  \begin{tabularx}{\linewidth}{Xr}
    b) \(H = -\dfrac{R_{2}}{R_{1}}\dfrac{2+jx}{1+2jx-x^{2}}\) &
    c) Signal triangulaire, avec \(\Delta V=\SI{19.6}{\volt}\). \\
    d) Crénaux légèrement déformés au début, avec \(\Delta V=\SI{20.4}{\volt}\) &
  \end{tabularx}
\end{cadre}

\begin{cadre}{Filtre passe-haut du premier ordre}
  \begin{enumerate}[label=\upshape\alph*)]
    \item Quelle est la fonction de transfert d'un filtre passe-haut du premier ordre (de gain maximal 1), et l'équation différentielle associée?
    \item En déduire une relation de récurrence d'un filtre numérique.
  \end{enumerate}
  \tcblower
  \[s(t)+\tau\dfrac{ds}{dt}=\tau\dfrac{de}{dt}\]
  \[s_{n+1}=(s_{n}+e_{n}-e_{n-1})\dfrac{\tau}{T_{e}+\tau}\]
\end{cadre}

\begin{cadre}{CNA bits en échelle}
  \begin{minipage}{0.32\linewidth}
    \paragraph*{}
    Dans le montage ci-contre, l'A.O. est idéal et fonctionne en régime linéaire. La tension \(E\) est constante. On définit pour chaque interrupteur une variable logique \(k_{i}\) telle que:
    \begin{itemize}
      \item \(k_{i} = 0\) si l'interrupteur est relié à la masse,
      \item \(k_{i} = 1\) dans le cas contraire.
    \end{itemize}
  \end{minipage}
  \begin{minipage}{0.65\linewidth}
    \shorthandoff{:!}
    \begin{circuitikz}[scale=0.9]
      \draw (0,2) to[vsource] (0,0);
      \draw[->, >=stealth] (-0.3,1.5) -- node[midway, left]{\(E\)} (-0.3,0.5);
      \draw (0.6,0) to[short] ++(0,1.8) arc(-90:90:0.2) to[generic, l=\(2R\)] ++(0,2) to[short] ++(1.1,0);
      \foreach \x in {0,1,2} {
        \draw (1.4+\x*1.5,0) to[short, -*] ++(0,0.8);
        \draw (1.4+\x*1.5,1.2) to[short, *-] ++(0,0.8);
        \draw (1.4+\x*1.5,1) node[left]{\(k_{\x}\)} to[short, -*] ++(0.3,0) to[short] ++(0,0.8) arc(-90:90:0.2) to[generic, l=\(2R\), -*] ++(0,2) to[generic, l=\(R\)] ++(1.5,0);
      }
      \draw (5.9,0) to[short, -*] ++(0,0.8);
      \draw (5.9,1.2) to[short, *-] ++(0,0.8);
      \draw (5.9,1) node[left]{\(k_{3}\)} to[short, -*] ++(0.3,0) to[short] ++(0,1.2) to[generic, l=\(2R\), -*] ++(0,2) coordinate (a);
      \draw (0,2) to[short] ++(5.9,0);
      \draw (0,0) to[short] ++(11.5,0) node[eground, rotate=90]{};
      \node at (9,3.83) [en amp] (ao) {};
      \draw (a) |- node[pos=0.8, name=c, scale=0.9]{\(\bullet\)} (ao.-);
      \draw (ao.+) |- (8,0);
      \draw (ao.out) -- node[midway, name=b]{} ++(1,0);
      \draw (b.center) to[short, *-] ++(0,1.7) to[generic, l_=\(R'\)] ++(-3.6,0) -| (c.center);
      \draw[->, >=stealth] (10.5, 0.3) -- node[midway, right]{\(s(t)\)} ++(0,3.3);
    \end{circuitikz}
    \shorthandon{:!}
  \end{minipage}

  \begin{enumerate}[label=\upshape\alph*)]
    \item En remplaçant de proche en proche l'échelle par le générateur de Thévnin équilvalent, montrer que l'ensemble de l'échelle équivaut à un générateur de résitance interne \(T\) et de fem :
   \[e=k_{0}\dfrac{E}{16}+k_{1}\dfrac{e}{8}+k_2\dfrac{E}{4}+k_{3}\dfrac{E}{2}\]
    \item En déduire la tension de sortie \(s(t)\). À quel nombre binaire peut-on l'associer? Quelle est la tension de sortie maximale de ce CNA? Quel est son pas?
  \end{enumerate}
  \tcblower
  \[s(t)=\pof{8k_{3}+4k_{2}+2k_{1}+k_{0}}\dfrac{ER'}{16R}\]
  Ceka correspond au nombre \(k_{3}k_{2}k_{1}k_{0}\).
\end{cadre}

\begin{cadre}{Compteur de franges d'interférences}
  \paragraph*{}
  Le circuit électronique suivant permet de compter les franges d'interférences produites par un dispositif optique. On utilise un détecteur \((D)\) fonctionnant comme une source de courant parfaite débitant un courant de court-circuit proportionnel à l'intensité lumineuse reçue, et on admet que celle-ci est alors de la forme:
  \[i_{cc} = I_{0}(1+\gamma\cos(\omega t))\]
  Avec \(I_{0}=\SI{10}{\milli\ampere}\), \(\gamma=0,4\) et \(\omega=\SI{20}{\radian\per\second}\)

  \shorthandoff{:!}
  \begin{circuitikz}
    \node at (0,0) [en amp] (ao1) {A.O. 1};
    \node at (4,-2.2) [en amp, noinv input up] (ao2) {A.O. 2};
    \node at (8.5,-0.2) [en amp] (ao3) {A.O. 3};
    \draw (-2, -0.5) to[isource, i<=\(i_{cc}\)] ++(0,1);
    \node at (-2.2, 0.8) {\((D)\)};
    \draw (-1,1.5) to[generic, label=\(R_{0}\)] ++(2,0);
    \draw (2,0) to[generic, label=\(R\)] ++(0,-2);
    \draw (5.4,-2.2) to[generic, label=\(R_{1}\)] ++(2,0);
    \draw (7.4,-2.2) to[generic, label=\(R_{2}\)] ++(2,0);
    \draw (11.5,-4) to[generic, label=\(R_{3}\)] ++(0,2);
    \draw (11.5,0) to[generic, label=\(R_{4}\)] ++(0,-2);
    \draw (2,-4.2) node[eground]{} -- ++(0,0.2) to[C, label=\(C\)] ++(0,2);
    \draw (12.1, -4) to[diode] ++(0, 2);
    \node at (14, -2) [draw] (c) {Compteur};
    \draw (ao1.out) -| (2,0)   |- (ao3.-);
    \draw (ao1.-) |- (-2,0.5);
    \draw (ao1.-) |- (-1,1.5);
    \draw (ao1.+) |- (-2,-0.5);
    \draw (ao1.+) |- (2,-4);
    \draw (1,1.5) -| (2,0);
    \draw[thin] (ao2.+) -| (2,-2);
    \draw (ao2.-) |- (5.2, -3.6) |- (ao2.out);
    \draw (ao2.out) -- (5.9, -2.2);
    \draw (7.3, -2.2) |- (ao3.+);
    \draw (ao3.out) |- (11.5,0);
    \draw (9.4, -2.2) -| (ao3.out);
    \draw (11.5,-2) to[short, i=\( \)] node[pos=0.1, above]{\(i=0\)} (c.west);
    \draw (c.south) |- (2, -4);
    \draw[->, >= stealth, thick] (1.1, -3.8) -- node[midway, left]{\(v_{1}\)} ++(0, 3.6);
    \draw[->, >= stealth, thick] (2.5, -3.8) -- node[very near end, right]{\(v_{2}\)} ++(0, 1.8);
    \draw[->, >= stealth, thick] (5.5, -3.9) -- node[midway, right]{\(v'_{2}\)} ++(0, 1.6);
    \draw[->, >= stealth, thick] (10.5, -3.85) -- node[midway, left]{\(v_{4}\)} ++(0, 3.7);
    \draw[->, >= stealth, thick] (12.7, -3.85) -- node[midway, right]{\(v_{5}\)} ++(0, 1.7);
  \end{circuitikz}
  \shorthandon{:!}

  \begin{enumerate}[label=\upshape\alph*)]
    \item La diode et les A.O. sont considérés comme parfaits. Les tensions de saturation des A.O. valent \(\pm V_{s} = \SI{\pm 12}{\volt}\) pour les A.O. 2 et 3, et \(\SI{\pm 15}{\volt}\) pour l'A.O. 1. Enfin, les A.O. 1 et 2 fonctionnent en régime linéaire, et l'A.O. 3 en régime saturé. Décomposer le montage en sous-sensembles simples dont on précisera la fonction.
    \item Calculer \(v_{1}(t)\), sachant que \(R_{0}=\SI{1000}{\ohm}\).
    \item Si \(R=\SI{100}{\kilo\ohm}\) et \(C=\SI{100}{\micro\farad}\), calculer \(v'_{2}(t)\) en régime établi (faire les approximations qui s'imposent).
    \item Quelles valeurs peut prendre la tension d'entrée \(v+\) de l'A.O. 3, si \(R_{1}=\SI{1}{\kilo\ohm}\) et \(R_{2=}=\SI{100}{\kilo\ohm}\)?
    \item Pour simplifier les calcules, on fait l'approximation \(v^{+}=v'_{2}\). Tracer alors sur un même graphe les courbes \(v_{1}(t)\) et \(v_{4}(t)\).
    \item Si l'on veut que la valeur maximale de la tension \(v_{5}(t)\) à l'entrée du compteur vaille \(\SI{5}{\volt}\), quelle doit être la valeur du rapport \(\frac{R_{3}}{R_{4}}\)? Tracer sur un même graphe l'allure de \(v_{4}(t)\) et \(v_{5}(t)\).
  \end{enumerate}
\end{cadre}

\end{document}
