\documentclass[french, a4paper, 11pt]{article}

\usepackage[utf8]{inputenc} % ~ Encodage
\usepackage[T1]{fontenc}    % ~ Encodage
\usepackage[left=1cm, right=1cm]{geometry} % ~ Mise en page et marges
\usepackage{amssymb} % ~ Pour écrire les maths
\usepackage{xspace}  % ~ Commandes à texte
\usepackage{varioref} % ~ Références croisées
\usepackage{enumitem} % ~ Listes
\usepackage{xcolor}   % ~ Couleurs fs
\usepackage[version=4]{mhchem}
\usepackage{float}
\usepackage[load-configurations = abbreviations, output-decimal-marker={,}]{siunitx}
\usepackage{graphicx}
\usepackage[many]{tcolorbox}
\usepackage{euler}
\usepackage[french]{babel}


% ______FONCTIONS______%
\newcommand{\inv}[1]{\dfrac{1}{#1}}
% ______CHIMIE______%
\newcommand{\s}{{}_{(s)}}
\newcommand{\g}{{}_{(g)}}

\newtcolorbox{cadre}[2][]
{
  enhanced,
  attach boxed title to top left={yshift=-3.4mm, xshift = -2.3mm},
  adjusted title=#2,
  colback=white, colframe=black,
  colbacktitle=white, coltitle=black, fonttitle=\bfseries,
  breakable, sharp corners,
  boxed title style={colback=white, sharp corners, colframe=white},
  boxrule = 0.5mm, drop fuzzy shadow
}
\newcommand{\ooint}{\ocircle\hspace{-3.65mm}\int\hspace{-2mm}\int}

\title{Exercices de Thermochimie}
\author{Martin \textsc{Andrieux}, Nathan \textsc{Maillet}}
\date{}

\begin{document}
\maketitle

\begin{cadre}{Température de flamme adiabatique}
  Un chalumeau oxhydrique réalise la combustion d'un mélange d'hydrogène et d'oxygène dans les
  proportions st\oe{}chiométriques et sous la pression de \(\SI{1}{\bar}\). On donne pour :
    \[\ce{H2(g) + (1/2)O2(g) \rightarrow H2O(g)} \Delta_rH^{\circ}=\SI{-243}{\kilo\joule\per\mole^{-1}} \text{ à } \SI{20}{\celsius}.\]
  Pour la vapeur d'eau : \(C_p^{\circ}=C_0+\alpha \theta (\theta \text{ en } ^{\circ} \text{C})\), avec \(C_0=\SI{36.8}{\joule\per\mole\per\celsius} \text{ et }
  \alpha=\SI{0.013}{\joule\per\mole\per\celsius\squared}.\) La température initiale étant prise à \(\SI{20}{celsius}\), calculer la température maximale que peut atteindre la flamme.
  
  \tcblower
  \(\theta_{\text{max}}=\SI{3 918}{\celsius}\)
\end{cadre}

\begin{cadre}{Variétés allotropiques du soufre}
  Le soufre existe à l'état solide sous les variétés allotropiques \(S_{\alpha}\) et \(S_{\beta}\). Sous la pression \(P^0=\SI{1}{\bar}\),
  la température de transition est de \(\SI{95.5}{\celsius}\). Dans ces conditions, la différence d'entropie molaire est
  \({S_m(S_{\alpha})-S_m(S_{\beta})=\SI{7.87}{\joule\per\kelvin\per\mole}}\) et la différence de volume molaire est
  \({V_m(S_{\alpha})-V_m(S_{\beta})=\SI{0.8}{\centi\metre\cubed\per\mole}}\), ces valeurs étant supposées indépendantes de \(T\) et \(P\) dans le domaine considéré.
  \begin{enumerate}
    \item Rappeler, pour un corps pur, l'expression de d$\mu$ en fonction de $\mathrm{d}P$ et $\mathrm{d}T$.
    \item Quelle est la variété allotropique stable à \(\SI{25}{\celsius}\) et \(\SI{1}{\bar}\) ?
    \item Calculer l'élévation de la température de transition lorsque la pression augmente de \(\SI{10}{\bar}\).
  \end{enumerate}
  
  \tcblower
  \begin{enumerate}
    \item \(\mathrm{d}(\mu_{\alpha}^0-\mu_{\beta}^0)=-[S_{m\alpha}^0-S_{m\beta}^0]\mathrm{d}T\); il en découle que,
    si \(\mathrm{d}T < 0\), alors \(\mu_{\alpha}^0<\mu_{\beta}^0\), de sorte que \(S_{\alpha}\) est la variété stable.
    \item \(\Delta T \equiv \SI{0.1}{\kelvin}\)
  \end{enumerate}    
\end{cadre}

\begin{cadre}{Équilibres hétérogènes}

  À $\SI{820}{\celsius}$, on considère les équilibres :
  \begin{align*}
    \ce{CaCO3(s) &= CaO(s) + CO2(g)}  &K^\circ_1=0,2 \\
    \ce{MgCO3(s) &= MgO(s) + CO2(g)}  &K^\circ_2=0,4
  \end{align*}
  \begin{enumerate}
    \item Dans un cylindre maintenu à $\SI{820}{\celsius}$, de volume $V_0=\SI{22.72}{\liter}$,
      on introduit $\SI{0.1}{\mol}$ de \ce{CaCO3}.
      \begin{itemize}
        \item Calculer la composition du système dans l'état final.
        \item On augmente le volume $V$ du cylindre. Représenter en fonction de $V$ la pression $P$ et le nombre de moles de \ce{CaO}.
      \end{itemize}
    \item Dans un récipient vide de volume $V_0=\SI{22,72}{\liter}$, maintenu à $\SI{820}{\celsius}$, on place $\SI{0,1}{\mol}$ de CaCo, et on introduit progressivement du \ce{CO2}.
    Représenter la pression $P$ du système en fonction du nombre de moles de \ce{CO2} introduites.
    \item Dans un cylindre de volume très grand, initialement vide et maintenu à $\SI{820}{\celsius}$, on introduit une mole de \ce{CaO}, une autre de \ce{MgO} et 3 moles de \ce{CO2}.
      \begin{itemize}
        \item Quelle est la variance du système ? Commenter.
        \item À l'aide d'un piston, on comprime lentement le système.
          Étudier et tracer la courbe donnant la pression $P$ en fonction du volume $V$ du cylindre.
      \end{itemize}
  \end{enumerate}

\end{cadre}
\end{document}
