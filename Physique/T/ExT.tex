\documentclass[french, a4paper, 11pt]{article}

\usepackage[utf8]{inputenc} % ~ Encodage
\usepackage[T1]{fontenc}    % ~ Encodage
\usepackage[left=1cm, right=1cm]{geometry} % ~ Mise en page et marges
\usepackage{amssymb} % ~ Pour écrire les maths
\usepackage{xspace}  % ~ Commandes à texte
\usepackage{varioref} % ~ Références croisées
\usepackage{enumitem} % ~ Listes
\usepackage{xcolor}   % ~ Couleurs fs
\usepackage[version=4]{mhchem}
\usepackage{float}
\usepackage[load-configurations = abbreviations, output-decimal-marker={,}]{siunitx}
\usepackage{graphicx}
\usepackage[many]{tcolorbox}
\usepackage{euler}
\usepackage[french]{babel}


% ______FONCTIONS______%
\newcommand{\inv}[1]{\dfrac{1}{#1}}
% ______CHIMIE______%
\newcommand{\s}{{}_{(s)}}
\newcommand{\g}{{}_{(g)}}

\newtcolorbox{cadre}[2][]
{
  enhanced,
  attach boxed title to top left={yshift=-3.4mm, xshift = -2.3mm},
  adjusted title=#2,
  colback=white, colframe=black,
  colbacktitle=white, coltitle=black, fonttitle=\bfseries,
  breakable, sharp corners,
  boxed title style={colback=white, sharp corners, colframe=white},
  boxrule = 0.5mm, drop fuzzy shadow
}
\newcommand{\ooint}{\ocircle\hspace{-3.65mm}\int\hspace{-2mm}\int}

\title{Exercices de Thermochimie}
\author{Martin \textsc{Andrieux}, Nathan \textsc{Maillet}}
\date{}

\begin{document}
\maketitle

\begin{cadre}{Température de flamme adiabatique}
  Un chalumeau oxhydrique réalise la combustion d'un mélange d'hydrogène et d'oxygène dans les
  proportions st\oe{}chiométriques et sous la pression de \(\SI{1}{\bar}\). On donne pour :
    \[\ce{H2(g) + (1/2)O2(g) \rightarrow H2O(g)} \Delta_rH^{\circ}=\SI{-243}{\kilo\joule\per\mole^{-1}} \text{ à } \SI{20}{\celsius}.\]
  Pour la vapeur d'eau : \(C_p^{\circ}=C_0+\alpha \theta (\theta \text{ en } ^{\circ} \text{C})\), avec \(C_0=\SI{36.8}{\joule\per\mole\per\celsius} \text{ et }
  \alpha=\SI{0.013}{\joule\per\mole\per\celsius\squared}.\) La température initiale étant prise à \(\SI{20}{celsius}\), calculer la température maximale que peut atteindre la flamme.
  
  \tcblower
  \(\theta_{\text{max}}=\SI{3 918}{\celsius}\)
\end{cadre}

\begin{cadre}{Variétés allotropiques du soufre}
  Le soufre existe à l'état solide sous les variétés allotropiques \(S_{\alpha}\) et \(S_{\beta}\). Sous la pression \(P^0=\SI{1}{\bar}\),
  la température de transition est de \(\SI{95.5}{\celsius}\). Dans ces conditions, la différence d'entropie molaire est
  \({S_m(S_{\alpha})-S_m(S_{\beta})=\SI{7.87}{\joule\per\kelvin\per\mole}}\) et la différence de volume molaire est
  \({V_m(S_{\alpha})-V_m(S_{\beta})=\SI{0.8}{\centi\metre\cubed\per\mole}}\), ces valeurs étant supposées indépendantes de \(T\) et \(P\) dans le domaine considéré.
  \begin{enumerate}
    \item Rappeler, pour un corps pur, l'expression de d$\mu$ en fonction de $\mathrm{d}P$ et $\mathrm{d}T$.
    \item Quelle est la variété allotropique stable à \(\SI{25}{\celsius}\) et \(\SI{1}{\bar}\) ?
    \item Calculer l'élévation de la température de transition lorsque la pression augmente de \(\SI{10}{\bar}\).
  \end{enumerate}
  
  \tcblower
  \begin{enumerate}
    \item \(\mathrm{d}(\mu_{\alpha}^0-\mu_{\beta}^0)=-[S_{m\alpha}^0-S_{m\beta}^0]\mathrm{d}T\); il en découle que,
    si \(\mathrm{d}T < 0\), alors \(\mu_{\alpha}^0<\mu_{\beta}^0\), de sorte que \(S_{\alpha}\) est la variété stable.
    \item \(\Delta T \equiv \SI{0.1}{\kelvin}\)
  \end{enumerate}    
\end{cadre}

\begin{cadre}{Dissociation de \(\ce{N2O4}\)}
  On introduit \(\SI{1.15}{\gram}\) de \(\ce{N2O4}\) solide dans un récipient \(A\) vide que l'on porte ensuite à
  la température de \(\SI{25}{\celsius}\). \(\ce{N2O4}\) se vaporise totalement et se dissocie en partie selon \(\left(1\right) : \ce{N2O4(g)=2NO2(g)}\).
  Le volume de \(A\) est \(V=\SI{1}{\liter}\), et la pression à l'équilibre vaut \(\SI{0.4}{\bar}\). A l'équilibre, \(n_{\ce{NO2}}=\SI{7.29e-3}{\mole}\).
  \begin{enumerate}
    \item Compléter le tableau, donnant les enthalpies et enthalpies libres standard de formations à \(\SI{25}{\celsius}\) :
      \begin{tabular}{|l|l|l|l}
        \hline Composé & \(\ce{NO2(g)}\) & \(\ce{N2O4(g)}\) \\
        \hline \(\Delta_fH^0(\si{\kilo\joule\per\mole})\) & \(33.4\) & \(12.5\) \\
        \hline \(\Delta_fG^0(\si{\kilo\joule\per\mole})\) & \(52.3\) & ? \\
        \hline
      \end{tabular}
    \item Calculer l'entropie standard de réaction pour la réaction \(\left(1\right)\). Le signe pouvait-il être prévu ?
    \item Un deuxième récipient \(B\) de même volume que \(A\) dans lequel l'équilibre précédent est réalisé. On brise la paroi 
    commune aux deux récipients : le mélange des gaz entraîne une nouvelle réaction équilibrée :
      \[\ce{NO(g) + NO2(g) = N2O3(g)} \text{ de constante } K^0.\]
    Lorsque l'équilibre final est atteint, la pression vaut \(\SI{0.386}{\bar} \left( \text{la température étant maintenue à } \SI{25}{\celsius}\right)\).
      \begin{itemize}
        \item Montrer que la somme des pressions partielles \(P_{\ce{NO}}+P_{\ce{N2O4}}\) peut être atteinte sans calcul.
        \item Calculer \(P_{\ce{N2O3}},P_{\ce{NO}}\) et l'ordre de grandeur de \(K^0\).
      \end{itemize}
  \end{enumerate}

  \tcblower
  \begin{enumerate}
    \item \(\Delta_fG^0(\ce{N2O4})=\SI{99.9}{\kilo\joule\per\mole}\)
    \item \(\Delta_rS^1=\SI{166.4}{\joule\per\kelvin\per\mole}\)
    \item \(P_{\ce{NO2}}=\SI{0.116}{\bar}\) et \(P_{\ce{N2O4}}=\SI{0.090}{\bar}\)
  \end{enumerate}
\end{cadre}

\begin{cadre}{Dissociation du pentachlorure de phosphore}
  On considère l'équilibre suivant : \(\ce{PCl5(g) = PCl3(g) + Cl2(g)}\)
  \begin{enumerate}
    \item Indiquer l'influence d'une élévation isotherme de \(P\), d'une augmentation isobare de \(T\),
    d'une introduction isotherme et isobare de \(\ce{PCl3}\), de \(\ce{Cl2}\), de \(\ce{PCl5}\) ou d'un gaz inactif.
    \item Sous une pression constante \(P=\SI{3}{\bar}\) et à \(\SI{500}\kelvin\), on mélange \(\SI{0.1}{\mole}\)
    de \(\ce{Cl2}, \SI{0.4}{\mole}\) de \(\ce{PCl3}\) et \(\SI{0.15}{\mole}\) de \(\ce{PCl5}\).
    A l'équilibre, \(\xi = \SI{4.39e-2}{\mole}\). Calculer la température d'inversion \(T_i\) de la réaction.
  \end{enumerate}

  \tcblower
  \begin{enumerate}
    \item \(K^0\) croît avec une augmentation de \(T\) ou la quantité de \(\ce{PCl5}\) ou d'un gaz inerte et 
    décroît avec une augmentation de \(P\) ou de \(\ce{PCl3}\) ou \(\ce{Cl2}\)
  \end{enumerate}
\end{cadre}

\begin{cadre}{Équilibres hétérogènes}
  À $\SI{820}{\celsius}$, on considère les équilibres :
  \begin{align*}
    \ce{CaCO3(s) &= CaO(s) + CO2(g)}  &K^\circ_1=0,2 \\
    \ce{MgCO3(s) &= MgO(s) + CO2(g)}  &K^\circ_2=0,4
  \end{align*}
  \begin{enumerate}
    \item Dans un cylindre maintenu à $\SI{820}{\celsius}$, de volume $V_0=\SI{22.72}{\liter}$,
      on introduit $\SI{0.1}{\mol}$ de \ce{CaCO3}.
      \begin{itemize}
        \item Calculer la composition du système dans l'état final.
        \item On augmente le volume $V$ du cylindre. Représenter en fonction de $V$ la pression $P$ et le nombre de moles de \ce{CaO}.
      \end{itemize}
    \item Dans un récipient vide de volume $V_0=\SI{22,72}{\liter}$, maintenu à $\SI{820}{\celsius}$, on place $\SI{0,1}{\mol}$ de CaCo, et on introduit progressivement du \ce{CO2}.
    Représenter la pression $P$ du système en fonction du nombre de moles de \ce{CO2} introduites.
    \item Dans un cylindre de volume très grand, initialement vide et maintenu à $\SI{820}{\celsius}$, on introduit une mole de \ce{CaO}, une autre de \ce{MgO} et 3 moles de \ce{CO2}.
      \begin{itemize}
        \item Quelle est la variance du système ? Commenter.
        \item À l'aide d'un piston, on comprime lentement le système.
          Étudier et tracer la courbe donnant la pression $P$ en fonction du volume $V$ du cylindre.
      \end{itemize}
  \end{enumerate}

  \tcblower
  \begin{enumerate}
    \item A la fin, \(\xi=0.05\). Au début \(\xi\) augmente et \(P=\SI{0.2}{\bar}\) puis en \(\xi=0.1\), \(P=\frac{0.1RT}{V}\) donc diminue
    \item \(P\) augmente en faisant un plateau quand \(n \in [0.05,0.15]\)
    \item \(V=0\). \(P\) diminue en faisant deux plateau, l'un quand \(V\in [10V_0,20V_0]\) et l'autre quand \(V\in [40V_0,60V_0]\)
  \end{enumerate}
\end{cadre}
\end{document}
