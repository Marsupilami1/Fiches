\documentclass[french, a4paper, 11pt]{article}

\usepackage[utf8]{inputenc} % ~ Encodage
\usepackage[T1]{fontenc}    % ~ Encodage
\usepackage[left=1cm, right=1cm, bottom=3cm]{geometry} % ~ Mise en page et marges
\usepackage{amssymb} % ~ Pour écrire les maths
\usepackage{xspace}  % ~ Commandes à texte
\usepackage{varioref} % ~ Références croisées
\usepackage{enumitem} % ~ Listes
\usepackage{xcolor}   % ~ Couleurs fs
\usepackage{tabularx}
\usepackage{float}
\usepackage{tikz}
\usepackage[straightvoltages]{circuitikz}
\usepackage[load-configurations = abbreviations]{siunitx}
\usepackage{mhchem}
\usepackage{graphicx}
\usepackage[f]{esvect}
\usepackage[many]{tcolorbox}
\usepackage{euler}
\usepackage[nointegrals]{wasysym}
\usepackage[french]{babel}


\sisetup{
  locale=FR,
  detect-all,
}

%______FONCTIONS______%
\newcommand{\ssi}{si et seulement si\xspace}		% ~ ssi
% --ENSEMBLES--%
\newcommand{\N}{\mathbb{N}}   % ~ Entiers naturels
\newcommand{\Z}{\mathbb{Z}}   % ~ Entiers relatifs
\newcommand{\D}{\mathbb{D}}   % ~ Decimaux
\newcommand{\Q}{\mathbb{Q}}   % ~ Rationnels
\newcommand{\R}{\mathbb{R}}   % ~ Réels
\newcommand{\C}{\mathbb{C}}   % ~ Complexes
% --TRIGO--%
\let\cosh\relax
\DeclareMathOperator{\cosh}{ch}       % ~ cosinus hyperbolique
\DeclareMathOperator{\sh}{sh}         % ~ sinus hyperbolique
\let\tanh\relax
\DeclareMathOperator{\tanh}{th}       % ~ tangente hyperbolique
\DeclareMathOperator{\argch}{Argch}   % ~ Argument cosinus hyperbolique
\DeclareMathOperator{\argsh}{Argsh}   % ~ Argument sinus hyperbolique
\DeclareMathOperator{\argth}{Argth}   % ~ Argument tangente hyperbolique
\DeclareMathOperator{\cotan}{cotan}   % ~ cotangente
% --PARENTHESES--%
\newcommand{\po}{\left(}         % ~ (
\newcommand{\pf}{\right)}        % ~ )
\newcommand{\pof}[1]{\po #1 \pf} % ~ ( )
\newcommand{\co}{\left[}         % ~ [
\newcommand{\cf}{\right]}        % ~ ]
\newcommand{\cof}[1]{\co #1 \cf} % ~ [ ]
\newcommand{\chof}[1]{\left\langle #1 \right\rangle } % ~ < >
\newcommand{\interoo}[2]{\left]#1\,;#2\right[}   % ~ ]a,b[
\newcommand{\interof}[2]{\left]#1\,;#2\right]}   % ~ ]a,b]
\newcommand{\interfo}[2]{\left[#1\,;#2\right[}   % ~ [a,b[
\newcommand{\interff}[2]{\left[#1\,;#2\right]}   % ~ [a,b]
% --VECTEURS--%
\newcommand{\ux}{\vect{u_x}}          % ~ Vecteur ux
\newcommand{\uy}{\vect{u_y}}          % ~ Vecteur uy
\newcommand{\uz}{\vect{u_z}}          % ~ Vecteur uz
\newcommand{\ur}{\vect{u_r}}          % ~ Vecteur ur
\newcommand{\uth}{\vect{u_\theta}}    % ~ Vecteur utheta
\newcommand{\uph}{\vect{u_\varphi}}   % ~ Vecteur uphi
\newcommand{\om}{\vect{OM}}           % ~ Vecteur position
\newcommand{\vvi}{\vect{v}}           % ~ Vecteur vitesse
\newcommand{\vvio}{\vect{v_0}}        % ~ Vecteur v0
\newcommand{\va}{\vect{a}}            % ~ Vecteur	accélération
\newcommand{\vp}{\vect{p}}            % ~ Vecteur quantité de mouvement
\newcommand{\fr}{\vect{F_r}}          % ~ Vecteur force de rappel
\newcommand{\vabla}{\vect{\nabla}}    % ~ nabla
\newcommand{\grad}{\vect{\mathrm{grad}}}  % ~ grad
\DeclareMathOperator{\diverg}{div}        % ~ grad
\newcommand{\rot}{\vect{\mathrm{rot}}}    % ~ grad

\newtcolorbox{cadre}[2][]
{
  enhanced,
  attach boxed title to top left={yshift=-3.4mm, xshift = -2.3mm},
  adjusted title=#2,
  colback=white, colframe=black,
  colbacktitle=white, coltitle=black, fonttitle=\bfseries,
  breakable, sharp corners,
  boxed title style={colback=white, sharp corners, colframe=white},
  boxrule = 0.5mm, drop fuzzy shadow
}
\newcommand{\ooint}{\ocircle\hspace{-3.65mm}\int\hspace{-2mm}\int}

\title{Exercices d'électromagnétisme}
\author{Martin \textsc{Andrieux}}
\date{}

\begin{document}
\maketitle

\begin{cadre}{Physique sur un lac}
  \paragraph*{}
  Les eaux d'un lac (de masse volumique \(\mu\)) s'abaissent d'une hauteur \(h=\SI{1}{\meter}\). Calculer la variation \(\Delta g\) qu'enregistre un gravimètre placé:
  \begin{itemize}[label=\(\bullet\)]
    \item Sur des pilotis, au milieu du lac, juste au dessus de la surface (avant qu'il ne baisse),
    \item à bord d'une barque ancrée au milieu du lac.
  \end{itemize}
  La rayon terrestre est \(R=\SI{6400}{\kilo\meter}\), et le champ de pesanteur à l'altitude du lac est \(g=\SI{9,8}{\meter\per\square\second}\)
  \tcblower
  \begin{tabularx}{\linewidth}{Xr}
    \(\Delta g = -2\pi G\mu h = \SI{-0.42e-6}{\meter\per\square\second}\)&
    \(\Delta g' = \Delta g + \dfrac{2gh}{R} = \SI{2.64e-6}{\meter\per\square\second}\)
  \end{tabularx}
\end{cadre}

\begin{cadre}{Cosmogonie du système solaire}
  \paragraph*{}
  Selon l'hypothèse de Laplace, le système solaire aurait été, à un moment de son évolution, constitué d'un tore fluide homogène de masse volumique \(mu\) et d'axe \((D)\), animé d'un mouvement de rotation uniforme de vitesse angulaire \(\omega\) autour de \((D)\).
  \begin{enumerate}[label=\upshape\alph*)]
    \item On note \(\vv h\) le champ de gravitation, \(V\) le potetiel dont dérive \(\vv h\), et \(G\) la constante de gravitation universelle. Quelle et l'équation locale vérifiée par \(V\)?
    \item On note \(\vv a\) le champ (massique) des forces d'inerties d'entraînement dans le référentiel lié au fluide, et \(U\) l potentiel dont dérice \(\vv a\). Calculer le laplacien \(\Delta U\).
    \item On admet que la stabilité du système exige que, en tout point de la surface du tore, le champ total \(\vv h + \vv a\) soit dirigé vers l'intérieur de celui-ci. Montrer que \(\omega\) est nécessairement inférieur à une valeur que l'on calculera en fonction de \(G\) et \(\mu\).
  \end{enumerate}
  \tcblower
  \begin{tabularx}{\linewidth}{XXr}
    \(\Delta V -4\pi G\mu = 0\) & \(\Delta U = -2\omega^{2}\) & \(\omega^{2} < 2\pi G \mu\)
  \end{tabularx}
\end{cadre}

\begin{cadre}{Particules colloîdales dans un électrolyte}
  Dans un colloïde, des particules chargées sont en suspension dans un électrolyte. Pourétudier l'effet de ions dur le champ et le potentiel électrique, on considère en coordonnées sphériques une particule colloîdale de centre \(O\), de rayon \(R\), portant une charge \(Q\) uniformément répartie sur sa surface.
  Les ions de charge \(\pm e\) (cations ou anions), auraint en l'abscence de paticule colloîdale la même densité particulaire \(n_{0}\). Soumis à un potentiel \(V(r)\), un ion de qharge \(a\) a (en raison de la loi de Boltzmann) la densité particulaire \(n_{0}\exp\pof{-\frac{qV(r)}{k_{B}T}}\).
  \begin{enumerate}[label=\upshape\alph*)]
    \item Exprimer, pour \(r>R\), la densité volumique de charge \(\rho(r)\) en fonction du potentiel \(V(r)\), de \(n_0\), \(T\) et de constantes. Simplifier cette expression en faisant l'hypothèse que \(eV(r)\ll k_{B}T\) (cela signifie que l'agitation thermique est prépondérente).
    \item Dans ce cas, calculer le potentiel \(V(r)\), en posant \(U(r) = rV(r)\) et \(\lambda^{2} = \frac{\varepsilon_{0}k_{B}T}{2n_{0}e^{2}}\).
      On donne le laplacien en coordonnées sphériques pour \(V(r)\) :
      \[\Delta V = \dfrac{1}{r^{2}}\dfrac{d}{dr}\pof{r^{2}\dfrac{dV}{dr}}\]
    \item Quelle est la signification de \(\lambda\), appelé \emph{longueur de Debye} de la suspension colloïdale ? Que devient \(V(r)\) si \(\lambda\) tend vers zéro ou l'infini? Calculer \(\lambda\) pour de l'eau pure à \(\SI{300}{\kelvin}\), et pour une solution molaire de \(\ce{NaCl}\). (On donne \(k_{B} = \SI{1.38e-23}{\joule\per\kelvin}\))
  \end{enumerate}
  Rappel: Un colloïde est une suspension d'une ou plusieurs substances dispersées régulièrement dans un liquide, formant un système à deux phases séparées.
  \tcblower
  \begin{enumerate}[label=\upshape\alph*)]
    \item \(\rho(r) \approx -2e^{2}n_{0}\dfrac{V(r)}{k_{B}T}\)
    \item Avec l'équation de Poisson : \(V(r) = \dfrac{A}{r}\exp\pof{\dfrac{-r}{\lambda}}\). On trouve la constante \(A\) en écrivant le champ életrique pour \(r=R\). Alors
      \[V(r) = \dfrac{Q\exp\pof{\frac{R-r}{\lambda}}}{4\pi\varepsilon_{0}r\pof{1+\frac{R}{\lambda}}}\]
    \item Pour l'eau pure, \(\lambda = \SI{0.15}{\micro\meter}\); pour \(\ce{NaCl}\) molaire, \(\lambda=\SI{49}{\pico\meter}\), de l'ordre d'un rayon ionique.
  \end{enumerate}
\end{cadre}

\begin{cadre}{Distribution quadripolaire : molécule de dioxyde de carbone}
  Compte tenu des différences d'électronégativité de l'oxygène et du carbone, on peut schématiser la molécule de dioxyde de carbone, d'un point de vue électrostatique, par trois charges alignées sur un axe $Oz$ : $+2q$ en $O$, $-q$ aux points d'abscisse $a$ et $-a$.
  \begin{enumerate}[label=\upshape\alph*)]
    \item Calculer le potentiel et le champ en un point $M$ éloigné de $O$ (on pose $OM=r$ et $(Oz, OM) = \theta$).
    \item Quelle est l'équation des lignes de champ?
    \item Tracer l'allure de la carte de champ avec les équipotentielles.
  \end{enumerate}
  \tcblower
  \begin{tabularx}{\linewidth}{Xr}
    $V(M) = qa^2\dfrac{1-3\cos^{2}(\theta)}{4\pi\varepsilon_{0}r^{3}}$ & $r=k\lvert\sin(\theta)\rvert\sqrt{\lvert\cos(\theta)\rvert}$
  \end{tabularx}
\end{cadre}

\begin{cadre}{Répartition surfacique de dipôles sur un disque}
  \paragraph*{}
  Un disque de centre \(O\) et de rayon \(R\) porte, répartis uniformément sur sa surface, des dipôles électriques dont les moments dipolaires lui sont orthogonaux. Soit \(\mu = \frac{dp}{dS}\) la densité surfacique de moment dipolaire.

  Calculer le champ et le potentiel en tout point de l'axe de révolution \(Oz\) du disque (plusieures méthodes sont possibles). Que deviennent ces résultats pour \(z\gg R\) ?
  \tcblower
  \begin{tabularx}{\linewidth}{Xr}
    \(V(z) = \dfrac{\mu}{2\varepsilon_{0}}\pof{\dfrac{z}{\lvert z\rvert}-\dfrac{z}{\sqrt{z^{2}+R^{2}}}}\) &
    \(E(z) = \dfrac{\mu R^{2}}{2\varepsilon_{0}}\pof{r^{2}+R^{2}}^{-\frac{3}{2}}\)
  \end{tabularx}
\end{cadre}

\begin{cadre}{Expérience de Nichols}
  \paragraph*{}
  Un métal contient par unité de volume lorsqu'il est immobile \(n_{0}\) ions positifs de charge \(e\) et \(n_{0}\) électrons libres de charge \(-e\) et de masse \(m\). Un long cylindre de ce métal, de rayon \(a\), est mis en rotation autour de son axe de révolution \(Oz\) avec la vitesse angulaire constante \(\omega\). À l'équilibre, les ions et les électrons sont entraînés à la vitesse de rotation \(\omega\), et n'ont donc pas de mouvement par rapport au métal. À l'équilibre, il apparaît une densité volumique de charge \(\rho(r)\) dans le cylindre, ainsi qu'une densité surfacique \(\sigma\) à la surface de celui-ci. En coordonnées cylindriques, on a pour un champ radial : \(\diverg \vv E = \frac{1}{r}\frac{d(rE_{r})}{dr}\).
  \begin{itemize}[label=\(\bullet\)]
    \item Calculer le champ électrique dans le métal, et en déduire la différence de potentiel \(U\) entre l'axe du cylindre et sa périphérie.
    \item En déduire la densité \(n(r)\) des électrons libres dans le volume du métal, et le charge surfacique \(\sigma\).
  \end{itemize}
  \tcblower
  \begin{tabularx}{\linewidth}{XXr}
    \(U = \dfrac{m\omega^{2}a^{2}}{2e}\) &
    \(n(r) = n_{0} - \dfrac{2m\varepsilon_{0}\omega^{2}}{e^{2}}\) &
    \(\sigma = -\dfrac{m\varepsilon_{0}\omega^{2}a}{e}\)
  \end{tabularx}
\end{cadre}

\begin{cadre}{Association de condensateurs et bilan d'énergie}
  \begin{minipage}{0.65\linewidth}
    \paragraph*{}
    On étudie le système représenté ci-contre, où tous les condensateurs ont même capacité \(C\). Quelle est la charge de chacun d'entre eux?

    \paragraph*{}
    On introduit un quatrième condensateur de capacité \(C\) entre \(A\) et \(B\). Il a été au préalable chargé sous la tension \(U\) positive et son armature chargée positivement est placé du côté de \(A\). Quelles sont les nouvelles charges de chaque condensateur à l'équilibre?

    \paragraph*{}
    Faire un bilan d'énergie entre l'état initial de la question précedente et l'état final.
  \end{minipage}
  \begin{minipage}{0.27\linewidth}
    \shorthandoff{:!}
    \begin{circuitikz}
      \draw (4,0) node{\(\bullet\)} node[below]{\(N\)} to[short] (2,0) node[above right]{\(B\)} node{\(\bullet\)} to[short] (0,0) node{\(\bullet\)} to[C, label=3] (0,2) to[short] (2,2) node[below right]{\(A\)} to[C, label=2, *-] (2,4) to[short] (4,4) node{\(\bullet\)} node[above]{\(M\)};
      \draw (0,2) to[C, label=1] (0,4) to[short] (2,4);
      \draw[->, >=stealth] (4,0.2) -- node[midway, left]{\(U\)} (4,3.8);
    \end{circuitikz}
    \shorthandon{:!}
  \end{minipage}
  \tcblower
  \begin{tabularx}{\linewidth}{Xr}
    \(2q_1=2q_2=q_3=2\dfrac{CU}{3}\)
    & \(q_1'=q_2'=\frac{CU}{4}\) et \(q_3'=q_4'=3\frac{CU}{4}\) \\
    L'énergie électrostatique diminue de \(\frac{5CU^2}{24}\), et les pertes par effet Joule valent \(\frac{CU^{2}}{24}\)
  \end{tabularx}
\end{cadre}

\end{document}
