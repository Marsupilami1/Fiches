\documentclass[french, a4paper, 11pt]{article}

\usepackage[utf8]{inputenc} % ~ Encodage
\usepackage[T1]{fontenc}    % ~ Encodage
\usepackage[left=1cm, right=1cm, bottom=3cm]{geometry} % ~ Mise en page et marges
\usepackage{amssymb} % ~ Pour écrire les maths
\usepackage{xspace}  % ~ Commandes à texte
\usepackage{varioref} % ~ Références croisées
\usepackage{enumitem} % ~ Listes
\usepackage{xcolor}   % ~ Couleurs fs
\usepackage{tabularx}
\usepackage{float}
\usepackage{tikz}
\usepackage[straightvoltages]{circuitikz}
\usepackage[load-configurations = abbreviations]{siunitx}
\usepackage{graphicx}
\usepackage[f]{esvect}
\usepackage[many]{tcolorbox}
\usepackage{euler}
\usepackage[nointegrals]{wasysym}
\usepackage[french]{babel}


\sisetup{
  locale=FR,
  detect-all,
}

%______FONCTIONS______%
\newcommand{\ssi}{si et seulement si\xspace}		% ~ ssi
% --ENSEMBLES--%
\newcommand{\N}{\mathbb{N}}   % ~ Entiers naturels
\newcommand{\Z}{\mathbb{Z}}   % ~ Entiers relatifs
\newcommand{\D}{\mathbb{D}}   % ~ Decimaux
\newcommand{\Q}{\mathbb{Q}}   % ~ Rationnels
\newcommand{\R}{\mathbb{R}}   % ~ Réels
\newcommand{\C}{\mathbb{C}}   % ~ Complexes
% --TRIGO--%
\let\cosh\relax
\DeclareMathOperator{\cosh}{ch}       % ~ cosinus hyperbolique
\DeclareMathOperator{\sh}{sh}         % ~ sinus hyperbolique
\let\tanh\relax
\DeclareMathOperator{\tanh}{th}       % ~ tangente hyperbolique
\DeclareMathOperator{\argch}{Argch}   % ~ Argument cosinus hyperbolique
\DeclareMathOperator{\argsh}{Argsh}   % ~ Argument sinus hyperbolique
\DeclareMathOperator{\argth}{Argth}   % ~ Argument tangente hyperbolique
\DeclareMathOperator{\cotan}{cotan}   % ~ cotangente
% --PARENTHESES--%
\newcommand{\po}{\left(}         % ~ (
\newcommand{\pf}{\right)}        % ~ )
\newcommand{\pof}[1]{\po #1 \pf} % ~ ( )
\newcommand{\co}{\left[}         % ~ [
\newcommand{\cf}{\right]}        % ~ ]
\newcommand{\cof}[1]{\co #1 \cf} % ~ [ ]
\newcommand{\chof}[1]{\left\langle #1 \right\rangle } % ~ < >
\newcommand{\interoo}[2]{\left]#1\,;#2\right[}   % ~ ]a,b[
\newcommand{\interof}[2]{\left]#1\,;#2\right]}   % ~ ]a,b]
\newcommand{\interfo}[2]{\left[#1\,;#2\right[}   % ~ [a,b[
\newcommand{\interff}[2]{\left[#1\,;#2\right]}   % ~ [a,b]
% --VECTEURS--%
\newcommand{\ux}{\vect{u_x}}          % ~ Vecteur ux
\newcommand{\uy}{\vect{u_y}}          % ~ Vecteur uy
\newcommand{\uz}{\vect{u_z}}          % ~ Vecteur uz
\newcommand{\ur}{\vect{u_r}}          % ~ Vecteur ur
\newcommand{\uth}{\vect{u_\theta}}    % ~ Vecteur utheta
\newcommand{\uph}{\vect{u_\varphi}}   % ~ Vecteur uphi
\newcommand{\om}{\vect{OM}}           % ~ Vecteur position
\newcommand{\vvi}{\vect{v}}           % ~ Vecteur vitesse
\newcommand{\vvio}{\vect{v_0}}        % ~ Vecteur v0
\newcommand{\va}{\vect{a}}            % ~ Vecteur	accélération
\newcommand{\vp}{\vect{p}}            % ~ Vecteur quantité de mouvement
\newcommand{\fr}{\vect{F_r}}          % ~ Vecteur force de rappel
\newcommand{\vabla}{\vect{\nabla}}    % ~ nabla
\newcommand{\grad}{\vect{\mathrm{grad}}}  % ~ grad
\DeclareMathOperator{\diverg}{div}        % ~ grad
\newcommand{\rot}{\vect{\mathrm{rot}}}    % ~ grad

\newtcolorbox{cadre}[2][]
{
  enhanced,
  attach boxed title to top left={yshift=-3.4mm, xshift = -2.3mm},
  adjusted title=#2,
  colback=white, colframe=black,
  colbacktitle=white, coltitle=black, fonttitle=\bfseries,
  breakable, sharp corners,
  boxed title style={colback=white, sharp corners, colframe=white},
  boxrule = 0.5mm, drop fuzzy shadow
}
\newcommand{\ooint}{\ocircle\hspace{-3.65mm}\int\hspace{-2mm}\int}

\title{Exercices d'électromagnétisme}
\author{Martin \textsc{Andrieux}}
\date{}

\begin{document}
\maketitle

\begin{cadre}{Physique sur un lac}
  \paragraph*{}
  Les eaux d'un lac (de masse volumique $\mu$) s'abaissent d'une hauteur $h=\SI{1}{\meter}$. Calculer la variation $\Delta g$ qu'enregistre un gravimètre placé:
  \begin{itemize}[label=$\bullet$]
    \item Sur des pilotis, au milieu du lac, juste au dessus de la surface (avant qu'il ne baisse),
    \item à bord d'une barque ancrée au milieu du lac.
  \end{itemize}
  La rayon terrestre est $R=\SI{6400}{\kilo\meter}$, et le champ de pesanteur à l'altitude du lac est $g=\SI{9,8}{\meter\per\square\second}$
  \tcblower
  \begin{tabularx}{\linewidth}{Xr}
    $\Delta g = -2\pi G\mu h = \SI{-0.42e-6}{\meter\per\square\second}$&
    $\Delta g' = \Delta g + \dfrac{2gh}{R} = \SI{2.64e-6}{\meter\per\square\second}$
  \end{tabularx}
\end{cadre}

\begin{cadre}{Répartition surfacique de dipôles sur un disque}
  \paragraph*{}
  Un disque de centre $O$ et de rayon $R$ porte, répartis uniformément sur sa surface, des dipôles électriques dont les moments dipolaires lui sont orthogonaux. Soit $\mu = \frac{dp}{dS}$ la densité surfacique de moment dipolaire.

  Calculer le champ et le potentiel en tout point de l'axe de révolution $Oz$ du disque (plusieures méthodes sont possibles). Que deviennent ces résultats pour $z\gg R$ ?
  \tcblower
  \begin{tabularx}{\linewidth}{Xr}
    $V(z) = \dfrac{\mu}{2\varepsilon_{0}}\pof{\dfrac{z}{\lvert z\rvert}-\dfrac{z}{\sqrt{z^{2}+R^{2}}}}$ &
    $E(z) = \dfrac{\mu R^{2}}{2\varepsilon_{0}}\pof{r^{2}+R^{2}}^{-\frac{3}{2}}$
  \end{tabularx}
\end{cadre}

\begin{cadre}{Ecpérience de Nichols}
  \paragraph*{}
  Un métal contient par unité de volume lorsqu'il est immobile $n_{0}$ ions positifs de charge $e$ et $n_{0}$ électrons libres de charge $-e$ et de masse $m$. Un long cylindre de ce métal, de rayon $a$, est mis en rotation autour de son axe de révolution $Oz$ avec la vitesse angulaire constante $\omega$. À l'équilibre, les ions et les électrons sont entraînés à la vitesse de rotation $\omega$, et n'ont donc pas de mouvement par rapport au métal. À l'équilibre, il apparaît une densité volumique de charge $\rho(r)$ dans le cylindre, ainsi qu'une densité surfacique $\sigma$ à la surface de celui-ci. En coordonnées cylindriques, on a pour un champ radial : $\diverg \vv E = \frac{1}{r}\frac{d(rE_{r})}{dr}$.
  \begin{itemize}[label=$\bullet$]
    \item Calculer le champ électrique dans le métal, et en déduire la différe,ce de potentiel $U$ entre l'axe du cylindre et sa périphérie.
    \item En déduire la densité $n(r)$ des électrons libres dans le volume du métal, et le charge surfacique $\sigma$.
  \end{itemize}
  \tcblower
  \begin{tabularx}{\linewidth}{XXr}
    $U = \dfrac{m\omega^{2}a^{2}}{2e}$ &
    $n(r) = n_{0} - \dfrac{2m\varepsilon_{0}\omega^{2}}{e^{2}}$ &
    $\sigma = -\dfrac{m\varepsilon_{0}\omega^{2}a}{e}$
  \end{tabularx}
\end{cadre}

\end{document}
