\documentclass[french, a4paper, 11pt, twocolumn]{article}

\usepackage[utf8]{inputenc} % ~ Encodage
\usepackage[T1]{fontenc}    % ~ Encodage
\usepackage[left=1cm, right=1cm]{geometry} % ~ Mise en page et marges
\usepackage{amssymb} % ~ Pour écrire les maths
\usepackage{xspace}  % ~ Commandes à texte
\usepackage{varioref} % ~ Références croisées
\usepackage{enumitem} % ~ Listes
\usepackage{xcolor}   % ~ Couleurs fs
\usepackage{float}
\usepackage[load-configurations = abbreviations]{siunitx}
\usepackage{graphicx}
\usepackage[f]{esvect}
\usepackage[many]{tcolorbox}
\usepackage{euler}
\usepackage[nointegrals]{wasysym}
\usepackage{babel}


%______FONCTIONS______%
\newcommand{\ssi}{si et seulement si\xspace}		% ~ ssi
\newcommand{\inv}[1]{\dfrac{1}{#1}}
% --PARENTHESES--%
\newcommand{\po}{\left(}         % ~ (
\newcommand{\pf}{\right)}        % ~ )
\newcommand{\pof}[1]{\po #1 \pf} % ~ ( )
\newcommand{\co}{\left[}         % ~ [
\newcommand{\cf}{\right]}        % ~ ]
\newcommand{\cof}[1]{\co #1 \cf} % ~ [ ]
\newcommand{\chof}[1]{\left\langle #1 \right\rangle } % ~ < >
\newcommand{\interoo}[2]{\left]#1\,;#2\right[}   % ~ ]a,b[
\newcommand{\interof}[2]{\left]#1\,;#2\right]}   % ~ ]a,b]
\newcommand{\interfo}[2]{\left[#1\,;#2\right[}   % ~ [a,b[
\newcommand{\interff}[2]{\left[#1\,;#2\right]}   % ~ [a,b]
% --VECTEURS--%
\newcommand{\vect}[1]{\vv{#1}}
\newcommand{\ux}{\vect{u_x}}          % ~ Vecteur ux
\newcommand{\uy}{\vect{u_y}}          % ~ Vecteur uy
\newcommand{\uz}{\vect{u_z}}          % ~ Vecteur uz
\newcommand{\ur}{\vect{u_r}}          % ~ Vecteur ur
\newcommand{\uth}{\vect{u_\theta}}    % ~ Vecteur utheta
\newcommand{\uph}{\vect{u_\varphi}}   % ~ Vecteur uphi
\newcommand{\om}{\vect{OM}}           % ~ Vecteur position
\newcommand{\vvi}{\vect{v}}           % ~ Vecteur vitesse
\newcommand{\vvio}{\vect{v_0}}        % ~ Vecteur v0
\newcommand{\va}{\vect{a}}            % ~ Vecteur	accélération
\newcommand{\vp}{\vect{p}}            % ~ Vecteur quantité de mouvement
\newcommand{\fr}{\vect{F_r}}          % ~ Vecteur force de rappel
\newcommand{\vabla}{\vect{\nabla}}    % ~ nabla
\newcommand{\grad}{\vect{\mathrm{grad}}}  % ~ grad
\DeclareMathOperator{\diverg}{div}        % ~ div
\newcommand{\rota}{\vv{\mathrm{rot}}}    % ~ rot

\newtcolorbox{cadre}[2][]
{
  enhanced,
  attach boxed title to top left={yshift=-3.4mm, xshift = -2.3mm},
  adjusted title=#2,
  colback=white, colframe=black,
  colbacktitle=white, coltitle=black, fonttitle=\bfseries,
  breakable, sharp corners,
  boxed title style={colback=white, sharp corners, colframe=white},
  boxrule = 0.5mm, drop fuzzy shadow
}
\newcommand{\ooint}{\ocircle\hspace{-3.65mm}\int\hspace{-2mm}\int}

\title{Électromagnétisme}
\author{Martin \textsc{Andrieux} \\ Nathan \textsc{Maillet}}
\date{}

\begin{document}
\maketitle

\section{Analyse vectorielle}
\begin{cadre}{Circulation}
  \[\mathcal C = \int_A^B \vv E(M)\cdot\vv{d M}\]
  \[\mathcal C = \oint \vv E(M)\cdot\vv{d M}\]
\end{cadre}

\begin{cadre}{Flux}
  \[\Phi = \ooint_{(S)} \vv E(M)\cdot\vv{d S}\]
\end{cadre}

\begin{cadre}{Théorème de Gauss}
  Dans un champ électrique:
  \[\ooint_{(S)}\vv E(M)\cdot\vv{d S} = \dfrac{Q_{int}}{\varepsilon_{0}}\]
\end{cadre}

\begin{cadre}{Énergie}
  Pour \(n\) charges ponctuelles:
  \[U_{e} = \dfrac{1}{2}\sum_{i=1}^n q_{i}v_{i}\]
  Pour une distributuion continue:
  \[U_{e} = \inv{2}\iiint_{\text{espace}}\varepsilon_{0} E^{2}(M) d\tau\]
\end{cadre}

\begin{cadre}{Théorème d'Ostrogradski}
  \[\ooint_{(S)}\vv E \cdot \vv{d S} = \iiint_{(\mathcal V)} \diverg\po\vv E\pf d\tau\]
\end{cadre}

\begin{cadre}{Équation de Poisson}
  \[\Delta V + \dfrac{\rho}{\varepsilon_{0}} = 0\]
  En l'absence de charges:
  \[\Delta V = 0\]
\end{cadre}

\begin{cadre}{Théorème de Stokes}
  \[\oint \vv E\cdot\vv{d l} = \iint_{(S)}\rota \vv E \cdot \vv{d S}\]
\end{cadre}

\begin{cadre}{Équivalence}
  \[\vv E = -\grad V \iff \oint_{(C)}\vv E\cdot \vv{d l} = 0 \iff \rota\vv E = \vv0\]
\end{cadre}

\section{Dipôle électrostatique}
\begin{cadre}{Moment dipolaire}
  \[\vv p = q \vv{NP}\]
\end{cadre}

\begin{cadre}{Potentiel loin d'un dipôle}
  \[V(M) = \dfrac{p\cos(\theta)}{4\pi\varepsilon_{0}r^{2}}\]
  D'où:
  \[E_{r} = \dfrac{2p\cos(\theta)}{4\pi\varepsilon_{0}r^{3}}\text{ et } E_{\theta} = \dfrac{p\sin(\theta)}{4\pi\varepsilon_{0}r^{3}}\]
\end{cadre}

\begin{cadre}{Force exercée sur un dipôle}
  Force:
  \[\vv F = \pof{\vv p \cdot \grad}\vv E(M) = \po\vv p\cdot\vabla\pf\cdot\vv E\]
  Moment pour un champ variant peu:
  \[\vv\Gamma(M) = \vv p \wedge \vv E(M)\]
\end{cadre}

\begin{cadre}{Énergie Potentielle}
  \[E_{p} = -\vv p\cdot \vv E\]
\end{cadre}

\section{Conducteurs}
\begin{cadre}{Équilibre}
  \[\vv E = \vv 0 \text{ donc } V = 0\]
  Or, \(\diverg \vv E = 0\) donc \(\rho = 0\). La densité volumique de charge est nulle dans tout le volume du conducteur: la charge se localise uniquement sur la surface.
  D'où:
  \[\vv {E_{ext}} = \dfrac{\sigma}{\varepsilon_{0}}\vv n\]
\end{cadre}

\begin{figure}[h]
  \centering
  \begin{tikzpicture}
    \draw (0,0) -- node[near end, left]{\(V_{1}\)} ++(0,4) node[right]{\(+\sigma\)};
    \draw (2,0) -- node[near end, right]{\(V_{2}\)} ++(0,4) node[right]{\(-\sigma\)};
    \draw[<->, >=latex] (0.2,0) -- node[midway, below]{\(e\)} ++(1.6,0);
    \draw[->, >=latex] (-1,1) node[above left]{\(S\)} to[bend right=20] (-0.1, 0.5);
  \end{tikzpicture}
\end{figure}
\begin{cadre}{Capacité}
  La capacité \(C\) est telle que:
  \[Q = C\cdot\pof{V1 - V2}\]
  \[C = \dfrac{\varepsilon_{0}S}{e}\]
\end{cadre}

\begin{cadre}{Densité volumique de courant}
  \[\vv \jmath = nq\vv v = \rho_{m}\vv v\]
  Dans les métaux:
  \[\vv \jmath = -ne\vv v\]
  \[i = \iint_{(S)}\vv\jmath\cdot\vv{dS}\]
\end{cadre}

\begin{cadre}{Conservation de la charge}
  \[\diverg \vv\jmath + \dfrac{\partial\rho}{\partial t} = 0\]
  La démonstration de ce résultat est à connaître, en voici les grandes lignes:
  \[-\dfrac{dq}{dt} = \ooint_{(S)}\vv \jmath \cdot\vv{dS}\]
  \[q(t) = \iiint_{(V)} \rho(M, t)d\tau\]
  Ensuite, dériver \(q\), permuter la dérivation et la sommation, appliquer Ostrogradski. L'égalité des intégrales entraîne l'égalité des grandeurs sommées.
\end{cadre}

\begin{cadre}{Loi d'Ohm locale}
  \[\vv \jmath = \sigma\vv E\]
  Avec \(\sigma\) la \emph{conductivité} du milieu, aussi notée \(\gamma\).
\end{cadre}

\begin{cadre}{Loi de Joule locale}
  \[\mathcal P = \vv\jmath\cdot\vv E\]
\end{cadre}

\begin{cadre}{Résistance d'un conducteur cylindrique}
  \[R = \dfrac{l}{\sigma S}\]
  Cette expression se retrouve rapidement avec un raisonnement purement intuitif du type \og Plus le fil est long plus la résistance est grande\fg{}.
\end{cadre}

\section{Magnétostatique}
\begin{cadre}{Force de Lorentz}
  \[ \vv f = q\pof{\vv E + \vv v \wedge \vv B}\]
\end{cadre}

\begin{cadre}{\(\vv B\) est à flux conservatif}
  \[\ooint_{(S)}\vv B \cdot\vv{dS} = 0\]
  Or d'après le théorème d'Ostrogradski,
  \[\ooint_{(S)}\vv B \cdot\vv{dS} = \iiint_{(V)}\diverg \vv B d\tau = 0\]
  D'où:
  \[\diverg \vv B = 0\]
\end{cadre}

\begin{cadre}{Théorème d'Ampère}
  \[\oint_{(C)} \vv B(M)\cdot \vv{dl} = \mu_{0}I_{\text{int}}\]
  Pour le théorème d'Ampère local, on a :
  \[\rota{\vv B}=\mu_{0} \vv \jmath\]
\end{cadre}

\begin{cadre}{Discontinuité}
  La discontinuité d'un champ de part et d'autre d'une surface est en \(\mu_{0}\vv{\jmath_{s}}\). Cela est à raprocher du \(\frac{\sigma}{\varepsilon_{0}}\) en électrostatique.
\end{cadre}

\begin{cadre}{Champ dans un solénoïde}
  \[\vv{B_{\text{int}}}=\mu_{0}nI\cdot\uz\]
\end{cadre}

\section{Dipôle magnétique}
\begin{cadre}{Moment dipolaire magnétique}
  \[\vv{\mathcal M} = IS\vv n\]
\end{cadre}

\begin{cadre}{Champ créé par un dipôle}
  \[\vv B(M) = \dfrac{\mu_0\mathcal M}{4\pi r^3}\pof{2\cos(\theta)\ur + \sin(\theta)\uth}\]
  Cette expression est totalement analogue à celle obtenue pour le dipôle électrostatique, il suffit en effet de remplacer \(p\) par \(\mathcal M\) et \(\frac{1}{\varepsilon_{0}}\) par \(\mu_{0}\).
\end{cadre}

\begin{cadre}{Moment des forces de Laplace}
\[\vv\Gamma(M) = \vv{\mathcal M}\wedge \vv B(M)\]
\end{cadre}

\begin{cadre}{Énergie Potentielle}
  \[E_{p} = -\vv{\mathcal M}\cdot \vv B\]
\end{cadre}

\begin{cadre}{Flux du champ magnétique}
  \[\Phi = \iint_{(S)}\vv B\cdot\vv{dS}\]
  \[\Phi = L_{1}i_{1} (+ M_{12} i_{2})\]
  Avec \(L\) les coefficients d'\emph{auto-induction} et \(M\) les coefficients d'\emph{induction mutelle}.
  \(L\) est bien sûr positif.
  On peut montrer que \(M_{ij} = M_{ji}\).
\end{cadre}

\begin{cadre}{Inductance propre}
  Lorsque que l'on connait \(\vv B\), il est possible de calculer le flux du champ, et donc l'inductance propre du circuit avec \(\Phi =\Lambda i\).
\end{cadre}

\begin{cadre}{Énergie magnétique}
  \[U_{m} = \inv{2}L_{1}i_{1}^{2}+\inv{2}L_{2}i_{2}^{2}+\inv{2}M_{12}i_{1}i_{2}\]
  \[U_{m} = \iiint_{\text{espace}}\dfrac{B^{2}}{2\mu_{0}}d\tau\]
\end{cadre}

\section{Équations de Maxwell}
\begin{cadre}{Égalité}
  \[\varepsilon_{0}\mu_{0}c^{2}=1\]
\end{cadre}

\begin{cadre}{Maxwell-Gauss}
    \[\diverg \vv E = \dfrac{\rho}{\varepsilon_{0}}\]
    \tcblower
    Cette équation est la forme locale du théorème de Gauss.
\end{cadre}

\begin{cadre}{Flux magnétique}
    \[\diverg \vv B = 0\]
    \tcblower
    Cette équation signifie que \(\vv B\) est à flux conservatif.
\end{cadre}

\begin{cadre}{Maxwell-Faraday}
  \[\rota \vv E = -\dfrac{\partial \vv B}{\partial t}\]
  \tcblower
  L'équation de Maxwell-Faraday traduit, au niveau local, la \emph{loi de Faraday de l'induction électromagnétique}.
\end{cadre}

\begin{cadre}{Maxwell-Ampère}
    \[\rota \vv B = \mu_{0}\vv\jmath+\dfrac{1}{c^{2}}\dfrac{\partial\vv E}{\partial t}\]
    \tcblower
    Ce résultat constitue le \emph{théorème d'Ampère généralisé}.
\end{cadre}

\begin{cadre}{Vecteur de Poynting}
  \[\vv\Pi = \dfrac{\vv E\wedge\vv B}{\mu_{0}}\]
  Le vecteur de Poynting correspond à un flux surfacique d'énergie, ou encore à une puissance surfacique. Il s'exprime en \(\si{\watt\per\meter\squared}\).
  Les calculs de puissances et d'énergies se font impérativement en réel.
\end{cadre}

\begin{cadre}{Équation de Poynting}
  \[\dfrac{\partial u_{\text{em}}}{\partial t} + \diverg \vv\Pi + \vv\jmath\cdot\vv E = 0\]
  Cette équation traduit localement la conservation de l'énergie.
\end{cadre}

\begin{cadre}{Énergie d'un photon}
  \[E = h\nu\]
  \[E = pc\]
\end{cadre}

\begin{cadre}{Quantité de mouvement volumique}
  \[\dfrac{d\vv p}{d\tau} = \dfrac{\vv \Pi}{c^{2}} = \dfrac{\vv E \wedge\vv B}{\mu_{0}\cdot c^{2}}=\varepsilon_{0}\vv E\wedge\vv B\]
\end{cadre}

\section{Induction}
\begin{cadre}{Loi de Faraday}
  \[e = -\dfrac{d\phi}{dt}\]
\end{cadre}

\begin{cadre}{Loi de Lenz}
  \og La f.é.m. induite tend par ses effets à s'opposer à la cause qui lui a donné naissance.\fg{}
\end{cadre}

\begin{cadre}{Puissances pour l'induction de Lorentz}
  En notant:
  \begin{itemize}
    \item \(\mathcal{P}_{l}\) la puissance des forces de Laplace
    \item \(\mathcal{P}_{e}\) la puissance de la f.é.m. induite
  \end{itemize}
  \[\mathcal{P}_{l}+\mathcal{P}_{e} = 0\]
\end{cadre}

\begin{cadre}{Effet de peau}
  Les courants sont presque entièrement concentrés dans une couche dont l'épaisseur \(\delta\) est de l'ordre de:
  \[\delta = \sqrt{\dfrac{2}{\mu_{0}\sigma\omega}}\]
\end{cadre}

\section{Ondes électromagnétique}
\begin{cadre}{Équations de propagation}
  \[\vv\Delta \vv E - \dfrac{1}{c^{2}}\dfrac{\partial^{2}\vv E}{\partial t^{2}} = \vv 0\]
  \[\vv\Delta \vv B - \dfrac{1}{c^{2}}\dfrac{\partial^{2}\vv B}{\partial t^{2}} = \vv 0\]
\end{cadre}

\begin{cadre}{Vecteur d'onde}
  \(\vv u\) est la direction du mouvement.
  \[\vv k = \dfrac{\mathrm{n}_{\lambda}\omega}{c}\vv u\]
\end{cadre}

\begin{cadre}{Opérateurs d'analyse vectorielle}
  Ces règles ne valent que pour des OPPM.
  \begin{align*}
    \dfrac{\partial\vv E}{\partial t} = j\omega\vv E && \diverg\vv E = -j\vv k\cdot\vv E \\
    \rota\vv E = -j\vv k\wedge\vv E && \vv\Delta\vv E = -k^{2}\vv E
  \end{align*}
\end{cadre}

\begin{cadre}{Relation de structure}
  \[\vv B=\dfrac{\vv k\wedge\vv E}{\omega}\]
  Il faut savoir retrouver cette relation, avec la loi de Maxwell-Faraday:
  \[\rota\vv E = -\dfrac{\partial\vv B}{\partial t} = -j\vv k\wedge\vv E = -j\omega\vv B\]
\end{cadre}

\begin{cadre}{Loi de Malus}
  Si l'angle entre les directions d'un analyseur et d'un polariseur est $\alpha$, l'intensité de l'onde après l'analyseur est $I_{0}\cos^{2}(\alpha)$.
\end{cadre}

\section{Dispersion}
\begin{cadre}{Vitesse de phase}
  \[v_{\varphi}=\dfrac{\omega}{k}\]
  Si $v_{\varphi}$ est indépendante de $\omega$, le milieu est \emph{non-dispersif}. Dans le cas contraire, le milieu est \emph{dispersif}.
  Le vide est le seul milieu rigoureusement non dispersif.
\end{cadre}

\begin{cadre}{Relation spectre-signal}
  \[\Delta\omega\cdot\Delta t \sim 2\pi\]
  \[\Delta\nu\cdot\Delta t \sim 1\]
\end{cadre}

\begin{cadre}{Vitesse de groupe}
  \[v_{g}=\pof{\dfrac{d\omega}{dk}}_{\omega_{0}}\]
\end{cadre}

\begin{cadre}{Relation de Klein-Gordon}
  C'est une relation de dispertion de la forme:
  \[k^{2}c^{2} = \omega^{2}-\omega_{c}^{2}\]
  Rencontrée notamment dans l'étude des plasmas.
\end{cadre}

\section{Rayonnement dipolaire électrique}
\begin{cadre}{Dipôle oscillant}
  Nous appellerons dipôle oscillant une distribution telle que:
  \[\vv p = p_{0}\cos(\omega t)\uz\]
\end{cadre}

\begin{cadre}{Hypothèses}
  \begin{itemize}[label=$\bullet$]
    \item $a\ll r$ permet d'utiliser les formules relatives aux dipôles,
    \item $a\ll\lambda$ peut être appelée \emph{approximation non relativiste},
    \item $\lambda\ll r$ n'a pas de signification physique, mais simplifie les expressions.
  \end{itemize}
\end{cadre}

\end{document}
