\documentclass[french, a4paper, 11pt, twocolumn]{article}

\usepackage[utf8]{inputenc} % ~ Encodage
\usepackage[T1]{fontenc}    % ~ Encodage
\usepackage[left=1cm, right=1cm]{geometry} % ~ Mise en page et marges
\usepackage{amssymb} % ~ Pour écrire les maths
\usepackage{xspace}  % ~ Commandes à texte
\usepackage{varioref} % ~ Références croisées
\usepackage{enumitem} % ~ Listes
\usepackage{xcolor}   % ~ Couleurs fs
\usepackage{graphicx}
\usepackage{tikz}
\usepackage[version=4]{mhchem}
\usepackage{float}
\usepackage[load-configurations = abbreviations, output-decimal-marker={,}]{siunitx}
\usepackage[many]{tcolorbox}
\usepackage{euler}
\usepackage{calrsfs}
\usepackage{babel}


%______FONCTIONS______%
\newcommand{\ssi}{si et seulement si\xspace}		% ~ ssi
\newcommand{\inv}[1]{\dfrac{1}{#1}}
\newcommand{\RNum}[1]{\uppercase\expandafter{\romannumeral #1\relax}}
% --PARENTHESES--%
\newcommand{\po}{\left(}         % ~ (
\newcommand{\pf}{\right)}        % ~ )
\newcommand{\pof}[1]{\po #1 \pf} % ~ ( )
\newcommand{\co}{\left[}         % ~ [
\newcommand{\cf}{\right]}        % ~ ]
\newcommand{\cof}[1]{\co #1 \cf} % ~ [ ]
\newcommand{\chof}[1]{\left\langle #1 \right\rangle } % ~ < >
\newcommand{\interoo}[2]{\left]#1\,;#2\right[}   % ~ ]a,b[
\newcommand{\interof}[2]{\left]#1\,;#2\right]}   % ~ ]a,b]
\newcommand{\interfo}[2]{\left[#1\,;#2\right[}   % ~ [a,b[
\newcommand{\interff}[2]{\left[#1\,;#2\right]}   % ~ [a,b]

\newtcolorbox{cadre}[2][]
{
  enhanced,
  attach boxed title to top left={yshift=-3.4mm, xshift = -2.3mm},
  adjusted title=#2,
  colback=white, colframe=black,
  colbacktitle=white, coltitle=black, fonttitle=\bfseries,
  breakable, sharp corners,
  boxed title style={colback=white, sharp corners, colframe=white},
  boxrule = 0.5mm, drop fuzzy shadow
}


\title{Oxydorécution}
\author{Nathan \textsc{Maillet}}
\date{}

\begin{document}
    \maketitle

    \section{Oxydoréduction}

    \begin{cadre}{Nombre d'oxydation}
        Pour tous les édifices comprenant de l'hydrogène, de l'oxygène et un unique autre atome, le nombre d'oxydation
        est donné par la formule :
        
        Somme des nombres d'oxydations = Charge totale de l'édifice considéré

        \tcblower
        Exemple :
        \begin{itemize}
            \item Dans \(\ce{MnO4-}\), l'oxygène étant au degré d'oxydation -\RNum{2},
            il en découle que l'on a \(\ce{Mn^{+\RNum{8}}}\) 
        \end{itemize}
    \end{cadre}

    \begin{cadre}{Formule de Nernst}
        Pour un couple avec la demi-équation :
        \[\alpha \mathrm{ox} + \mathrm{n} \mathrm{e^-} = \beta \mathrm{red},\] on a :
        \begin{align*}
            E &=E^0(\mathrm{T})+\frac{\mathrm{RT}}{\mathrm{n}\mathcal{F}}\ln\pof{\frac{\mathrm{a_{ox}^{\alpha}}}{\mathrm{a_{red}^{\beta}}}} \\
            E &=E^0(\mathrm{T})+\frac{0,06}{\mathrm{n}}\ln\pof{\frac{\mathrm{a_{ox}^{\alpha}}}{\mathrm{a_{red}^{\beta}}}}
        \end{align*}
    \end{cadre}

    \begin{cadre}{Thermodynamique de l'oxydoréduction}
        Avec la même demi-équation que ci-dessus, on a :
        \[\Delta_rG^0=-\mathrm{n}\mathcal{F}(\mathrm{E^0_1-E^0_2})\]
        et
        \[\Delta_rG^0=-\mathrm{n_i}\mathcal{F}\mathrm{E_i}\]

        \tcblower
        Très utile dans les calculs de \(\mathrm{E^0}\) pour trouver facilement les potentiels standard de couples inconnus
        à partir de couples connus. En effet, il faut dans ce cas appliquer la loi de Hess et ne \emph{surtout pas} écrire bêtement
        que \(\mathrm{E^0_3=E^0_1+E^0_2}\)
    \end{cadre}

    \begin{cadre}{L'électrode standard à hydrogène}
        Par convention, la référence pour les potentiels est :
        \[\mathrm{E^0(H^+/H_2(g))=\SI{0}{\volt}}\]
    \end{cadre}

    \section{Cinétique et corrosion}
    \begin{cadre}{Convention}
        On compte positivement le courant dans le sens : électrode \(\rightarrow\) solution, donc si \(I>0\) il y a oxydation,
        sinon il y a réduction.
    \end{cadre}

    \begin{cadre}{Intensité et vitesse de réaction}
        Soit \(\mathscr{V}\) la vitesse de réaction.
        On a : \(I=\frac{\mathrm{d}q}{\mathrm{d}t}=n\mathcal{F}\mathscr{V}\)
    \end{cadre}

    \begin{cadre}{Types d'électrodes}
        Il y a 3 types d'électrodes :
        \begin{itemize}
            \item l'électrode de travail, qui mesure l'intensité
            \item l'électrode auxilliaire (ou contre-électrode), qui sert a fermer le circuit
            \item l'électrode de référence, sans circulation de courant
        \end{itemize}
    \end{cadre}

    \begin{cadre}{Intensité-potentiel}
        \begin{itemize}[label=\(\bullet\)]
            \item Si \(I=0\) le potentiel est égal au potentiel rédox donné par la formule de Nersnt à l'équilibre. Sinon il n'y a plus équilibre thermodynamique
            
            \item La différence entre le potentiel de l'électrode et redox est appelée surtension
            \item On parle de surtension anodique lorsque cette différence est positive, cathodique si elle est négative
            \item Les surtensions à vide (ou de seuil) sont telles que l'intensité commence à ne plus être négligeable
            
            \item Lorsque la tension devient importante, il arrive que \(I\) cesse d'augmenter, à cause d'un palier de diffusion
            \item Il n'y a pas de palier de diffusion pour les ions ou molécules du solvant ni pour l'oxydation du métal de l'électrode
        \end{itemize}
    \end{cadre}

    \begin{cadre}{Murs de solvant}
        En solution aquese, les réactions électrochimiques autres que celles de l'eau ne peuvent avoir lieu que dans un
        domaine de potentiel limité par les deux murs du solvant (asymptotes verticales).
    \end{cadre}

    \begin{cadre}{Vagues successives}
        Lorsque plusieurs couples peuvent réagir au niveau d'une électrode, l'intensité totale correspond à la somme des 
        intensités associées à chaque couple. 
        
        Ces augmentations successives (parfois simultanées quand elles coexistes)
        de l'intensée lorsqu'une nouvelle réaction commence sont qualifiées de vagues successives.
    \end{cadre}

    \begin{cadre}{Potentiel mixte}
        Pour une réaction thermodynamiquement et cinétiquement possible,
        l'intensité commune à chaque couple est proportionnelle à la vitesse de 
        réaction rédox. Le potentiel \(\mathrm{E_m}\) associé est appelé potentiel mixte.
        
        \begin{tikzpicture}[scale=1.5]
          \draw[->, >=stealth] (0,-1.6) -- (0, 1.6) node[left]{$\mathrm{I}$};
          \draw[->, >=stealth] (-0.3,0) -- (3.5, 0) node[below right]{$\mathrm{V}$};
          \draw[dashed] (0, 1) node[left]{\(\mathrm{I}\)}-- (1.3, 1);
          \draw[dashed] (0, -1) node[left]{\(\mathrm{-I}\)}-- (1.3, -1);
          \draw[dashed] (1.3,1) -- (1.3,-1);
          \draw [domain=0:3.5,scale=0.4] plot(\x,{exp(\x-2.3)-exp(-2.3)});
          \draw [domain=2.3:5,scale=0.5] plot(\x,{ln(\x-2.22)-1.0225});
          \draw (1.3,0) node[thick, above right]{\(\mathrm{E_m}\)};
        \end{tikzpicture}

        \tcblower
        \begin{itemize}
            \item \(\mathrm{E_m}\) est différent de la moyenne
            \item Le point de fonctionnement est le potentiel mixte correspondant à l'intensité de corrosion,
                appelé potentiel de corrosion.
        \end{itemize}
        
    \end{cadre}

    \begin{cadre}{Domaines et diagramme}
        Sur un diagrame potentiel-pH, le domaine de prédominance du métal est appelé le domaine d'\emph{immunité} du métal.
        Pour les ions ont parle de \emph{corrosion} et de \emph{passivité} pour les autres solides.
    \end{cadre}

    \begin{cadre}{Passivation}
        Il y a \emph{passivation} si un solide adhère au métal et y forme une couche imperméable qui protège le métal d'une oxydation. 
    \end{cadre}

    \section{Piles et électrolyse}

    \begin{cadre}{Tension d'une pile}
        La tension \(U_{CA}\) aux bornes d'une pile dépend d'un terme
        thermodynamique, un cinétique et un ohmique et vérifie : 
            \[\mathrm{U_{CA}=E_1-E_2}-(\eta_a-\eta_c)-R_{int}I\]
        
        \tcblower
        \begin{itemize}
            \item Dans le sens direct : \(\mathrm{U_{CA}\leq E_2-E_2}\)
            \item Les \emph{surtensions} sont liées à la diffusion des ions dans les électrolytes,
                aux transfets d'électrons au niveau des électrodes \dots
            \item La \emph{chute ohmique} est due aux électrodes, mais surtout au milieu électrolytiques (notamment le pont salin).
        \end{itemize}
    \end{cadre}

    \begin{cadre}{Pile idéale}
        Une bonne pile doit vérifier :
        \begin{itemize}[label=\(\bullet\)]
            \item grande différence entre les potentiels de Nernst
            \item faibles surtensions
            \item intensité de court-circuit grande
            \item résistance interne faible
        \end{itemize}
    \end{cadre}

    \begin{cadre}{Électrolyse}
        Le schéma d'une électrolyse est différent de celui d'une pile :
            \begin{itemize}[label=\(\bullet\)]
                \item Le pont salin est enlever
                \item La charge de la pile est remplacée par un générateur pour forcer la réaction
                \item Il n'y a qu'un bac pour les 2 réactifs
            \end{itemize}

        \tcblower
        Pour une électrolyse on utilise la règle du gamma "inversée" car l'électrolyse réalise
        la réaction qui ne devrait pas avoir lieu.
    \end{cadre}

    \begin{cadre}{Tension d'une électrolyse}
        La tension pour une électrolyse vérifie :
            \[\mathrm{U_{CA}=E_1-E_2}+(\eta_a-\eta_c)+R_{int}I\]
        \tcblower
            \(\mathrm{U_{CA}>E_1-E_2}\)
    \end{cadre}
\end{document}
