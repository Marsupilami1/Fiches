\documentclass[french, a4paper, 11pt, twocolumn]{article}

\usepackage[utf8]{inputenc} % ~ Encodage
\usepackage[T1]{fontenc}    % ~ Encodage
\usepackage[left=1cm, right=1cm]{geometry} % ~ Mise en page et marges
\usepackage{amssymb} % ~ Pour écrire les maths
\usepackage{xspace}  % ~ Commandes à texte
\usepackage{varioref} % ~ Références croisées
\usepackage{enumitem} % ~ Listes
\usepackage{xcolor}   % ~ Couleurs fs
\usepackage[version=4]{mhchem}
\usepackage{float}
\usepackage[load-configurations = abbreviations, output-decimal-marker={,}]{siunitx}
\usepackage{graphicx}
\usepackage[many]{tcolorbox}
\usepackage{euler}
\usepackage{calrsfs}
\usepackage[french]{babel}


%______FONCTIONS______%
\newcommand{\ssi}{si et seulement si\xspace}		% ~ ssi
\newcommand{\inv}[1]{\dfrac{1}{#1}}
\newcommand{\RNum}[1]{\uppercase\expandafter{\romannumeral #1\relax}}
% --PARENTHESES--%
\newcommand{\po}{\left(}         % ~ (
\newcommand{\pf}{\right)}        % ~ )
\newcommand{\pof}[1]{\po #1 \pf} % ~ ( )
\newcommand{\co}{\left[}         % ~ [
\newcommand{\cf}{\right]}        % ~ ]
\newcommand{\cof}[1]{\co #1 \cf} % ~ [ ]
\newcommand{\chof}[1]{\left\langle #1 \right\rangle } % ~ < >
\newcommand{\interoo}[2]{\left]#1\,;#2\right[}   % ~ ]a,b[
\newcommand{\interof}[2]{\left]#1\,;#2\right]}   % ~ ]a,b]
\newcommand{\interfo}[2]{\left[#1\,;#2\right[}   % ~ [a,b[
\newcommand{\interff}[2]{\left[#1\,;#2\right]}   % ~ [a,b]

\newtcolorbox{cadre}[2][]
{
  enhanced,
  attach boxed title to top left={yshift=-3.4mm, xshift = -2.3mm},
  adjusted title=#2,
  colback=white, colframe=black,
  colbacktitle=white, coltitle=black, fonttitle=\bfseries,
  breakable, sharp corners,
  boxed title style={colback=white, sharp corners, colframe=white},
  boxrule = 0.5mm, drop fuzzy shadow
}


\title{Oxydorécution}
\author{Nathan \textsc{Maillet}}
\date{}

\begin{document}
    \maketitle

    \begin{cadre}{Nombre d'oxydation}
        Pour tous les édifices comprenant de l'hydrogène, de l'oxygène et un unique autre atome, le nombre d'oxydation
        est donné par la formule :
        
        Somme des nombres d'oxydations = Charge totale de l'édifice considéré

        \tcblower
        Exemple :
        \begin{itemize}
            \item Dans \(\ce{MnO_{4-}}\), l'oxygène étant au degré d'oxydation -\RNum{2},
            il en découle que l'on a \(\ce{Mn^{+\RNum{8}}}\) 
        \end{itemize}
    \end{cadre}

    \begin{cadre}{Formule de Nernst}
        Pour un couple avec la demi-équation :
        \[\alpha \mathrm{ox} + \mathrm{n_1} \mathrm{e^-} = \beta \mathrm{red},\] on a :
        \begin{align*}
            E &=E^0(\mathrm{T})+\frac{\mathrm{RT}}{\mathrm{n}\mathcal{F}}\ln\pof{\frac{\mathrm{a_{ox}^{\alpha}}}{\mathrm{a_{red}^{\beta}}}} \\
            E &=E^0(\mathrm{T})+\frac{0,06}{\mathrm{n}}\ln\pof{\frac{\mathrm{a_{ox}^{\alpha}}}{\mathrm{a_{red}^{\beta}}}}
        \end{align*}
    \end{cadre}

    \begin{cadre}{Thermodynamique de l'oxydorécution}
        Avec la même demi-équation que ci-dessus, on a :
        \[\Delta_rG^0=-\mathrm{n}\mathcal{F}(\mathrm{E^0_1-E^0_2})\]
        et
        \[\Delta_rG^0=-\mathrm{n_i}\mathcal{F}\mathrm{E_i}\]

        \tcblower
        Très utile dans les calculs de \(\mathrm{E^0}\) pour trouver facilement les potentiels standard de couples inconnus
        à partir de couples connus. En effet, il faut dans ce cas appliquer la loi de Hess et ne \textbf{surtout pas} écrire bêtement
        que \(\mathrm{E^0_3=E^0_1+E^0_2}\)
    \end{cadre}

    \begin{cadre}{L'électrode standard à hydrogène}
        Par convention, la référence pour les potentiels est :
        \[\mathrm{E^0(H^+/H_2(g))=\SI{0}{\volt}}\]
    \end{cadre}
\end{document}