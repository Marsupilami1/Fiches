\documentclass[french, a4paper, 11pt, twocolumn]{article}

\usepackage[utf8]{inputenc} % ~ Encodage
\usepackage[T1]{fontenc}    % ~ Encodage
\usepackage[left=1cm, right=1cm, bottom=3cm]{geometry} % ~ Mise en page et marges
\usepackage{amssymb} % ~ Pour écrire les maths
\usepackage{xspace}  % ~ Commandes à texte
\usepackage{varioref} % ~ Références croisées
\usepackage{enumitem} % ~ Listes
\usepackage{xcolor}   % ~ Couleurs fs
\usepackage{tabularx}
\usepackage{float}
\usepackage{tikz}
\usepackage[straightvoltages]{circuitikz}
\usepackage[load-configurations = abbreviations]{siunitx}
\usepackage{graphicx}
\usepackage[f]{esvect}
\usepackage[many]{tcolorbox}
\usepackage{euler}
\usepackage[nointegrals]{wasysym}
\usepackage{array}
\usepackage{babel}


\sisetup{
  locale=FR,
  detect-all,
}

% ______FONCTIONS______%
\newcommand{\ssi}{si et seulement si\xspace}		% ~ ssi
% --ENSEMBLES--%
\newcommand{\N}{\mathbb{N}}   % ~ Entiers naturels
\newcommand{\Z}{\mathbb{Z}}   % ~ Entiers relatifs
\newcommand{\D}{\mathbb{D}}   % ~ Decimaux
\newcommand{\Q}{\mathbb{Q}}   % ~ Rationnels
\newcommand{\R}{\mathbb{R}}   % ~ Réels
\newcommand{\C}{\mathbb{C}}   % ~ Complexes
% --TRIGO--%
\let\cosh\relax
\DeclareMathOperator{\cosh}{ch}       % ~ cosinus hyperbolique
\DeclareMathOperator{\sh}{sh}         % ~ sinus hyperbolique
\let\tanh\relax
\DeclareMathOperator{\tanh}{th}       % ~ tangente hyperbolique
\DeclareMathOperator{\argch}{Argch}   % ~ Argument cosinus hyperbolique
\DeclareMathOperator{\argsh}{Argsh}   % ~ Argument sinus hyperbolique
\DeclareMathOperator{\argth}{Argth}   % ~ Argument tangente hyperbolique
\DeclareMathOperator{\cotan}{cotan}   % ~ cotangente
% --PARENTHESES--%
\newcommand{\po}{\left(}         % ~ (
  \newcommand{\pf}{\right)}        % ~ )
\newcommand{\pof}[1]{\po #1 \pf} % ~ ( )
\newcommand{\co}{\left[}         % ~ [
  \newcommand{\cf}{\right]}        % ~ ]
\newcommand{\cof}[1]{\co #1 \cf} % ~ [ ]
\newcommand{\chof}[1]{\left\langle #1 \right\rangle } % ~ < >
\newcommand{\interoo}[2]{\left]#1\,;#2\right[}   % ~ ]a,b[
\newcommand{\interof}[2]{\left]#1\,;#2\right]}   % ~ ]a,b]
\newcommand{\interfo}[2]{\left[#1\,;#2\right[}   % ~ [a,b[
\newcommand{\interff}[2]{\left[#1\,;#2\right]}   % ~ [a,b]
% --VECTEURS--%
\newcommand{\ux}{\vv{u_x}}          % ~ Vecteur ux
\newcommand{\uy}{\vv{u_y}}          % ~ Vecteur uy
\newcommand{\uz}{\vv{u_z}}          % ~ Vecteur uz
\newcommand{\ur}{\vv{u_r}}          % ~ Vecteur ur
\newcommand{\uth}{\vv{u_\theta}}    % ~ Vecteur utheta
\newcommand{\uph}{\vv{u_\varphi}}   % ~ Vecteur uphi
\newcommand{\om}{\vv{OM}}           % ~ Vecteur position
\newcommand{\vvi}{\vv{v}}           % ~ Vecteur vitesse
\newcommand{\vvio}{\vv{v_0}}        % ~ Vecteur v0
\newcommand{\va}{\vv{a}}            % ~ Vecteur	accélération
\newcommand{\vp}{\vv{p}}            % ~ Vecteur quantité de mouvement
\newcommand{\fr}{\vv{F_r}}          % ~ Vecteur force de rappel
\newcommand{\vabla}{\vv{\nabla}}    % ~ nabla
\newcommand{\grad}{\vv{\mathrm{grad}}}  % ~ grad
\DeclareMathOperator{\diverg}{div}        % ~ grad
\newcommand{\rot}{\vv{\mathrm{rot}}}    % ~ grad

\newtcolorbox{cadre}[2][]
{
  enhanced,
  attach boxed title to top left={yshift=-3.4mm, xshift = -2.3mm},
  adjusted title=#2,
  colback=white, colframe=black,
  colbacktitle=white, coltitle=black, fonttitle=\bfseries,
  breakable, sharp corners,
  boxed title style={colback=white, sharp corners, colframe=white},
  boxrule = 0.5mm, drop fuzzy shadow
}
\newcommand{\ooint}{\ocircle\hspace{-3.65mm}\int\hspace{-2mm}\int}

\title{Physique quantique}
\author{Martin \textsc{Andrieux}}
\date{}

\begin{document}
\maketitle

\begin{cadre}{Relation de Planck-Einstein}
  \[E = h\nu = \hbar\omega = \dfrac{hc}{\lambda}\]
\end{cadre}

\begin{cadre}{Quantité de mouvement}
  \[E = pc\]
  \[p = \dfrac{E}{c} = \dfrac{h\nu}{c} = \dfrac{h}{\lambda}\]
  D'où la relation de de Broglie:
  \[\lambda = \dfrac{h}{p}\]
\end{cadre}

\begin{cadre}{Domaine de la mécanique quantique}
  Pour un milieu de dimension caractéristique \(l\), si \(\lambda = \frac{h}{p} \ll l\), la mécanique classique peut suffire.

  Sinon, la mécanique quantique est nécessaire.
\end{cadre}

\begin{cadre}{Fonction d'onde}
  Une particule est caractérisée par sa fonction d'onde \(\Psi(x, y, z, t)\).
  Lorsque \(\Psi\) est de carré sommable, il est possible de la normaliser pour avoir
  \[\iiint_{\text{espace}}\lvert\Psi\rvert^{2}d\tau = 1\]
\end{cadre}

\begin{cadre}{Équation de Schrödinger}
  \[-\dfrac{\hbar^{2}}{2m}\Delta\Psi + V\Psi = i\hbar\dfrac{\partial \Psi}{\partial t}\]
  \[\widehat H\Psi = i\hbar\dfrac{\partial \Psi}{\partial t}\]
\end{cadre}
\begin{cadre}{États stationnaires}
  \[\Psi(x, t) = \psi(x)f(t)\]
  Dans le cas d'une particule à un seul degré de liberté \((x)\) et dont l'énergie potentielle \(V(x)\) ne dépend pas du temps, l'équation de Schrödinger indépendante du temps est simplement:
  \[-\dfrac{\hbar^{2}}{2m}\dfrac{\Delta\psi(x)}{\psi(x)} + V(x) = i\hbar\dfrac{f'(t)}{f(t)} = E\]
  \(E\) est l'énergie de la fonction d'onde.
\end{cadre}

\begin{cadre}{Onde de de Broglie}
  Dans le cas d'une particule libre (ne subissant aucune force), l'énergie potentielle étant choisie nulle, on peut écrire:
  \[-\dfrac{\hbar^{2}}{2m}\dfrac{d^{2}\psi}{dx^{2}}(x)=E\psi(x)\]
  On pose \(k=\frac{\sqrt{2me}}{\hbar}\), les solutions sont alors de la forme:
  \[\psi(x) = Ae^{-ikx} + Be^{-ikx}\]
  \(f\) est de la forme suivante:
  \[f(t) = Ce^{-i\frac{Et}{\hbar}}=Ce^{-i\omega t}\]
  D'où une solution générale de la forme:
  \[\Psi(x,t) = \alpha e^{i(kx-\omega t)} + \beta e^{i(-kx -\omega t)}\]
  \[\Psi(x,t) = \alpha e^{-i(\omega t - kx)} + \beta e^{-i(\omega t + kx)}\]
  \tcblower
  \begin{align*}
    k = \dfrac{2\pi}{\lambda} && \vv p = \hbar\vv k \\
    E = \dfrac{k^{2}\hbar^{2}}{2m} = \hbar\omega && \omega = \dfrac{k^{2}\hbar}{2m}
  \end{align*}
\end{cadre}

\begin{cadre}{Principe d'incertitude de Heisenberg}
  \[\Delta E \cdot \Delta t    \geqslant \dfrac{\hbar}{2}\]
  \[\Delta x \cdot \Delta p_{x} \geqslant \dfrac{\hbar}{2}\]
\end{cadre}

\begin{cadre}{Densité de courant de probabilité}
  Pour une particule en mouvement à la vitesse \(\vv v\):
  \[\vv J = \lvert\Psi\rvert^2\cdot \vv v = \lvert\Psi\rvert^2\dfrac{\hbar\vv k}{m}\]
  \tcblower
  \[\diverg \vv J + \dfrac{\partial\lvert\Psi\rvert^2}{\partial t} = 0\]
\end{cadre}

\end{document}
