\documentclass[french, a4paper, 11pt]{article}

\usepackage[utf8]{inputenc} % ~ Encodage
\usepackage[T1]{fontenc}    % ~ Encodage
\usepackage[left=1cm, right=1cm, bottom=3cm]{geometry} % ~ Mise en page et marges
\usepackage{amssymb} % ~ Pour écrire les maths
\usepackage{xspace}  % ~ Commandes à texte
\usepackage{varioref} % ~ Références croisées
\usepackage{enumitem} % ~ Listes
\usepackage{xcolor}   % ~ Couleurs fs
\usepackage{tabularx}
\usepackage{multicol}
\usepackage{float}
\usepackage{tikz}
\usepackage[straightvoltages]{circuitikz}
\usepackage[load-configurations = abbreviations]{siunitx}
\usepackage{graphicx}
\usepackage[f]{esvect}
\usepackage[many]{tcolorbox}
\usepackage{euler}
\usepackage[nointegrals]{wasysym}
\usepackage[french]{babel}


\sisetup{
  locale=FR,
  detect-all,
}

% ______FONCTIONS______%
\newcommand{\ssi}{si et seulement si\xspace}		% ~ ssi
% --ENSEMBLES--%
\newcommand{\N}{\mathbb{N}}   % ~ Entiers naturels
\newcommand{\Z}{\mathbb{Z}}   % ~ Entiers relatifs
\newcommand{\D}{\mathbb{D}}   % ~ Decimaux
\newcommand{\Q}{\mathbb{Q}}   % ~ Rationnels
\newcommand{\R}{\mathbb{R}}   % ~ Réels
\newcommand{\C}{\mathbb{C}}   % ~ Complexes
% --TRIGO--%
\let\cosh\relax
\DeclareMathOperator{\cosh}{ch}       % ~ cosinus hyperbolique
\DeclareMathOperator{\sh}{sh}         % ~ sinus hyperbolique
\let\tanh\relax
\DeclareMathOperator{\tanh}{th}       % ~ tangente hyperbolique
\DeclareMathOperator{\argch}{Argch}   % ~ Argument cosinus hyperbolique
\DeclareMathOperator{\argsh}{Argsh}   % ~ Argument sinus hyperbolique
\DeclareMathOperator{\argth}{Argth}   % ~ Argument tangente hyperbolique
\DeclareMathOperator{\cotan}{cotan}   % ~ cotangente
% --PARENTHESES--%
\newcommand{\po}{\left(}         % ~ (
\newcommand{\pf}{\right)}        % ~ )
\newcommand{\pof}[1]{\po #1 \pf} % ~ ( )
\newcommand{\co}{\left[}         % ~ [
\newcommand{\cf}{\right]}        % ~ ]
\newcommand{\cof}[1]{\co #1 \cf} % ~ [ ]
\newcommand{\chof}[1]{\left\langle #1 \right\rangle } % ~ < >
\newcommand{\interoo}[2]{\left]#1\,;#2\right[}   % ~ ]a,b[
\newcommand{\interof}[2]{\left]#1\,;#2\right]}   % ~ ]a,b]
\newcommand{\interfo}[2]{\left[#1\,;#2\right[}   % ~ [a,b[
\newcommand{\interff}[2]{\left[#1\,;#2\right]}   % ~ [a,b]
% --VECTEURS--%
\newcommand{\vect}[1]{\vv{#1}}
\newcommand{\ux}{\vect{u_x}}          % ~ Vecteur ux
\newcommand{\uy}{\vect{u_y}}          % ~ Vecteur uy
\newcommand{\uz}{\vect{u_z}}          % ~ Vecteur uz
\newcommand{\ur}{\vect{u_r}}          % ~ Vecteur ur
\newcommand{\uth}{\vect{u_\theta}}    % ~ Vecteur utheta
\newcommand{\uph}{\vect{u_\varphi}}   % ~ Vecteur uphi
\newcommand{\urho}{\vv{u_\rho}}       % ~ Vecteur urho
\newcommand{\om}{\vect{OM}}           % ~ Vecteur position
\newcommand{\vvi}{\vect{v}}           % ~ Vecteur vitesse
\newcommand{\vvio}{\vect{v_0}}        % ~ Vecteur v0
\newcommand{\va}{\vect{a}}            % ~ Vecteur	accélération
\newcommand{\vp}{\vect{p}}            % ~ Vecteur quantité de mouvement
\newcommand{\fr}{\vect{F_r}}          % ~ Vecteur force de rappel
\newcommand{\vabla}{\vect{\nabla}}    % ~ nabla
\newcommand{\grad}{\vect{\mathrm{grad}}}  % ~ grad
\DeclareMathOperator{\diverg}{div}        % ~ grad
\newcommand{\rot}{\vect{\mathrm{rot}}}    % ~ grad

\newtcolorbox{cadre}[2][]
{
  enhanced,
  attach boxed title to top left={yshift=-3.4mm, xshift = -2.3mm},
  adjusted title=#2,
  colback=white, colframe=black,
  colbacktitle=white, coltitle=black, fonttitle=\bfseries,
  breakable, sharp corners,
  boxed title style={colback=white, sharp corners, colframe=white},
  boxrule = 0.5mm, drop fuzzy shadow
}
\newcommand{\ooint}{\ocircle\hspace{-3.65mm}\int\hspace{-2mm}\int}

\title{Exercices de mécanique}
\author{Martin \textsc{Andrieux}}
\date{}

\begin{document}
\maketitle

\begin{cadre}{Rembobinage d'un fil}
  \paragraph*{}
  \begin{minipage}{0.6\linewidth}
    Un fil inextensible et sans masse de longueur $L$ est raccordé tangentiellement à une bobine circulaire plate de rayon $R$.
    À son extrémité libre est accroché un point matériel $M$ de masse $m$. Le fil étant tendu, on lance $M$ dans le plan de la bobine, avec une vitesse $\vv{v_{0}}$ prependiculaire au fil, et dans le sens correspondant à l'enroulement.
    On note $\theta$ l'angle correspondant au fil enroulé ($\theta = 0$ à $t=0$)1. On néglige le poids de $M$. Il est conseillé d'utiliser pour les calculs la base $\pof{\urho, \uth}$ (cf. figure).
    \begin{enumerate}[label=\upshape\alph*)]
      \item Quelle relation existe-t-il entre $L$, $R$, $\theta$ et $\rho = IM$?
      \item Calculer la vitesse $\vv v$ et l'accélération $\vv a$ dans la base $\pof{\urho, \uth}$. Faire le lien avec la base de Frénet; quel est le centre de courbure de la trajectoire?
      \item Montrer que le mouvement est uniforome ($v$ constant).
      \item Calculer la durée $\tau$ du rembobinage.
      \item Calculer la tension $\vv T(t)$ du fil en fonction du temps. Commenter.
    \end{enumerate}
  \end{minipage}
  \begin{minipage}{0.35\linewidth}
    \begin{tikzpicture}
      \draw (0,0) circle(2);
      \draw[dashed] (0,0) node[below right]{$O$} -- node[midway, below]{$R$} ++(-2,0) -- node[midway, left]{$L$} ++(0,5) node[name=N]{$\bullet$};
      \draw[densely dashed, thick, ->, >=stealth] (N) node[left]{$M_{0}$} --  node[midway, above]{$\vv{v_{0}}$} ++(2,0);
      \draw (0,0) -- node[midway, right]{$\theta$}(120:2) node[above left]{I} -- ++(30:3.3) node[name=M]{$\bullet$};
      \draw[thick, ->, >=stealth] (M.center) node[above left]{$M$} -- ++(30:2) node[above left]{$\urho$};
      \draw[thick, ->, >=stealth] (M.center) -- ++(-60:2) node[right]{$\uth$};
      \draw[->] (-1,0) arc(180:120:1);
    \end{tikzpicture}
  \end{minipage}
  \tcblower
  \begin{multicols}{2}
    \begin{enumerate}[label=\upshape\alph*)]
      \item $L = \rho + \theta R$
      \item $\vv v = \rho\dot\theta\uth$

        $\vv a = \frac{d}{dt}\pof{\rho\dot\theta}\uth - \rho\theta^2\urho$
      \item $\frac{dE_{c}}{dt}=\mathcal P_{\text{force}} = 0$ d'où $E_{c}$ constante.
      \item $\tau = \dfrac{L^2}{2v_{0}R}$
      \item $T(t) = \dfrac{mv_{0}^{2}}{L\sqrt{1-\frac{t}{\tau}}}$, la tension diverge quand $M$ devient proche de la bobine.
    \end{enumerate}
  \end{multicols}
\end{cadre}

\end{document}
