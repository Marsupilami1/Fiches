\documentclass[french, a4paper, 11pt, twocolumn]{article}

\usepackage[utf8]{inputenc} % ~ Encodage
\usepackage[T1]{fontenc}    % ~ Encodage
\usepackage[left=1cm, right=1cm, bottom=3cm]{geometry} % ~ Mise en page et marges
\usepackage{amssymb} % ~ Pour écrire les maths
\usepackage{xspace}  % ~ Commandes à texte
\usepackage{varioref} % ~ Références croisées
\usepackage{enumitem} % ~ Listes
\usepackage{xcolor}   % ~ Couleurs fs
\usepackage{tabularx}
\usepackage{float}
\usepackage{tikz}
\usepackage[straightvoltages]{circuitikz}
\usepackage[load-configurations = abbreviations]{siunitx}
\usepackage{graphicx}
\usepackage[f]{esvect}
\usepackage[many]{tcolorbox}
\usepackage{euler}
\usepackage[nointegrals]{wasysym}
\usepackage[french]{babel}


\sisetup{
  locale=FR,
  detect-all,
}

%______FONCTIONS______%
\newcommand{\ssi}{si et seulement si\xspace}		% ~ ssi
% --ENSEMBLES--%
\newcommand{\N}{\mathbb{N}}   % ~ Entiers naturels
\newcommand{\Z}{\mathbb{Z}}   % ~ Entiers relatifs
\newcommand{\D}{\mathbb{D}}   % ~ Decimaux
\newcommand{\Q}{\mathbb{Q}}   % ~ Rationnels
\newcommand{\R}{\mathbb{R}}   % ~ Réels
\newcommand{\C}{\mathbb{C}}   % ~ Complexes
% --TRIGO--%
\let\cosh\relax
\DeclareMathOperator{\cosh}{ch}       % ~ cosinus hyperbolique
\DeclareMathOperator{\sh}{sh}         % ~ sinus hyperbolique
\let\tanh\relax
\DeclareMathOperator{\tanh}{th}       % ~ tangente hyperbolique
\DeclareMathOperator{\argch}{Argch}   % ~ Argument cosinus hyperbolique
\DeclareMathOperator{\argsh}{Argsh}   % ~ Argument sinus hyperbolique
\DeclareMathOperator{\argth}{Argth}   % ~ Argument tangente hyperbolique
\DeclareMathOperator{\cotan}{cotan}   % ~ cotangente
% --PARENTHESES--%
\newcommand{\po}{\left(}         % ~ (
\newcommand{\pf}{\right)}        % ~ )
\newcommand{\pof}[1]{\po #1 \pf} % ~ ( )
\newcommand{\co}{\left[}         % ~ [
\newcommand{\cf}{\right]}        % ~ ]
\newcommand{\cof}[1]{\co #1 \cf} % ~ [ ]
\newcommand{\chof}[1]{\left\langle #1 \right\rangle } % ~ < >
\newcommand{\interoo}[2]{\left]#1\,;#2\right[}   % ~ ]a,b[
\newcommand{\interof}[2]{\left]#1\,;#2\right]}   % ~ ]a,b]
\newcommand{\interfo}[2]{\left[#1\,;#2\right[}   % ~ [a,b[
\newcommand{\interff}[2]{\left[#1\,;#2\right]}   % ~ [a,b]
% --VECTEURS--%
\newcommand{\ux}{\vv{u_x}}          % ~ Vecteur ux
\newcommand{\uy}{\vv{u_y}}          % ~ Vecteur uy
\newcommand{\uz}{\vv{u_z}}          % ~ Vecteur uz
\newcommand{\ur}{\vv{u_r}}          % ~ Vecteur ur
\newcommand{\uth}{\vv{u_\theta}}    % ~ Vecteur utheta
\newcommand{\uph}{\vv{u_\varphi}}   % ~ Vecteur uphi
\newcommand{\om}{\vv{OM}}           % ~ Vecteur position
\newcommand{\vvi}{\vv{v}}           % ~ Vecteur vitesse
\newcommand{\vvio}{\vv{v_0}}        % ~ Vecteur v0
\newcommand{\va}{\vv{a}}            % ~ Vecteur	accélération
\newcommand{\vp}{\vv{p}}            % ~ Vecteur quantité de mouvement
\newcommand{\fr}{\vv{F_r}}          % ~ Vecteur force de rappel
\newcommand{\vabla}{\vv{\nabla}}    % ~ nabla
\newcommand{\grad}{\vv{\mathrm{grad}}}  % ~ grad
\DeclareMathOperator{\diverg}{div}        % ~ grad
\newcommand{\rot}{\vv{\mathrm{rot}}}    % ~ grad

\newtcolorbox{cadre}[2][]
{
  enhanced,
  attach boxed title to top left={yshift=-3.4mm, xshift = -2.3mm},
  adjusted title=#2,
  colback=white, colframe=black,
  colbacktitle=white, coltitle=black, fonttitle=\bfseries,
  breakable, sharp corners,
  boxed title style={colback=white, sharp corners, colframe=white},
  boxrule = 0.5mm, drop fuzzy shadow
}
\newcommand{\ooint}{\ocircle\hspace{-3.65mm}\int\hspace{-2mm}\int}

\title{Mécanique}
\author{Martin \textsc{Andrieux}}
\date{}

\begin{document}
\maketitle

\begin{cadre}{Coordonées de Frenet}
  \[\vv a = \dfrac{dv}{dt}\cdot\vv T + \dfrac{v^{2}}{R}\cdot\vv N\]
\end{cadre}

\begin{cadre}{Travail et puissance}
  \[\delta\mathcal T = \vv f\cdot\vv{dM}\]
  \[\mathcal P = \vv F \cdot \vv V\]
\end{cadre}

\section{Changement de référentiel}
\begin{cadre}{Compositions}
  \[\vv v \pof{M/\mathcal R} = \vv v\pof{M/\mathcal R'} + \vv{v_{e}}(M)\]
  \(\vv{v_{e}}\) est la \emph{vitesse d'entraînement}, c'est la vitesse qu'aurait \(M\) dans \((\mathcal R)\) s'il était fixe dans \(\pof{\mathcal R'}\).

  Pour l'accélération, on a de même:
  \[\vv a(M/\mathcal R) = \vv a\pof{M/\mathcal R'} + \vv{a_{e}}(M) + \vv{a_{c}}(M)\]
\end{cadre}

\begin{cadre}{Translation}
  \[\vv{v_{e}}(M) = \vv v\pof{O'/\mathcal R}\]
  \[\vv{a_{e}}(M) = \vv a\pof{O'/\mathcal R}\]
  \[\vv{a_{c}}(M) = \vv 0\]
\end{cadre}

\begin{cadre}{Rotation}
  \[\vv{v_{e}}(M) = R\cdot\Omega\cdot\uth\]
  \[\vv{a_{e}}(M) = -R\cdot\Omega^{2}\cdot\ur\]
  \[\vv{a_{c}}(M) = 2\vv\Omega\wedge\vv v\pof{M/\mathcal R'}\]
\end{cadre}

\section{Rotation autour d'un axe fixe}
\begin{cadre}{Moment d'inertie}
  \[J_{\Delta} = \iiint_{(S)}r^{2}\,dm\]
  Le moment cinétique du solide par rapport à son axe de rotation est:
  \[L_{\Delta} = J_{\Delta}\omega\]
\end{cadre}

\begin{cadre}{Théorème d'Huygens}
  \[J_{\Delta'} = j_{\Delta} + md^{2}\]
  Où \(\Delta\) et \(\Delta'\) sont deux axes parallèles séparés d'une distance \(d\). L'axe \(\Delta\) doit passer par \(G\).
\end{cadre}

\begin{cadre}{Une trivialité}
  Pour un cercle de masse \(m\), de rayon \(R\):
  \[J_{\Delta} = mR^{2}\]
\end{cadre}

\begin{cadre}{Moment d'une force}
  \[\vv{\mathcal M_{O}} = \vv{OM}\wedge \vv f\]
  La méthode du bras de levier est souvent plus rapide.
\end{cadre}

\begin{cadre}{Théorème du moment cinétique}
  \[\dfrac{d\vv L}{dt}=J\ddot\theta=\sum\vv{\mathcal M_{O}}\]
\end{cadre}

\begin{cadre}{Énergie cinétique}
  \[E_{c} = \dfrac{1}{2}J_{\Delta}\omega^{2}\]
\end{cadre}

\begin{cadre}{Théorèmes}
  \[\dfrac{dE_{c}}{dt}=\mathcal P_{\mathrm{ext}}\]
  \[\Delta E_{c}=\mathcal T_{\mathrm{ext}}\]
  \[\dfrac{dE_{m}}{dt}=\mathcal P_{\text{ext, non conservatives}}\]
\end{cadre}

\section{Contact de deux solides}
\begin{cadre}{Lois de Coulomb}
  S'il y a glissement:
  \[R_{T} = \mu_{d}R_{N}\]
  S'il n'y a pas glissement:
  \[R_{T} \leqslant \mu_{s}R_{N}\]
  La plupart du temps, les coefficients de frottement statique et dynamique ne sont pas distingués, on a alors:
  \[\mu_{s} = \mu_{d} = \mu\]
\end{cadre}

\begin{cadre}{Puissance des forces de glissement}
  \[\mathcal P = \vv{R_{T}}\cdot\vv{v_{g}}\leqslant 0\]
\end{cadre}

\end{document}
