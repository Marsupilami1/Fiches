\documentclass[french, a4paper, 11pt, twocolumn]{article}

\usepackage[utf8]{inputenc} % ~ Encodage
\usepackage[T1]{fontenc}    % ~ Encodage
\usepackage[left=1cm, right=1cm, bottom=3cm]{geometry} % ~ Mise en page et marges
\usepackage{amssymb} % ~ Pour écrire les maths
\usepackage{xspace}  % ~ Commandes à texte
\usepackage{varioref} % ~ Références croisées
\usepackage{enumitem} % ~ Listes
\usepackage{xcolor}   % ~ Couleurs fs
\usepackage{tabularx}
\usepackage{float}
\usepackage{tikz}
\usepackage[straightvoltages]{circuitikz}
\usepackage[load-configurations = abbreviations]{siunitx}
\usepackage{graphicx}
\usepackage[f]{esvect}
\usepackage[many]{tcolorbox}
\usepackage{euler}
\usepackage[nointegrals]{wasysym}
\usepackage{array}
\usepackage{babel}


\sisetup{
  locale=FR,
  detect-all,
}

% ______FONCTIONS______%
\newcommand{\ssi}{si et seulement si\xspace}		% ~ ssi
% --ENSEMBLES--%
\newcommand{\N}{\mathbb{N}}   % ~ Entiers naturels
\newcommand{\Z}{\mathbb{Z}}   % ~ Entiers relatifs
\newcommand{\D}{\mathbb{D}}   % ~ Decimaux
\newcommand{\Q}{\mathbb{Q}}   % ~ Rationnels
\newcommand{\R}{\mathbb{R}}   % ~ Réels
\newcommand{\C}{\mathbb{C}}   % ~ Complexes
% --TRIGO--%
\let\cosh\relax
\DeclareMathOperator{\cosh}{ch}       % ~ cosinus hyperbolique
\DeclareMathOperator{\sh}{sh}         % ~ sinus hyperbolique
\let\tanh\relax
\DeclareMathOperator{\tanh}{th}       % ~ tangente hyperbolique
\DeclareMathOperator{\argch}{Argch}   % ~ Argument cosinus hyperbolique
\DeclareMathOperator{\argsh}{Argsh}   % ~ Argument sinus hyperbolique
\DeclareMathOperator{\argth}{Argth}   % ~ Argument tangente hyperbolique
\DeclareMathOperator{\cotan}{cotan}   % ~ cotangente
% --PARENTHESES--%
\newcommand{\po}{\left(}         % ~ (
  \newcommand{\pf}{\right)}        % ~ )
\newcommand{\pof}[1]{\po #1 \pf} % ~ ( )
\newcommand{\co}{\left[}         % ~ [
  \newcommand{\cf}{\right]}        % ~ ]
\newcommand{\cof}[1]{\co #1 \cf} % ~ [ ]
\newcommand{\chof}[1]{\left\langle #1 \right\rangle } % ~ < >
\newcommand{\interoo}[2]{\left]#1\,;#2\right[}   % ~ ]a,b[
\newcommand{\interof}[2]{\left]#1\,;#2\right]}   % ~ ]a,b]
\newcommand{\interfo}[2]{\left[#1\,;#2\right[}   % ~ [a,b[
\newcommand{\interff}[2]{\left[#1\,;#2\right]}   % ~ [a,b]
% --VECTEURS--%
\newcommand{\ux}{\vv{u_x}}          % ~ Vecteur ux
\newcommand{\uy}{\vv{u_y}}          % ~ Vecteur uy
\newcommand{\uz}{\vv{u_z}}          % ~ Vecteur uz
\newcommand{\ur}{\vv{u_r}}          % ~ Vecteur ur
\newcommand{\uth}{\vv{u_\theta}}    % ~ Vecteur utheta
\newcommand{\uph}{\vv{u_\varphi}}   % ~ Vecteur uphi
\newcommand{\om}{\vv{OM}}           % ~ Vecteur position
\newcommand{\vvi}{\vv{v}}           % ~ Vecteur vitesse
\newcommand{\vvio}{\vv{v_0}}        % ~ Vecteur v0
\newcommand{\va}{\vv{a}}            % ~ Vecteur	accélération
\newcommand{\vp}{\vv{p}}            % ~ Vecteur quantité de mouvement
\newcommand{\fr}{\vv{F_r}}          % ~ Vecteur force de rappel
\newcommand{\vabla}{\vv{\nabla}}    % ~ nabla
\newcommand{\grad}{\vv{\mathrm{grad}}}  % ~ grad
\DeclareMathOperator{\diverg}{div}        % ~ grad
\newcommand{\rot}{\vv{\mathrm{rot}}}    % ~ grad

\newtcolorbox{cadre}[2][]
{
  enhanced,
  attach boxed title to top left={yshift=-3.4mm, xshift = -2.3mm},
  adjusted title=#2,
  colback=white, colframe=black,
  colbacktitle=white, coltitle=black, fonttitle=\bfseries,
  breakable, sharp corners,
  boxed title style={colback=white, sharp corners, colframe=white},
  boxrule = 0.5mm, drop fuzzy shadow
}
\newcommand{\ooint}{\ocircle\hspace{-3.65mm}\int\hspace{-2mm}\int}

\title{Optique}
\author{Nathan \textsc{Maillet}}
\date{}

\begin{document}
\maketitle

\section{Ondes lumineuses}
\begin{cadre}{Intensité avec le modèle scalaire}
  Le vecteur de Poynting donne, avec le modèle scalaire :
  \[I=\mathcal{E}\propto \langle s^2\rangle \propto a^2\]
\end{cadre}

\begin{cadre}{Théorème de Mallus}
  Le théorème de Mallus est utile pour trouver la différence de marche.
  Il stipule que les surfaces d'onde sont orthogonales aux rayons lumineux.
\end{cadre}

\section{Interférences}
\begin{cadre}{Formule de Fresnel}
  En écrivant que pour deux ondes \(s_1\) et \(s_2\), \(I=\left\langle(s_1+s_2)^2\right\rangle\), on trouve:
  \[I(M)=I_1+I_2+2\sqrt{I_1I_2}\cos\left(\varphi(S_1)-\varphi(S_2)+\frac{2\pi \delta}{\lambda_0}\right)\]

  \tcblower
  Dans le cas où \(\mathcal{E}_1=\mathcal{E}_2=\mathcal{E}_0\) et \(\varphi(S_1)=\varphi(S_2)\) on a donc :
  \[\mathcal{E}(M)=2\mathcal{E}_0\pof{1+\cos\pof{\frac{2\pi \delta}{\lambda_0}}}\]
\end{cadre}

\begin{cadre}{Contraste ou visibilité}
  On définit le contraste (ou visibilité) par:
  \[C=V=\dfrac{\mathcal{E_{\mathrm{max}}-E_{\mathrm{min}}}}{\mathcal{E_{\mathrm{max}}+E_{\mathrm{min}}}}\]
\end{cadre}

\begin{cadre}{Fentes d'Young}
  Dans le cas des fentes d'Young on a \(\delta=\frac{ax}{D}\) avec a l'écartement entre les fentes, D la distance entre
  les fentes et l'écran et \(x\) la position du point sur l'écran.

  Dans ce cas, on a: \(i=\frac{\lambda_0D}{a}\).
  Si l'on ajoute une lentille avec des franges d'Young à l'infini, on a: \(\delta=\frac{ax}{f}\) et \(i=\frac{\lambda_0f}{a}\).
\end{cadre}

\begin{cadre}{Degré de cohérence temporelle}
  Dans le cas d'une onde polychromatique, on a: \(I\propto 1+\gamma_t\cos(2\pi \delta \sigma_0)\), avec \(\gamma_t\) le degré de cohérence temporelle.

  \tcblower
  Le degré de cohérence temporelle donne l'enveloppe de la courbe \(I(\delta)\).
\end{cadre}

\begin{cadre}{Brouillage}
  Le contraste est maximal lorsque la différence des ordres \(p_1-p_2\) est entier.
  Quand le contraste est nul, il y a anti-coïncidence et les brouillages sont données par \(p_1-p_2=q+\frac{1}{2}\) avec \(q\in \Z\)
  Le premier brouillage est alors donné pour \(p_1-p_2=\delta\Delta\sigma=\frac{1}{2}\)
\end{cadre}

\begin{cadre}{Finesse}
  On définit la finesse par : \(\mathcal{F}=\frac{\nu_0}{\Delta \nu}=\frac{\omega_0}{\Delta \omega}=\frac{\sigma_0}{\Delta \sigma}\)
\end{cadre}

\begin{cadre}{Observation des interférences}
  Pour observer des interférences \(\delta\) doit vérifier : \(|\delta|<c\tau_c=\emph{l}_c\).
  \(\emph{l}_c\) est alors la longueur de cohérence de la source lumineuse.

  \tcblower
  \(\tau_c, \Delta\nu\) et \(\emph{l}_c\) vérifient :
  \[\tau_c \Delta\nu \sim 1 \text{ et } \emph{l}_c \Delta\nu \sim c\]
\end{cadre}

\section{Interféromètre de Michelson}
\begin{cadre}{Différence de marche}
  \(\delta=2ne\cos(i)\) avec \(i\) l'angle incident que fait le rayon en passant par l'image de (\(M_1\)) par la séparatrice.

  \tcblower
  Quand l'on place un verre d'indice \(n\) entre la séparatrice et un des miroirs et en notant \(\delta'\) la nouvelle différence de marche,
  on a \(|\delta-\delta'|=2(n-1)e\) car l'onde passe deux fois par le verre.
\end{cadre}

\begin{cadre}{Vocabulaire relatif au Michelson}
    \begin{tabularx}{\linewidth}{X|X}
      Séparatrice & Lame semi-réfléchissante \\
      \hline
      Division d'amplitude & Conséquence de la séparatrice \\
      \hline
      Compensatrice & Assure l'indépendance entre \(\delta\) et \(\lambda\) \\
      \hline
      Lame d'air & Espace entre (\(M'_1\)) et (\(M_2\)) \\
      \hline
      Non localisé & Observables dans tout l'espace \\
      \hline
      Contact optique & Lame d'air d'épaisseur nulle \\
      \hline
      Éclairement uniforme & \(\delta(M)=0\) \\
      \hline
      Franges d'égale inclinaison& \(i=\mathrm{cste}\)\\
      \hline
      Franges d'égale épaisseur & Lame d'air constante
    \end{tabularx}
\end{cadre}

\section{Diffraction}
\begin{cadre}{Ordres de grandeurs}
  La lumière diffractée est concentrée dans des directions limités par rapport à la direction de l'onde incidente par le demi-angle \(\theta\sim \frac{\lambda}{a}\).

  Le rayon de la tache d'Airy est donné par:
  \(\sin(\theta)=\frac{0,61\lambda}{R}\)
\end{cadre}

\begin{cadre}{Critère de Rayleigh}
  Deux taches de diffraction sont séparées si le maximum central de l'une est au-delà
  du premier minimum de l'autre.
\end{cadre}

\section{Réseaux de diffraction}
Soit \(N\) le nombre de motifs, \(h\) le pas et \(L=Nh\) la largeur du réseau.

\begin{cadre}{Maxima principaux}
  Les maxima principaux sont atteints pour un déphasage entre deux motifs successifs
  \(\varphi=0[2\pi]\) ou encore \(\delta=p\lambda, p \in \Z\).
  Sur la courbe \(I(\varphi)\), les maxima principaux sont étroits avec :
  \begin{itemize}
    \item \(N-1\) annulations de l'Intensité
    \item \(N-2\) maxima secondaires (jamais observés en pratique quand \(N\) est grand)
  \end{itemize}
\end{cadre}

\begin{cadre}{Relation des réseaux}
  Relation des réseaux en prenant les angles dans le sens trigonométrique. \(i\) est l'angle de déviation.
  Pour un réseau par transmission, on a:
  \[h(\sin(\theta)-\sin(i))=p\lambda=\delta, p \in \Z\]
  Pour un réseau par réflexion, on a:
  \[h(\sin(\theta)+\sin(i))=p\lambda=\delta, p \in \Z\]
  \tcblower
  Avec d'autres conventions d'orientations pour les angles, les signes \og \(+\)\fg{} et \og \(-\)\fg{} sont inversés.
\end{cadre}

\begin{cadre}{Pouvoir de résolution}
  Le pouvoir de résolution du réseau est défini par
  \(\mathcal{R}=\frac{\lambda}{\Delta\lambda_{\text{min}}}\)
  On peut donc écrire:
  \[\mathcal{R}=\frac{\lambda}{\Delta\lambda_{\text{min}}}=\lvert p\rvert\cdot N\]
  \tcblower
  En écrivant la limite du critère de Rayleigh pour \(p>0\), on a:
  \[\delta=\pof{p+\frac{1}{N}}\lambda=p\pof{\lambda + \Delta\lambda_{\text{min}}}\]
  ce qui donne l'écriture finale du pouvoir de résolution.
\end{cadre}
\end{document}
