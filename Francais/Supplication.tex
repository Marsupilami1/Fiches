

% Toute citation sans thème ni commentaire est une citation dont on a parler en cours mais dont je n'ai pas noter le commentaire...
% Tout commentaire est donc le bienvenue


\documentclass[french,a4paper,11pt,answers]{exam}

\usepackage[utf8]{inputenc} 
\usepackage[T1]{fontenc}
\usepackage[left=1cm, right=1cm]{geometry}
\usepackage[many]{tcolorbox}
\usepackage[french]{babel}

\newcommand{\cit}[2]{\og #1 \fg{} \begin{solution}{ #2 }\end{solution}} % Ajoute les guillemets puis les commentaires

\usepackage{color} % définit une nouvelle couleur
\shadedsolutions % définit le style de la case thèmes et commentaires
\definecolor{SolutionColor}{rgb}{0.8,0.9,1} % bleu ciel
\renewcommand{\solutiontitle}{\noindent\textbf{Thèmes et commentaires :}\par\noindent} % Définit le titre des encadrés bleu

\newtcolorbox{cadre}[2][]
{
  enhanced,
  attach boxed title to top left={yshift=-3.4mm, xshift = -2.3mm},
  adjusted title=#2,
  colback=white, colframe=black,
  colbacktitle=white, coltitle=black, fonttitle=\bfseries,
  breakable, sharp corners,
  boxed title style={colback=white, sharp corners, colframe=white},
  boxrule = 0.5mm, drop fuzzy shadow
}

\title{Citations \emph{La Supplication} de Svetlana Alexievitch}
\author{Maillet Nathan\\MP*}
\date{}

\begin{document}
	\maketitle

	\begin{cadre}{Une voix solitaire}
		\begin{itemize}
			\item \cit {Les brûlures remontaient à la surface\ldots Dans la bouche, sur la langue, les joues\ldots D'abord, ce ne furent que de petits chancres, puis ils s'élargirent\ldots La muqueuse se décollait par couches\ldots En pellicules blanches\ldots La couleur du visage\ldots La couleur du corps\ldots Bleu\ldots Rouge\ldots Gris-brun\ldots Et tout cela m'appartient, et tout cela est tellement aimé ! On ne peut pas le raconter ! On ne peut pas l'écrire !} %citation
				{Mort, Souffrance, Indicible, Amour} %Thèmes/Commentaires
			\item \cit {Ils meurent, mais personne ne les a véritablement interrogés sur ce que nous avons vécu\ldots Les gens n'ont pas envie d'entendre parler de la mort. De l'horrible.}
				{Oubli, Mort, Déni, Rapport à l'autre}
		\end{itemize}
	\end{cadre}
	
	\begin{cadre}{Interview de l'auteur par elle-même sur l'histoire manquée}
		\begin{itemize}
			\item \cit {Tchernobyl est un mystère qu'il nous faut encore élucider. C'est peut-être une tâche pour le $XXI^e$ siècle. Un défi pour ce nouveau siècle. Ce que l'homme a appris, deviné, découvert sur lui-même et dans son attitude envers le monde. Reconstituer les sentiments et non les événements.}
				{Indicible, Nature Humaine, Métaphysique, Création}
			\item \cit{Mes interlocuteurs m'ont souvent tenu des propos similaires: \og Je ne peux pas trouver de mots pour dire ce que j'ai vu et vécu\ldots Je n'ai lu rien de tel dans aucun livre et je ne l'ai pas vu au cinéma\ldots Personne ne m'a jamais raconté des choses semblables à celles que j'ai vécues. \fg{} De tels aveux se répétaient et, volontairement, je n'ai pas reitré ces répétitions de mon livre.}
				{Indicible, Création}
		\end{itemize}
	\end{cadre}
	
	\begin{cadre}{Monologue sur la nécessité du souvenir}
		\begin{itemize}
			\item \cit {Alors, pourquoi les gens se souviennent-ils ? Pour rétablir la vérité ? La justice ? Se libérer et oublier ? Parce qu'ils comprennent qu'ils ont participé à un événement hors du commum ? Cherchent-ils à se réfugier dans le passé ? Mais les souvenirs sont fragiles, éphémères, ils ne forment pas un savoir exact, mais plutôt ce que l'homme devine sur lui-même. Ce ne sont pas encore des connaissances, seulement des émotions.}
				{Souvenir, Oubli, Nature Humaine, Indicible}
		\end{itemize}
	\end{cadre}
	
	\begin{cadre}{Monologue sur ce dont on peut parler avec les vivants et les morts}
		\begin{itemize}
			\item \cit {Tout vit ici. Absoluement tout ! Le lézard vit, la grenouille vit. Et le ver de terre vit. Et il y a des souris ! Tout y est !}
				{Énergie cosmique, Nature}
		\end{itemize}
	\end{cadre}
	
	\subparagraph{Monologue sur l'homme qui n'est raffiné que dans le mal{,}}
	\begin{cadre}{mais simple et accessible dans les mots tout bêtes de l'amour}
		\begin{itemize}
			\item \cit {Et je vais vous dire autre chose: les oiseaux, les arbres, les fourmis sont plus proches de moi qu'auparavant}
				{Nature, Rapport au Monde}
		\end{itemize}
	\end{cadre}
	
	\begin{cadre}{Le c\oe{}ur des soldats}
		\begin{itemize}
			\item \cit {Je me tais. Personne ne trouve les mots qui me feraient répondre. Dans ma langue à moi\ldots Personne ne comprend d'où je suis revenu\ldots Et il m'est impossible de le raconter !}
				{Indicible, Rapport à l'autre}
		\end{itemize}
	\end{cadre}

	\begin{cadre}{Monologue à propos d'un paysage lunaire}
		\begin{itemize}
			\item \cit {Je me suis soudain mis à avoir des doutes. Que valait-il mieux: se souvenir ou oublier ? J'ai posé cette question à des amis. Les uns ont oublié, les autres ne veulent pas se souvenir parce qu'on n'y peut rien changer.}
				{Souvenir et oubli, Déni}
		\end{itemize}
	\end{cadre}
	
	\begin{cadre}{Monologue sur un témoin qui avait mal aux dents et qui a vu Jésus tomber et gémir}
		\begin{itemize}
			\item \cit {L'écrivain Leonid Andreïev, que j'aime beaucoup, a une parabole sur Lazare qui a regardé derrière le trait de l'interdit. Après cela, il est devenu étranger parmi les siens, même si Jésus l'a ressucité\ldots}
				{Vie et mort, Croyance, Métaphysique}
			\item \cit{La vie est une lutte. Il faut toujours surmonter quelque chose. C'est de là que vient notre amour pour les inondations, les incendies, les tempêtes. Nous avons besoin de lieux pour “manifester du courage et de l'héroïsme''.}
				{Lutte, Risque et souffrance}
		\end{itemize}
	\end{cadre}
		
	\begin{cadre}{Monologue sur ce que saint François prêchait aux oiseaux}
		\begin{itemize}
			\item \cit{L'humour était notre seule planche de salut. On racontait des blagues sans arrêt.}
				{Gaieté, Rapport à l'autre}
			\item \cit {On y attrapait des saboteurs et des espions. La passion ! La chasse ! C'est pour cela que nous sommes faits. Si l'on a tous les jours du travail et de quoi manger à satiété, cela devient incorfortable, ennuyeux.}
				{Pour vivre il faut vivre dangereusement}
		\end{itemize}
	\end{cadre}
	
	\begin{cadre}{Monologue à deux voix pour un homme et une femme}
		\begin{itemize}
			\item \cit{On ne peut ni les étonner ni les rendre heureux. Ils sont toujours somnolents, fatigués. Ils sont pâles, et même gris. Ils ne jouent pas, ne s'amusent pas.}
				{}
			\item \cit{L'impensable s'est produit: les gens se sont mis à vivre comme avant. Renoncer aux concombres de son potager était plus grave que Tchernobyl.}
				{Oubli, Déni}
		\end{itemize}
	\end{cadre}
	
	\begin{cadre}{Monologue sur une chose totalement inconnue qui rampe et se glisse à l'intérieur de soi}
		\begin{itemize}
			\item \cit{Nous sommes tous des fatalistes. Nous n'entreprennons rien parce que nous croyons que rien ne peut changer.}
				{Dieu, Lutte, Révolte}
		\end{itemize}
	\end{cadre}
	
	\begin{cadre}{Monologue sur ce regret du rôle et du sujet}
		\begin{itemize}
			\item \cit{L'homme qui se sacrifie ne se percevrait pas comme un être unique et exceptionnel. En fait, il souhaiterait simplement avoir un rôle et passer de simple figurant à personnage principal. Il s'agirait également d'une quête de sens. Notre propagande aurait proposé la mort comme moyen de donner un sens à la vie. Elle donnerait une grande valeur à la mort, parce qu'elle préfigurerait l'éternité}
				{}
			\item \cit{Le prêtre disait l'office des morts. Il était tête nue  “Vous n'avez pas froid ?'' lui ai-je demandé, après.  “Non ! m'a-t-il répondu. Dans de tels moments, je me sens tout-puissant. Aucune cérémonie religieuse ne me donne autant d'énergie que l'office des morts.''}
				{Vie et mort, Énergie, Dieu, Métaphysique}
		\end{itemize}
	\end{cadre}
	
	\begin{cadre}{Monologue sur la physique{,} dont nous étions tous amoureux}
		\begin{itemize}
			\item \cit{Je remarques soudain chaque feuille, la couleur vive des fleurs, le ciel brillant, l'asphalte d'un gris éclatant et, dans ses fleurs, le ciel brillant, mais qui s'affairent. Je penses: “Non, il faut les contourner.'' J'ai pitié d'elles. Pourquoi faudrait-il qu'elles meurent ? Et l'odeur ! L'odeur de la forêt me donne le vertige\ldots Je la perçois encore plus fortement que la couleur. Les bouleaux si légers, les sapins si lourds\ldots Et je ne verrai plus tout cela ? Vivre une minute, une seconde de plus ! Pourquoi ai-je perdu tant d'heures et de jours devant la télé ou un tas de journaux ? Le principal, c'est la vie et la mort.}
				{Vie et mort, Nature, Devenir, Rapport au monde}
		\end{itemize}
	\end{cadre}
	
	\begin{cadre}{Monologue sur la liberté et le rêve d'une mort ordinaire}
		\begin{itemize}
			\item \cit{La peur et la liberté ! Nous respirions pleinement. Vous autres qui avez des vies ordinaires, vous ne pouvez pas le concevoir\ldots}
				{Pour vivre il faut vivre dangereusement}
			\item \cit{Les mots “grandiose'' ou “fantastique'' ne parviennent pas à tout retranscrire. Je n'ai jamais éprouvé un tel sentiment, même pendant l'amour\ldots}
				{Énergie, Lutte, Indicible}
		\end{itemize}
	\end{cadre}

	\begin{cadre}{Monologue sur ce qu'il faut ajouter à la vie quotidienne pour la compnredre}
		\cit {Combien de temps croyez-vous que nous avons conservé cela en mémoire ? A peine quelques jours. (...) Bien sûr nous buvions comme des trous. Le soir, plus personne n'était sobre.}
		{Usage de la vodka chez les liquidateurs pour oublier}
	\end{cadre}
	
	\begin{cadre}{Monologue sur l'éternel et le maudit: que faire et qui est coupable ?}
		\begin{itemize}
			\item \cit{L'homme sans idéal ? C'est horrible\ldots Que voyons-nous se passer, maintenant ? La débâcle. L'anarchie. Les idéaux sont indispensables\ldots}
				{Dieu, Métaphysique}
		\end{itemize}
	\end{cadre}
	
	\begin{cadre}{Monologue sur le pouvoir démesuré d'un homme sur un autre}
		\begin{itemize}
			\item \cit{Un complot de l'ignorance et du corporatisme.}
				{}
		\end{itemize}
	\end{cadre}
	
	\begin{cadre}{Une autre voix solitaire}
		\begin{itemize}
			\item \cit{Alors, nous l'attendrons ensemble. Je réciterai en chuchotant ma supplication pour Tchernobyl et lui, il regardera le monde avec des yeux d'enfants}
				{Devenir, Croyance, Rapport au monde, Souffrance}
			\item \cit{Ce qui m'a sauvée ? Ce qui m'a rendue à la vie ? Mon fils...}
				{Le malheur ou le manque de l'autre peut être dépassé grâce à la présence d'un autre, à même de ramener celui qui souffre à la vie}
		\end{itemize}
	\end{cadre}
\end{document}
