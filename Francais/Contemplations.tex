

% Toute citation sans thème ni commentaire est une citation dont on a parler en cours mais dont je n'ai pas noter le commentaire...
% Tout commentaire est donc le bienvenue


\documentclass[french,a4paper,11pt,answers]{exam}

\usepackage[utf8]{inputenc} 
\usepackage[T1]{fontenc}
\usepackage[left=1cm, right=1cm]{geometry}
\usepackage[many]{tcolorbox}
\usepackage[french]{babel}

\newcommand{\cit}[2]{\og #1 \fg{} \begin{solution}{ #2 }\end{solution}} % Ajoute les guillemets puis les commentaires

\usepackage{color} % définit une nouvelle couleur
\shadedsolutions % définit le style de la case thèmes et commentaires
\definecolor{SolutionColor}{rgb}{0.8,0.9,1} % bleu ciel
\renewcommand{\solutiontitle}{\noindent\textbf{Thèmes et commentaires :}\par\noindent} % Définit le titre des encadrés bleu

\newtcolorbox{cadre}[2][]
{
  enhanced,
  attach boxed title to top left={yshift=-3.4mm, xshift = -2.3mm},
  adjusted title=#2,
  colback=white, colframe=black,
  colbacktitle=white, coltitle=black, fonttitle=\bfseries,
  breakable, sharp corners,
  boxed title style={colback=white, sharp corners, colframe=white},
  boxrule = 0.5mm, drop fuzzy shadow
}

\title{Citations \emph{Les Contemplations} de Victor Hugo}
\author{Maillet Nathan\\MP*}
\date{}

\begin{document}
	\maketitle
	\section{Livre quatrième \og Pauca Meae \fg}
	\begin{cadre}{III. Trois ans après}
		\begin{itemize}
			\item \cit{Si ce Dieu n'a pas voulu clore / L'\oe{}uvre qu'il me fit commencer, / S'il veut que je travaille encore, / Il n'avait qu'à me la laisser !} %citation
				{Thèmes : Révolte, Croyance, Mort. Dans tout ce poème Hugo cherche la quiétude, qui s'assimile à un désir de mort.} %Thèmes/Commentaires
		\end{itemize}
	\end{cadre}
	
	\begin{cadre}{IV}
		\begin{itemize}
			\item \cit{Pères, mères, dont l'âme a souffert ma souffrance, / Tout ce que j'éprouvais, l'avez-vous éprouvé ?\vspace{-5mm}}
				{Universalité, Rapport à l'autre, Souffrance}
			\item \cit{Je voulais me briser le front sur le pavé; / Puis je me révoltais, et, par moments, terrible, / Je fixais mes regards sur cette chose horrible, / Et je n'y croyais pas, et je m'écriais : Non !}
				{Révolte, Nature, Déni, Souffrance}
		\end{itemize}
	\end{cadre}
		
	\begin{cadre}{V}
		\begin{itemize}
			\item \cit{J'appelais cette vie être content de peu ! / Et dire qu'elle est morte ! hélas ! que Dieu m'assiste ! / Je n'étais jamais gai quand je la sentais triste}
				{Rapport à l'autre, Dieu, Regret, Mort}
		\end{itemize}
	\end{cadre}
	
	\begin{cadre}{VIII}
		\begin{itemize}
			\item \cit{À qui donc sommes-nous ? Qui nous a ? qui nous mène ? / Vautour fatalité, tiens-tu la race humaine ?}
				{Mort, Doute, Métaphysique}
		\end{itemize}
	\end{cadre}

	\begin{cadre}{IX}
		\begin{itemize}
			\item \cit{Alors, prodiguant les carnages, / J'inventais un conte profond / Dont je trouvais les personnages / Parmis les ombres du plafond.}
				{L'invention et l'écriture viennent de l'ombre, de la réalité}
		\end{itemize}
	\end{cadre}
	
	\begin{cadre}{X}
		\begin{itemize}
			\item \cit{Moi, je cherche autre chose en ce ciel vaste et pur. / Mais que ce saphir sombre est un abîme obscur !}
				{Doute métaphysique, Quête poétique}
		\end{itemize}
	\end{cadre}
	
	\begin{cadre}{XI}
		\begin{itemize}
			\item \cit{Tout vient et passe; on est en deuil, on est en fête}
				{Mort, Lutte, Devenir}
		\end{itemize}
	\end{cadre}
	
	\begin{cadre}{XII}
		\begin{itemize}
			\item \cit{Hermann reprit alors : \og Le malheur, c'est la vie. les morts ne souffrent plus. Ils sont heureux ! \fg}
				{Souffrance, Vie et mort}
		\end{itemize}
	\end{cadre}
	
	\begin{cadre}{XIII}
		\begin{itemize}
			\item \cit{Je ne daigne plus même, en ma sombre paresse, / Répondre à l'envieux dont la bouche me nuit. / Ô Seigneur ! ouvrez-moi les portes de la nuit, / Afin que je m'en aille et que je disparaisse !}
				{Dieu, Vie et mort}
		\end{itemize}
	\end{cadre}
	
	\begin{cadre}{XIV}
		\begin{itemize}
			\item \cit{Je marcherai les yeux fixés sur mes pensées, / Sans rien voir au dehors, sans entendre aucun bruit, / Seul, inconnu, le dos courbé, les mains croisées, / Triste, et le jour pour moi sera comme la nuit.}
				{Deuil, Rapport au monde}
		\end{itemize}
	\end{cadre}
	
	\begin{cadre}{XV. À Villequier}
		\begin{itemize}
			\item \cit{Maintenant que du deuil qui m'a fait l'âme obscure je sors, pâle et vainqueur, / Et que je sens la paix de la grande nature qui m'entre dans le c\oe{}ur}
				{Deuil, Lutte, Homme et Nature}
		\end{itemize}
	\end{cadre}
	
	\begin{cadre}{XVI. Mors}
		\begin{itemize}
			\item \cit{Et les femmes criaient: - Rends-nous ce petit être. Pour le faire mourir, pourquoi l'avoir fait naître ?}
				{Révolte, Mort, Universalité}
		\end{itemize}
	\end{cadre}
	
	\begin{cadre}{XVII. Charles Vacquerie}
		\begin{itemize}
			\item \cit{Vivez ! aimez ! ayez les bonheurs infinis, / Oh ! les anges pensifs, bénissant et bénis, savent seuls, sous les sacrés voiles, / Ce qu'il entre d'extase, et d'ombre, et de ciel bleu, / Dans l'éternel baiser de deux âmes que Dieu tout à coup change en deux étoiles !}
				{Dieu, Vie et mort}
		\end{itemize}
	\end{cadre}

	\section{Livre cinquième \og En Marche \fg}
	
	\begin{cadre}{III. Écrit en 1846}
		\begin{itemize}
			\item \cit{Le passé ne veut pas s'en aller. Il revient sans cesse sur ses pas, reveut, reprend, revient, use à tout ressaisir ses ongles noirs; fait rage; il gonfle son vieux flot, souffle son vieil orage, vomit sa vieille nuit, cire : À bas ! crie : À mort ! Pleure, tonne, tempête, éclate, hurle, mord. L'avenir souriant lui dit : Passe, bonhomme}
				{Temporalité, Devenir, Oubli}
		\end{itemize}
	\end{cadre}
	
	\begin{cadre}{V. À Mademoiselle Louise B.}
		\begin{itemize}
			\item \cit{Ingrates ! vous n'avez ni regrets, ni mémoire. / Vous vous réjouissez dans toute votre gloire; vous n'avez point pâli. / Ah ! je ne suis qu'un homme et qu'un roseau qui ploie, / Mais je ne voudrais pas, quant à moi, d'une joie faite de tant d'oubli !}
				{Deuil, Oubli, Homme et Nature}
		\end{itemize}
	\end{cadre}
	
	\begin{cadre}{X. Aux feuillantines}
		\begin{itemize}
			\item \cit{Nous lûmes tous les trois ainsi tout le matin, Joseph, Ruth et Booz, le bon Samaritain, et, toujours plus charmés, le soir nous le relûmes.}
				{Vocation édifiante, la Bible aide les hommes à penser} 
		\end{itemize}
	\end{cadre}
	
	\begin{cadre}{XI. Ponto}
		\begin{itemize}
			\item \cit{Toujours l'homme en sa nuit trahi par ses veilleurs ! Toutes les grandes mains, hélas ! de sang rougie ! Alexandre ivre et fou, César perdu d'orgies [...] Ô triste humanité, je fuis dans la nature ! Et pendant que je dis \og tout est leurre, imposture, mensonge, iniquité, mal de splendeur vêtu ! \fg{} Mon chien Ponto me suit. Le chien, c'est la vertu qui, ne pouvant se faire homme, s'est faite bête. Et Ponto me regarde avec son \oe{}il honnête}
				{}
		\end{itemize}
	\end{cadre}
	
	\begin{cadre}{XII. Dolorosæ}
		\begin{itemize}
			\item \cit{Mère, nous n'avons pas plié, quoique roseaux, ni perdu la bonté vis-à-vis l'un de l'autre, ni demandé la fin de mon deuil et du vôtre à cette lâcheté qu'on appelle l'oubli}
				{Amour, Deuil}
		\end{itemize}
	\end{cadre}
	
	\begin{cadre}{XIII. Paroles sur la dune}
		\begin{itemize}
			\item \cit{Ne verrai-je plus rien de tout ce que j'aimais ? Au dedans de moi le soir tombe. / Ô terre, dont la brume efface les sommets, suis-je le spectre et toi la tombe ?}
				{Mort, Mélancolie, Paysage, État d'âme}
		\end{itemize}
	\end{cadre}
	
	\begin{cadre}{XIV. Claire P.}
		\begin{itemize}
			\item \cit{Claire, tu dors. Ta mère, assise sur ta fosse, dit : - Le parfum des fleurs est faux, l'aurore est fausse, l'oiseau qui chante au bois ment, et le cygne ment, l'étoile n'est pas vraie au fond du firmament, le ciel n'est pas le ciel et là-haut rien ne brille}
				{Mort, Homme et Nature, Doute métaphysique}
		\end{itemize}
	\end{cadre}
	
	\begin{cadre}{XVII. Mugitusque Boum}
		\begin{itemize}
			\item \cit{Vis, bête; vis, caillou; vis, homme; vis, buisson !}
				{Nature, Homme et Nature}
		\end{itemize}
	\end{cadre}
	
	\begin{cadre}{XXVI. Les malheureux}
		\begin{itemize}
			\item \cit{je vais méditant, et toujour un instinct me ramène à connaître le fond de la souffrance humaine. L'abîme des douleurs m'attire. D'autres sont les sondeurs frémissants de l'océan profond; ils fouillent, vent des cieux, l'onde que tu balaies; ils plongent dans les mers; je plonge dans les plaies}
				{Rapport à l'autre, Quête poétique}
		\end{itemize}
	\end{cadre}
	
\end{document}
